\startexercises [title={\getvariable{document}{author}  \hfill  \getvariable{document}{course}  \hfill  2019-09-10}]
	
	\startproposition [title={Exercise 1.8}]
		For any sets \m{A} and \m{B}, we have \m{A ∩ B ⊆ A}.
	\stopproposition
	\startproof
		Let \m{x ∈ A ∩ B} be arbitrary. This means \m{x ∈ A} and \m{x ∈ B}. Therefore \m{x ∈ A}. Since every element in \m{A ∩ B} is also an element of \m{A}, we have \m{A ∩ B ⊆ A}.
	\stopproof

	\startproposition [title={Exercise 1.10}]
		For any set \m{A}, we have \m{A ∩ ∅ = ∅}.
	\stopproposition
	\startproof
		\bold{(\m{⊆})}
		Let \m{x ∈ A ∩ ∅} be arbitrary. This means \m{x ∈ A} and \m{x ∈ ∅}. But there does not exist \m{x ∈ ∅}. Therefore, the statement is vacuously true.

		\bold{(\m{⊇})}
		Now, let \m{x ∈ ∅} be arbitrary. Again, since there does not exist \m{x ∈ ∅}, the statement vacuously true.
	\stopproof

	\startproposition [title={Exercise 1.13}]
		For any sets \m{A} and \m{B}, if \m{A ⊆ B}, then \m{A ∪ B = B}.
	\stopproposition
	\startproof
		\bold{(\m{⊆})}
		Let \m{x ∈ A ∪ B} be arbitrary. This means \m{x ∈ A} or \m{x ∈ B}. If \m{x ∈ A}, then by the condition \m{A ⊆ B}, we obtain \m{x ∈ B}. Therefore, in either case, \m{x ∈ B}.

		\bold{(\m{⊇})}
		Let \m{x ∈ B} be arbitrary. Therefore, \m{x ∈ A} or \m{x ∈ B}. Hence \m{x ∈ A ∪ B}.
	\stopproof

\stopexercises