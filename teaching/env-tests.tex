% macros=mkvi    % https://wiki.contextgarden.net/MkVI

\startenvironment *

\setuppapersize[letter]

\usecolors [x11]    % X11 colors, use \showcolor[x11] to get a list

%% Header and footer
% \setupheadertexts[\framed{\tfd Name: } \hfill \getvariable{document}{date}]    % Possibility 1

%% Possibility 2
% \setupheader
%   [text]
%   [before={\startframed[frame=off, bottomframe=on, framecolor=\getvariable{document}{color-headings}]},
%    after={\stopframed}]
% \setupheadertexts[]    % Clear the header
% \setupheadertexts[\tfb \sc Name: ][{\tfc MATH 1550} \qquad \getvariable{document}{date}]

%% Possibility 3
\startsetups[headertext]
	\startframed[frame=on, corner=03, width=\hsize, frameoffset=4pt, framecolor=\getvariable{document}{color-headings}, align=middle]
		% \blank[small]
		{\tfa MATH 1550}  \qquad \qquad {\tfb \getvariable{document}{assessment}}  \qquad \qquad  {\tfa \getvariable{document}{date}}
		\blank[medium]
		{\tfc \sc Name: \color[\getvariable{document}{color-background}]{Print your name here in capitals}}    % Dummy text to left align
		% \blank[small]
	\stopframed
	\blank[small]
\stopsetups
\setupheadertexts[\directsetup{headertext}]


% \define[2]\test{
% 	\framed [
% 		frame=on,
% 		corner=03,
% 		width=\hsize,
% 		frameoffset=4pt,
% 		framecolor=\getvariable{document}{color-headings},
% 		align=middle]
% 	% {#1\\#2}}
% 	{\vbox{\headtext{MATH 1550} #1\blank#2}}}

% \setuphead [chapter] [
% 	command=\test,
% 	color=\getvariable{document}{color-headings},
% 	alternative=margin,
% 	style=\tfc\smallcaps]

% \definehead [test] [chapter] [
% 	before=\hairline\blank,
% 	after=\nowhitespace\hairline]



%% Mathematics

% I have not completely figured out a way to create a dot between \cdot and \bullet
% The following might provide a way, but is too primitive.
% The problem is that the scaling is about the bottom of the line.
\def \argdotsub{\scale[factor=2]{∙}}    % wiki/Command/scale
\def \argdotmid{\raisebox{0.125em}\vbox{\scale[factor=2]{∙}}}
\def \argdotsup{\raisebox{0.125em}\vbox{\scale[factor=2]{∙}}}
% \def \argdot{\raisebox{-0.2em}\vbox{\scale[factor=6]{⋅}}}
% \def \argdotu{\raisebox{-0em}\vbox{\scale[factor=6]{⋅}}}
% \def \argdot{\raisebox{-0.2em}\vbox{\scale[factor=6]{⋅}}}
% \scale[scale=2]{\cdot}} + Z(\scale[sx=3, sy=2]{⋅})

% Caution: spaces do not work between the elements)
\def \delim#left#right#arguments{\left#left #arguments \right#right}

% Intervals
\def \intoo#arguments{\delim{(}{)}{#arguments}}
\def \intoc#arguments{\delim{(}{]}{#arguments}}
\def \intco#arguments{\delim{[}{)}{#arguments}}
\def \intcc#arguments{\delim{[}{]}{#arguments}}

% Brackets
\def \brnd#arguments{\delim{(}{)}{#arguments}}
\def \bcrl#arguments{\delim{\{}{\}}{#arguments}}
\def \bsqr#arguments{\delim{[}{]}{#arguments}}
\def \bngl#arguments{\delim{⟨}{⟩}{#arguments}}

% Commonly used delimiters
\def \inn#arguments{\delim{⟨}{⟩}{#arguments}}
\def \pair#arguments{\delim{(}{)}{#arguments}}
\def \norm#arguments{\delim{‖}{‖}{#arguments}}
\def \abs#arguments{\delim{|}{|}{#arguments}}
\def \floor#arguments{\delim{⌊}{⌋}{#arguments}}
\def \ceil#arguments{\delim{⌈}{⌉}{#arguments}}

% Logic
\def \ntuple#arguments{\delim{⟨}{⟩}{#arguments}}
\def \logicimpl[#structure][#satisfaction]#arguments{⊨_{#structure} #arguments \ \bsqr{#satisfaction}}
\def \nlogicimpl[#structure][#satisfaction]#arguments{⊭_{#structure} #arguments \ \bsqr{#satisfaction}}

% Algebra
\def \linspan#arguments{\mfunction{span} \delim{\{}{\}}{#arguments}}
\def \inv#arguments{#arguments^{-1}}

% Differentials  (We cannot use space before differentials here because we might want to use μ(\d x).)
\def \d{\text{d}}
\def \D{\text{D}}

% Topology
\def \clsr#set{\overline{#set}}
\def \intr#set{{#set}^°}
% \def \intr[#set]{\mathring{#set}}
\def \bdy[#set]{∂ #set}

% Limits
\definemathcommand [ess] [limop] {\mfunction{ess}}
\definemathcommand [limsupm] [limop] {\overline{\lim}}
\definemathcommand [liminfm] [limop] {\underline{\lim}}

% Algebra
\definemathcommand [ker] [nolop] {\mfunction{ker}}
\definemathcommand [im] [nolop] {\mfunction{im}}

% Arithmetic
\definemathcommand [gcd] [nolop] {\mfunction{gcd}}
\definemathcommand [lcm] [nolop] {\mfunction{lcm}}
\definemathcommand [dim] [limop] {\mfunction{dim}}
% \def \congruent[#modulus]{≡_{#modulus}}

% Probability
\definemathcommand [infinitelyoften] [nolop] {\,\mfunction{i.o.}}
\definemathcommand [eventually] [nolop] {\,\mfunction{ev}}

% Matrices
\definemathmatrix
	[rndmatrix]
	[left={\,\left(\,},right={\,\right)\,}]


% Styling for mathematical environments

%  Randomized frame
\startuseMPgraphic{background:mathematicalargumentframe}
	path p;
	for i = 1 upto nofmultipars :
		p = (multipars[i]  topenlarged 1pt  bottomenlarged 1pt)  randomized 1pt;
		fill p withcolor green + 0.9375 * white;    % hard to change (works: red, green, blue, white, lightgray + linear combinations)
		draw p withcolor \MPvar{linecolor}
		withpen pencircle scaled \MPvar{linewidth};
	endfor;
\stopuseMPgraphic

\definetextbackground [mathematicalargumentframe] [
	mp=background:mathematicalargumentframe,
	location=paragraph,
	rulethickness=0.5pt,
	framecolor=green,    % hard to change (works: red, green, blue, white, lightgray)
	width=broad,
	leftoffset=4pt,
	rightoffset=4pt,
	before={\testpage[3]\blank[small]},
	after={\blank[small]}]


% Highlighting
\definehighlight [comment] [color=darkgray]
\definehighlight [good] [color=darkgreen]
\definehighlight [okay] [color=darkorange]
\definehighlight [ugly] [color=orangered]


%% Mathematical environments

\definereferenceformat [eqref] [left=(, right=)]    % Referencing formulas

\defineenumeration [mathematicalstatement] [
	title=yes,    % now we can use title={}
	% number=no,    % uncomment if statements shoud not be numbered
	way=bychapter,    % eg wiki/Command/defineenumeration
	prefix=chapter,    % eg wiki/Command/defineenumeration
	headstyle=bold,
	headcolor=\getvariable{document}{color-headings},
	width=fit,
	alternative=hanging,  % top
	hang=fit]

\defineenumeration [mathematicalargument] [
	title=yes,    % now we can use title={}
	number=no,    % removes numbering
	headstyle=italic,
	titlestyle=italic,
	style=normal,
	headcolor=\getvariable{document}{color-headings},
	width=fit,
	alternative=hanging,  % top
	hang=fit,
	% before={\startmathematicalargumentframe},
	% after={\stopmathematicalargumentframe},
	% before=,
	% after=,
	closesymbol={□}]

\defineenumeration [exercise] [mathematicalstatement] [
	text=Exercise,
	list=all,
	listtext={Exercise }]

\defineenumeration [solution] [mathematicalargument] [
	text=Solution.]


\stopenvironment
