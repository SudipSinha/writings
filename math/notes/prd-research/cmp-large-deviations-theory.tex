\startcomponent *

\product  prd-research


\startchapter [title={Friedlin-Wentzell theorem}, reference=chp:Friedlin-Wentzell-theorem]


\stopchapter


\startchapter [title={Friedlin-Wentzell theorem for anticipating initial condition with extension of filtration}, reference=chp:FW-anticipating-IC-extension-filtration]

	\emph{Notation}: In what follows, \m{T < ∞} and \m{t ∈ [0, T]}.

	Our aim is to formulate a large deviations principle for a stochastic differential equation with anticipating initial conditions. Consider a very simple case
	\startformula
		X_t^ε = W_T + \sqrt{ε} ∫_0^t σ(X_t^ε) \d W_t , \quad t ∈ [0, T],
	\stopformula
	and \m{σ} satisfies the following:
	\startitemize [5, joinedup]
		\item  \emph{bounded}: \qquad  \m{\abs{σ(x)} ≤ M_σ}.
		\item  \emph{Lipschitz}: \qquad  \m{\abs{σ(x) - σ(y)} ≤ L_σ \abs{x - y}}.
		\item  \emph{linear growth}: \qquad  \m{\abs{σ(x)}^2 ≤ G_σ(1 + \abs{x}^2)}.
	\stopitemize

	We shall look at the method of enlargement of filtration by \cite[short][Itô1978]. We denote the enlarged filtration by \m{\tilde{ℱ}_t = ℱ_t ∨ σ(W_T)}. For \m{t ∈ [0, T]}, define the process \m{A_t = ∫_0^t \frac{W_T - W_u}{T - u} \d u}. Then \m{W_t = \tilde{W}_t + A_t}, where \m{\tilde{W}_⋅} is a Wiener process w.r.t. \m{\tilde{ℱ}_⋅}.

	\startsection [title={\m{\tilde{W}_⋅} is a Wiener process}]

		We show that \m{(\tilde{W}_t)} is a \m{(\tilde{ℱ}_t)}-martingale with quadratic variation \m{t}. Then by Lévy's Characterization of Wiener process, we obtain that \m{\tilde{W}_⋅} is a Wiener process.

		First we prove two lemmas.

		\startlemma [reference=lemma:equality-of-σalgebras]
			The σ-algebras \m{ℱ_s ∨ σ(W_T)} and \m{ℱ_s ∨ σ(W_T - W_s)} are the same.
		\stoplemma

		\startproof
			For any Borel set \m{B}, the set \m{\bcrl{W_T ∈ B} = \bcrl{(W_T - W_t) + W_t ∈ B}}.

			TODO
		\stopproof

		\startlemma [reference=lemma:conditional-expectation-of-W_t-W_s]
			For \m{0 ≤ s ≤ t ≤ T}, we have
			\startformula
				𝔼(W_t - W_s ∣ W_T - W_s) = \frac{t - s}{T - s} (W_T - W_s) .
			\stopformula
		\stoplemma

		\startproof
			We partition the interval \m{[0, T]} into \m{n = k n_0} equal parts, where \m{n_0 = \brnd{\min\bcrl{s, t - s, T - t}}^{-1}} and \m{k ∈ ℕ}. Let \m{n_s = s \frac{n}{T}} and \m{n_t = t \frac{n}{T}}. That is, the partition is
			\startformula
				P = \bcrl{0, \frac{T}{n}, …, \frac{n_s T}{n} = s, …, \frac{n_t T}{n} = t, …, \frac{(n - 1)T}{n}, T} .
			\stopformula

			Let \m{Δ_i W = W_{\frac{(i + 1) T}{n}} - W_{\frac{i T}{n}}}.

			Firstly, note that the \m{Δ_i W}s are independent and identically distributed from the definition of Wiener process. Now, using the linearity of conditional expectation, we have
			\startformula  \startalign
				\NC  𝔼(W_t - W_s ∣ W_T - W_s)  =  \NC  𝔼 \brnd{∑_{i = n_s}^{n_t - 1} Δ_i W ∣ ∑_{i = n_s}^{n_t - 1} Δ_i W}  \NR
				\NC  =  \NC  ∑_{i = n_s}^{n_t - 1} 𝔼 \brnd{Δ_i W ∣ ∑_{i = n_s}^{n - 1} Δ_i W}  \NR
				\NC  =  \NC  ∑_{i = n_s}^{n_t - 1} \frac{1}{n - n_s} ∑_{i = n_s}^{n - 1} 𝔼 \brnd{Δ_i W ∣ ∑_{i = n_s}^{n - 1} Δ_i W}  \NR
				\NC  =  \NC  ∑_{i = n_s}^{n_t - 1} \frac{1}{n - n_s}  𝔼 \brnd{∑_{i = n_s}^{n - 1} Δ_i W ∣ ∑_{i = n_s}^{n - 1} Δ_i W}  \NR
				\NC  =  \NC  ∑_{i = n_s}^{n_t - 1} \frac{1}{n - n_s}  ∑_{i = n_s}^{n - 1} Δ_i W  \NR
				\NC  =  \NC  ∑_{i = n_s}^{n_t - 1} \frac{1}{n - n_s}  (W_T - W_s)  \NR
				\NC  =  \NC  \frac{n_t - n_s}{n - n_s}  (W_T - W_s)  \quad  =  \quad  \frac{t - s}{T - s}  (W_T - W_s) . \NR
			\stopalign  \stopformula
		\stopproof


		\startproposition
			\m{\tilde{W}_⋅} is a \m{\tilde{ℱ}_⋅}-martingale.
		\stopproposition

		\startproof
			Let \m{0 ≤ s ≤ t ≤ T}. Then
			\startformula
				\tilde{W}_t - \tilde{W}_s
				=  (W_t - W_s) - ∫_s^t \frac{W_T - W_u}{T - u} \d u
				=  (W_t - W_s) - ∫_s^t \brnd{\frac{W_T - W_s}{T - u} - \frac{W_u - W_s}{T - u}} \d u .
			\stopformula

			Moreover, since \m{W_t - W_s} is independent of \m{ℱ_s} for every \m{t ≥ s}, using lemmas \in[lemma:equality-of-σalgebras] and \in[lemma:conditional-expectation-of-W_t-W_s], we get
			\startformula
				𝔼(W_t - W_s ∣ \tilde{ℱ}_s)
				=  𝔼 \brnd{W_t - W_s ∣ ℱ_s ∨ σ(W_T - W_s)}
				=  𝔼 \brnd{W_t - W_s ∣ W_T - W_s} = \frac{t - s}{T - s} (W_T - W_s) .
			\stopformula

			Therefore, using the fact that \m{W_T} and \m{W_s} are \m{\tilde{ℱ}_s}-measurable with conditional Fubini's theorem, we get
			\startformula  \startalign
				\NC  𝔼(\tilde{W}_t - \tilde{W}_s ∣ \tilde{ℱ}_s)
				     =  \NC  𝔼 \brnd{W_t - W_s ∣ \tilde{ℱ}_s} - ∫_s^t \brnd{\frac{W_T - W_s}{T - u} - \frac{𝔼 \brnd{W_u - W_s ∣ \tilde{ℱ}_s}}{T - u}} \d u  \NR
				\NC  =  \NC  \frac{t - s}{T - s} (W_T - W_s) - ∫_s^t \brnd{\frac{W_T - W_s}{T - u} - \frac{u - s}{T - s} \frac{W_T - W_s}{T - u}} \d u  \NR
				\NC  =  \NC  \frac{t - s}{T - s} (W_T - W_s) - ∫_s^t \frac{W_T - W_s}{T - s} \d u  \NR
				\NC  =  \NC  \frac{t - s}{T - s} (W_T - W_s) - \frac{t - s}{T - s} (W_T - W_s)  \quad  =  \quad  0 . \NR
			\stopalign  \stopformula

			Now, since \m{\tilde{W}_s} is \m{\tilde{ℱ}_s}-measurable, \m{𝔼(\tilde{W}_t ∣ \tilde{ℱ}_s) = 𝔼(\tilde{W}_t - \tilde{W}_s ∣ \tilde{ℱ}_s) + \tilde{W}_s = \tilde{W}_s}.
		\stopproof


		\startproposition
			The quadratic variation of \m{\tilde{W}_t} is \m{t}.
		\stopproposition

		\startproof
			TODO
		\stopproof

	\stopsection


	\startsection [title={Reformulating the problem}]

		For now, we shall bound \m{t ∈ [0, T_b]}, where \m{T_b ∈ [0, T)}.
		\comment{What happens when \m{t → T}?}

		Using this, we write our original stochastic differential equation as
		\startformula
			X_t^ε = W_T + \sqrt{ε} ∫_0^t σ(X_t^ε) \d \tilde{W}_t + \sqrt{ε} ∫_0^t σ(X_t^ε) \frac{W_T - W_s}{T - s} \d s .
		\stopformula

		Let \m{\tilde{X}_t^ε = X_t^ε - W_T} and \m{Y_t^ε = \sqrt{ε} (W_T - W_t)}. Then we have
		\startformula  \startalign
			\NC  \tilde{X}_t^ε  =  \NC  \sqrt{ε} ∫_0^t σ\brnd{\tilde{X}_t^ε + \frac{Y_0^ε}{\sqrt{ε}}} \d \tilde{W}_t + ∫_0^t σ\brnd{\tilde{X}_t^ε + \frac{Y_0^ε}{\sqrt{ε}}} \frac{Y_s^ε}{T - s} \d s , \text{ and}  \NR
			\NC  Y_t^ε  =  \NC  \sqrt{ε} W_T - \sqrt{ε} ∫_0^t \d \tilde{W}_t - ∫_0^t \frac{Y_s^ε}{T - s} \d s .
		\stopalign  \stopformula

		So together we have the joint process
		\placeformula[eq:Z] \startformula
			Z_t^ε :=
			\startrndmatrix
				\tilde{X}_t^ε  \NR  Y_t^ε  \NR
			\stoprndmatrix
			=
			\startrndmatrix
				0  \NR  \sqrt{ε} W_T  \NR
			\stoprndmatrix
			+  \sqrt{ε} ∫_0^t
			\startrndmatrix
				σ\brnd{\tilde{X}_t^ε + \frac{Y_0^ε}{\sqrt{ε}}}  \NR  -1  \NR
			\stoprndmatrix
			\d \tilde{W}_s  +  ∫_0^t
			\startrndmatrix
				σ\brnd{\tilde{X}_t^ε + \frac{Y_0^ε}{\sqrt{ε}}}  \NR  -1  \NR
			\stoprndmatrix
			\frac{Y_s^ε}{T - s} \d s .
		\stopformula

		Note that we expect \m{Z_t^ε → 0} as \m{ε ↘ 0}. We first obtain a large deviation principle for \m{Z_t^ε}.

	\stopsection

	\startsection [title={Existence and uniqueness of \m{Z^ε} in \ineq[eq:Z]}]

		First we aim for local existence. Fix \m{R > 0}, and define the exit time from the \m{R}-ball centered at the origin as
		\startformula
			τ_R = \inf\bcrl{t: Y^ε_0 > R} ∧ T_b .
		\stopformula
		Clearly \m{τ_R ↗ T_b} as \m{R ↗ ∞}.
		
		We now show that that there is a unique solution of \m{Z_t^ε} on \m{t ∈ \intcc{0, τ_R}}. For convenience, let
		\startformula
			\tilde{σ}^ε_{t, υ} (x, y) = \brnd{σ \brnd{x + \frac{υ}{\sqrt{ε}}}, -1}
			\qquad \text{and} \qquad
			\tilde{b}^ε_{t, υ} (x, y) = \frac{y}{T - s} \brnd{σ \brnd{x + \frac{υ}{\sqrt{ε}}}, -1} .
		\stopformula

		\startlemma
			The process \m{Z_t^ε} exists and is unique for \m{t ∈ \intcc{0, τ_R}}.
		\stoplemma
		\startproof
			Let \m{υ = Y^ε_0}. We first show that \m{\tilde{σ}} and \m{\tilde{b}} satisfy the linear growth and Lipshitz conditions locally.

			\startitemize [5]
				
				\item  \emph{Lipschitz condition for \m{\tilde{σ}}}: \qquad
					Since \m{σ} is Lipschitz, we have
					\startformula  \startalign
						\NC  \norm{\tilde{σ}^ε_{t, υ_2} (x_2, y_2) - \tilde{σ}^ε_{t, υ_1} (x_1, y_1)}  =  \NC  \abs{σ \brnd{x_2 + \frac{υ_2}{\sqrt{ε}}} - σ \brnd{x_1 + \frac{υ_1}{\sqrt{ε}}}}  \NR
						\NC  ≤  \NC  L_σ \brnd{\abs{x_2 - x_1} + \frac{1}{\sqrt{ε}} \abs{υ_2 - υ_1}}  \NR
						\NC  ≤  \NC  L_σ \brnd{1 ∨ \frac{1}{\sqrt{ε}}} \brnd{\abs{x_2 - x_1} + \abs{υ_2 - υ_1}} . \NR
					\stopalign  \stopformula

				\item  \emph{Lipschitz condition for \m{\tilde{b}}}: \qquad
					Using the boundedness of \m{σ}, we get
					\startformula  \startalign
						\NC     \NC  \norm{\tilde{b}^ε_{t, υ_2} (x_2, y_2) - \tilde{b}^ε_{t, υ_1} (x_1, y_1)}  \NR
						\NC  ≤  \NC  \frac{1}{T-t} \brnd{\abs{σ \brnd{x_2 + \frac{υ_2}{\sqrt{ε}}} y_2 - σ \brnd{x_1 + \frac{υ_1}{\sqrt{ε}}} y_1} + \abs{y_2 - y_1}}  \NR
						\NC  ≤  \NC  \frac{1}{T-t} \brnd{\abs{σ \brnd{x_2 + \frac{υ_2}{\sqrt{ε}}}} \abs{y_2 - y_1} + \abs{σ \brnd{x_2 + \frac{υ_2}{\sqrt{ε}}} - σ \brnd{x_1 + \frac{υ_1}{\sqrt{ε}}}} \abs{y_1} + \abs{y_2 - y_1}}  \NR
						\NC  ≤  \NC  \frac{1}{T-t} \brnd{M_σ \abs{y_2 - y_1} + L_σ \brnd{1 ∨ \frac{1}{\sqrt{ε}}} \abs{y_1} \brnd{\abs{x_2 - x_1} + \abs{υ_2 - υ_1}} + \abs{y_2 - y_1}}  \NR
						\NC  ≤  \NC  \good{\frac{1}{T-t} \brnd{(M_σ + 1) ∨ \brnd{L_σ \brnd{1 ∨ \frac{1}{\sqrt{ε}}} R}}} \brnd{\abs{x_2 - x_1} + \abs{y_2 - y_1} + \abs{υ_2 - υ_1}} , \NR
					\stopalign  \stopformula
					where \m{\abs{y_1} ≤ R} since \m{t ∈ \intcc{0, τ_R}}.

				\item  \emph{Linear growth condition for \m{\tilde{σ}}}: \qquad
					\startformula  \startalign
						\NC  \norm{\tilde{σ}^ε_{t, υ} (x, y)}^2  =  \NC  1 + \abs{σ \brnd{x + \frac{υ}{\sqrt{ε}}}}^2  \NR
						\NC  ≤  \NC  1 + G_σ \brnd{1 + \abs{x + \frac{υ}{\sqrt{ε}}}^2}  \NR
						\NC  ≤  \NC  1 + G_σ \brnd{1 + 2 \abs{x}^2 + 2\frac{\abs{υ}^2}{ε}}  \NR
						\NC  ≤  \NC  \good{2 G_σ \brnd{1 ∨ \inv{ε}}} \brnd{1 + \abs{x}^2 + \abs{υ}^2} . \NR
					\stopalign  \stopformula

				\item  \emph{Linear growth condition for \m{\tilde{b}}}: \qquad
					\startformula  \startalign
						\NC  \norm{\tilde{b}^ε_{t, υ} (x, y)}^2  =  \NC  1 + \abs{σ \brnd{x + \frac{υ}{\sqrt{ε}}} \frac{y}{T - t}}^2  \NR
						\NC  ≤  \NC  \good{\frac{1}{T - t} 2 G_σ \brnd{1 ∨ \inv{ε}} R^2} \brnd{1 + \abs{x}^2 + \abs{υ}^2} . \NR
					\stopalign  \stopformula
			\stopitemize

			The above implies that \m{Z^ε_t} exists and is unique for \m{t ∈ \intcc{0, τ_R}}.
		\stopproof

		\blank[big]
		Now, let \m{η} be a smooth function such that \m{η(x) ≡ 1} for \m{\abs{x} ≤ R} and \m{η ≡ 0} for \m{\abs{x} > R+1}. Define \m{\hat{σ}_R(z) = η(z) \tilde{σ}^ε_{t, υ} (z)} and \m{\hat{b}_R(z) = η(z) \tilde{b}^ε_{t, υ} (z)}. Now consider the equation
		\startformula
			\hat{Z}^ε_t = Z_0
		\stopformula
	\stopsection

\stopchapter

\stopcomponent

% \startformula  \startalign
	
% \stopalign  \stopformula
