\environment env-talks

% lsu-math-301d
% Hilbert space er orthonormal basis
% Vector spacer basis
% Hermite polynomial



%%%%%%%%%%%%%%%%%%%%%%%%%%%%%%%%%%%%
% This is where the document starts.
%%%%%%%%%%%%%%%%%%%%%%%%%%%%%%%%%%%%

\starttext

\startfrontmatter

%%%%%%%%%%%%%%%%
% Front matter %
%%%%%%%%%%%%%%%%

\setupbackgrounds [page] [
	background={color,backgraphics,foreground},
	backgroundcolor=\getvariable{document}{color-background-0}]
\startcolor [\getvariable{document}{color-foreground-0}]    % text color


% \startmode [handout]
% \startcolumns
% \stopmode


% Introduction
\startcolor [\getvariable{document}{color-foreground-0}]    % text color

\startmode [presentation]

\startslide

\startalign [middle]

	{\tfd
		\color[\getvariable{document}{color-foreground-1}]{A generalization of Itô calculus}\\
		and\\
		\color[\getvariable{document}{color-foreground-2}]{large deviations theory}}

	\blank[2*line]

	{\tfb \getvariable{document}{author}}

	\blank[line]

	{\tfa \getvariable{document}{date}}

	\blank[2*line]

	Advisors

	\color[\getvariable{document}{color-foreground-1}]{Prof. Hui-Hsiung Kuo}

	\color[\getvariable{document}{color-foreground-2}]{Prof. Padmanabhan Sundar}
\stopalign
\stopslide


% Table of contents
\startslide [title={Outline}]
	\placecontent

\stopslide
\stopmode

% \startmode [manuscript]

% This presentation is going to be on two topics:
% \startitemize[n,nowhite,after]
% 	\item  Generalization of stochastic integrals developed primarily by Professor H.-H. Kuo
% 	\item  Applications of generalization in large deviations theory
% \stopitemize

% \stopmode

\stopfrontmatter



\startbodymatter

%%%%%%%%%%%%%%%%
% Introduction %
%%%%%%%%%%%%%%%%

\setupbackgrounds [page] [
	background={color,backgraphics,foreground},
	backgroundcolor=\getvariable{document}{color-background-0}]
\startcolor [\getvariable{document}{color-foreground-0}]    % text color

\startsection [title={Introduction and motivation}, reference=sec:introduction]

\startmode [presentation]

\startslide [title={Quick revision and notations}]

	\startitemize [4]

		\item  Let \m{T ∈ (0, ∞)}, and denote \m{t ∈ [0, T]}.

		\item  Let \m{(Ω, ℱ, ℱ_{\argdotsub}, ℙ)} be a filtered probability space.

		\item  \m{B_{\argdotsub}} is a Brownian motion on \m{(Ω, ℱ, ℱ_{\argdotsub}, ℙ)}.

		\item  Properties of \m{B_{\argdotsub}}
			\startitemize [5, columns, joinedup]
				\item  starts at 0
				\item  has independent increments
				\item  \m{B_t - B_s ∼ 𝓝(0, t - s)}
				\item  continuous paths (a.s.)
				\item  is a.s. nowhere differentiable
				\item  has \bad{unbounded linear variation \frown}
				\item  has \good{bounded quadratic variation \smile}
				\item  \m{𝔼(B_t B_s) = s ∧ t}
				\item  martingale
			\stopitemize

		\item  \bad{Naive integration w.r.t. \m{B_t}: not possible}.

		\item  A stochastic process \m{X_{\argdotsub}} is called \important{adapted} to \m{ℱ_{\argdotsub}} if \m{X_t} is measurable w.r.t. \m{ℱ_t} \m{∀t}.

	\stopitemize
\stopslide

\startslide[title={Martingales and Markov processes}]

	\startdefinition

		Let \m{X_{\argdotsub}} be an integrable \m{ℱ_{\argdotsub}}-adapted stochastic process and let \m{0 ≤ s ≤ t ≤ T}. Then \m{X_{\argdotsub}} is called a \important{martingale} if \m{𝔼(X_t ∣ ℱ_s) = X_s}.
		% \startitemize [5, joinedup]
		% 	\item  \important{martingale} if \m{𝔼(X_t ∣ ℱ_s) = X_s},
		% 	\item  \important{submartingale} if \m{𝔼(X_t ∣ ℱ_s) ≥ X_s},
		% 	\item  \important{supermartingale} if \m{𝔼(X_t ∣ ℱ_s) ≤ X_s}.
		% \stopitemize
	\stopdefinition

	\startremark

		If \m{(X_n)} is a discrete-time martingale and \m{(H_n)} is an adapted process, then the process \m{Y_n = ∑_{j = 0}^{n - 1} H_j (X_{j + 1} - X_j) =: (H ∙ X)_n} is itself a martingale and called a \important{martingale transform} of \m{(X_n)}.
	\stopremark

	\startdefinition

		A stochastic process \m{X_{\argdotsub}} is called \important{Markov} if for any \m{0 ≤ s ≤ t ≤ T}, we have
		\startformula
			ℙ(X_t ∈ E ∣ ℱ_s) = ℙ(X_t ∈ E ∣ X_s) .
		\stopformula
	\stopdefinition
\stopslide

\startslide [title={Wiener integral for \m{f ∈ L^2[0, T]}}]

	\startitemize [4]

		\item  Definition of the integral:
			\startitemize [n, joinedup]
				\item  Step functions \m{f = ∑_{j = 0}^{n-1} c_j 𝟙_{[t_j, t_{j + 1})}(t)}: Define \m{∫_0^T f(t) \d B_t  =  ∑_{j = 0}^{n-1} c_j Δ B_j}, where

					\important{\m{Δ B_j = B_{t_{j + 1}} - B_{t_j}}}.

				\item  \m{f ∈ L^2[0, T]}: Use step functions approximating \m{f} to extend the integral \important{a.s.}
			\stopitemize

		\item  Properties of the integral:
			\startitemize [5, joinedup]
				\item  Linear.
				\item  \good{Gaussian distribution} with mean 0 and variance \m{\norm[f]_{L^2[0, T]}^2} (Itô isometry).
				\item  Agrees with the Riemann–Stieltjes integral for continuous functions of bounded variation.
			\stopitemize

		\item  Properties of the associated process \m{I_{\argdotsub} = ∫_0^{\argdotsup} f(t) \d B_t}:
			\startitemize [5, joinedup]
				\item  continuity.
				\item  martingale.
			\stopitemize

		\item  Problem: Cannot integrate stochastic processes.

	\stopitemize
\stopslide

\startslide [title={Trying to integrate stochastic processes}]

	\startitemize [4]

		\item  \m{∫_0^T B_t \d B_t ≟}

			Since \m{B_t} is continuous, let us try the Riemann–Stieltjes integral. Consider a sequence of partitions \m{Δ_n} such that \m{\norm[Δ_n] → 0}. Then
			\startformula
				∫_0^T B_t \d B_t  =  \lim ∑_{j = 0}^{n - 1} B_{t_j^*} Δ B_j .
			\stopformula

		\item  Choosing different endpoints for \m{t_j^*} gives us different results.
			\starttabulate [|c|m|c|m|c|c|]
				\NC  \m{t_j^*}  \NC  ∫_0^t B_s \d B_s  \NC  Intuitive?  \NC  𝔼  \NC  Martingale?  \NC  Theory  \NR
				\FL
				\NC  \good{left}   \NC  \good{\frac12 \brnd[B_t^2 - t]}  \NC  \good{\frown}  \NC  \good{0}  \NC  \good{\smile}  \NC  \good{Itô}  \NR
				\NC  mid    \NC  \frac12 \brnd[B_t^2]  \NC  \smile  \NC  \frac12 t  \NC  \frown  \NC  Stratonovich  \NR
				\NC  \bad{right}  \NC  \bad{\frac12 \brnd[B_t^2 + t]}  \NC  \bad{\frown}  \NC  \bad{t}  \NC  \bad{\frown}  \NC    \NR
				\BL
			\stoptabulate

		\item  Which one do we choose?

	\stopitemize
\stopslide

\startslide [title={Itô integral \cite[short][Itô1944SI] for \m{X_{\argdotsub} ∈ L^2_{\text{ad}} ([0, T] × Ω)}}]

	\startitemize [4]

		\item  Definition of the integral:
			\startitemize [n, joinedup]
				\item  Adapted step processes \m{X_t(ω) = ∑_{j = 0}^{n-1} ξ_j(ω) 𝟙_{[t_j, t_{j + 1})}(t)}: define \m{∫_0^T X_t \d B_t  =  ∑_{j = 0}^{n-1} ξ_j Δ B_j}.
				\item  \m{X ∈ L^2_{\text{ad}} ([0, T] × Ω)}: use step processes approximating \m{X} to extend the integral \important{in \m{L^2(Ω)}}.
			\stopitemize

		\item  Properties of the integral:
			\startitemize [5, joinedup]
				\item  Linear.
				\item  Mean 0 and variance \m{\norm[f]_{L^2[0, T]}^2} \good{(Itô isometry)}.
				\item  For \m{X_{\argdotsub}} continuous,
					\m{∫_0^T X_t \d B_t
						=  \lim ∫_0^T X_{\floor[\frac{t n}{n}]} \d B_t
						=  \lim ∑_{j = 0}^{n - 1} X_{t_j} Δ B_j}.
			\stopitemize

		\item  Properties of the associated process \m{I_{\argdotsub} = ∫_0^{\argdotsup} X_t \d B_t}:
			\startitemize [5, joinedup]
				\item  continuity.
				\item  martingale.
			\stopitemize

		\item  Example: \m{∫_0^t B_u \d B_u = \frac12 (B_t^2 - t) \quad ∀ t}.

		% \item  Problem: Cannot integrate many continuous functions of \m{B_t}, for example \m{e^{B_t^2}}.

	\stopitemize
\stopslide

\startslide [title={Itô integral for \m{X_{\argdotsub}} such that \m{∫_0^T X_t^2 \d t < ∞} a.s.}]

	\startitemize [4]

		\item  Definition: Use sequences of processes in \m{L^2_{\text{ad}} ([0, T] × Ω)} approximating \m{X} in probability to extend the integral \important{in probability}.

		\item  Properties of the integral:
			\startitemize [5, joinedup]
				\item  Linear.
				\item  \bad{Mean and variance? \frown}
			\stopitemize

		\item  Properties of the associated process \m{I_{\argdotsub} = ∫_0^{\argdotsup} X_t \d B_t}:
			\startitemize [5, joinedup]
				\item  continuity.
				\item  \okay{local} martingale.
			\stopitemize

		\item  Example: \m{∫_0^T e^{B_t^2} \d B_t = ∫_0^{B_T} e^{t^2} \d t - ∫_0^T B_t e^{B_t^2} \d t}.

	\stopitemize
\stopslide

\startslide [title={The Itô formula}]

	\startitemize [4]

		% \item  A (local, continuous) \important{semimartingale} is a process \m{X_t} that can be written as \m{X_t = X_0 + M_t + A_t}, where
		% \startitemize [n, joinedup]
		% 	\item  \m{M_t} is a mean-zero (local, continuous) martingale, and
		% 	\item  \m{A_t} is an right-continuous adapted process of locally bounded variation.
		% \stopitemize
		% This is equivalently represented in the \important{differential form} as \m{\d X_t = \d M_t + \d A_t}.

		\item  An \important{Itô process} is a process of the form \m{X_{\argdotsub} = X_0 + ∫_0^{\argdotsup} m_t \d t + ∫_0^{\argdotsup} σ_t \d B_t}.

		Equivalently expressed as \m{\d X_t = m_t \d t + σ_t \d B_t}.

		% [Only makes sense when \m{∫_0^T \brnd[{\abs[m_s] + \abs[σ_s]^2}] \d s < ∞} a.s.]

	\stopitemize

	\starttheorem[title={\cite[short][Itô1944SI]}]
		Let \m{X_t} be a \m{d}-dimensional Itô process, and assume \m{f ∈ C^{1, 2}(ℝ × ℝ)}. Then \m{f(t, X_t)} is also a \m{d}-dimensional Itô process given by
			\startformula
				\d f(t, X_t)  =  \frac{∂ f}{∂ t} (t, X_t) \d t
				 +  \inn[(\D f) (t, X_t), \d X_t]
				 +  \frac12 \inn[\d X_t, (D^2 f)(t, X_t) \, \d X_t] ,
			\stopformula
			% \startformula
			% 	\d Y_t
			% 	=  \d f(X_t)
			% 	=  ∑_{j = 1}^d  \frac{∂f}{∂x_j} (X_t)  \d A_t^{(j)}
			% 	 + ∑_{j = 1}^d  \frac{∂f}{∂x_j} (X_t) \d M_t^{(j)}
			% 	 + \frac12 ∑_{j, k = 1}^d  \frac{∂^2 f}{∂x_j ∂x_k} (X_t) \d \inn[M^{(j)}, M^{(k)}]_t ,
			% \stopformula
			where we use the rule \important{\m{\d B_t ⊗ \d B_t = I_d \d t}}.
	\stoptheorem

	\startitemize [4]

		\item  Example: For \m{σ} constant, \m{𝓔_t = \exp\brnd[σ B_t - \frac12 σ^2 t]}, \m{\d 𝓔_t = \comment{- \frac12 σ^2 𝓔_t \d t} + σ 𝓔_t \d B_t \comment{+ \frac12 σ^2 𝓔_t (\d B_t)^2}}.

	\stopitemize
\stopslide

\startslide [title={Exponential processes and the Girsanov theorem}]

	\startitemize [4]

		\item  Let \m{h_{\argdotsub}} be an adapted stochastic process. The associated \important{exponential process} is defined as
			\startformula
				𝓔^{(h)}_{\argdotsub} = \exp\brnd[∫_0^{\argdotsup} h_t \d B_t - \frac12 ∫_0^{\argdotsup} h_t^2 \d t] .
			\stopformula

		\item  The exponential process is a martingale if and only if \m{𝔼 𝓔^{(h)}_t = 1 \ ∀ t}.

		\item  The \important{Novikov condition}:
			The exponential process is a martingale if \m{𝔼 \exp\brnd[\frac12 ∫_0^T h_t^2 \d t] < ∞}.

		\item  The \important{Girsanov theorem} \cite[short][Girsanov1960]:
			The process \m{W_{\argdotsub} = B_{\argdotsub} - ∫_0^{\argdotsup} h_t \d t} is a Brownian motion under the probability measure \m{\widetilde{ℙ}} defined by the Radon-Nikodym derivative
			\m{\frac{\d \widetilde{ℙ}}{\d ℙ} = 𝓔^{(h)}_T}.

			% Moreover the process \m{Z_t := 𝔼 \brnd[𝓔^{h}_T ∣ 𝓕_t]} is a martingale.

	\stopitemize
\stopslide

\startslide [title={Stochastic differential equations}]

	\startitemize [4]

		\item  Let \m{ξ ∈ L^2(Ω)} be independent of \m{B_{\argdotsub}}, and \m{m, σ: [0, T] × ℝ → ℝ} have \quote{nice} measurability.
		% \m{m, σ: [0, T] × ℝ × Ω → ℝ} be \m{𝓑[0, T] × 𝓑(ℝ) × 𝓕} measurable such that \m{m(t, ⋅, ⋅)} and \m{σ(t, ⋅, ⋅)} are \m{𝓑(ℝ) × 𝓕_t} measurable \m{∀ t}.

			Then a \m{𝓕_t}-adapted stochastic process \m{X_t} is called a solution of the \important{stochastic \emph{integral} equation} \important{\m{X_t = ξ + ∫_0^t m(s, X_s) \d s + ∫_0^t σ(s, X_s) \d B_s}} if for each \m{t}, the \m{X_t} satisfies the integral equation a.s.

		\item  \important{Stochastic \emph{differential} equation} \m{\d X_t = m(t, X_t) \d t + σ(t, X_t) \d B_t, \ X_0 = ξ} is a \emph{formal representation}.
	\stopitemize

	\starttheorem [title={Existence and uniqueness, Markov property}]

		% The SDE above has a unique solution if there exists an \m{M > 0} such that the following two conditions are satisfied:
		% \startitemize [5, joinedup]
		% 	\item  (Lipschitz condition) \m{\abs[m(t, x) - m(t, y)]^2 + \abs[σ(t, x) - σ(t, y)]^2 ≤ M \abs[x - y]^2} a.s.
		% 	\item  (growth condition) \m{\abs[m(t, x)]^2 + \abs[σ(t, y)]^2 ≤ M \brnd[1 + {\abs[x]}^2]} a.s.
		% \stopitemize
		The SDE above has a \important{unique} solution if \m{m} and \m{σ} are \important{Lipschitz} and satisfy the \important{linear growth condition}.

		The solution is a Markov process.

		Moreover if \m{ξ ∈ ℝ} and \m{m, σ} are functions of only \m{x}, then the solution is also stationary.

	\stoptheorem

	\startitemize [1, joinedup]

		\item  	Example: For \m{σ} constant, \m{\d 𝓔_t = σ 𝓔_t \d B_t, 𝓔_0 = 1} is solved by \m{𝓔_t = \exp\brnd[σ B_t - \frac12 σ^2 t]}.

	\stopitemize
\stopslide

\startslide [title={Multiple Wiener–Itô integrals}]

	\startitemize [4]

		\item  How do we define the double integral?

		\item  Naive idea: \m{∫_0^t ∫_0^t \d B_u \d B_v = ∫_0^t \d B_u ∫_0^t \d B_v = B_t^2}.

			But \m{𝔼 B_t^2 = \ugly{t ≠ 0}}, so \ugly{no martingale property}. \frown

		\item  Itô's idea: remove the diagonal to get
		\startformula
			∫_0^t ∫_0^t \d B_u \d B_v  =  2 ∫_0^t ∫_0^v \d B_u \d B_v  =  2 ∫_0^t B_v \d B_v  =  B_t^2 - t .
		\stopformula


	\stopitemize

	\starttheorem[title={\cite[short][Itô1951MWI]}]
		Let \m{f ∈ L^2\brnd[{[0, T]}^n]} and \m{\bad{\hat{f}}} be its symmetrization. Then
		\startformula
			∫_{[0, T]^n} f(t_1, …, t_n) \d B_{t_1} ⋯ \d B_{t_n}  =  \bad{n!} ∫_0^T ⋯ ∫_0^{\bad{t_{n-2}}} \brnd[∫_0^{\bad{t_{n-1}}} \bad{\hat{f}}(t_1, …, t_n) \d B_{t_n}] \d B_{t_{n-1}} ⋯ \d B_{t_1} .
		\stopformula
		% \m{\hat{f}(t_1, …, t_n)  =  \frac{1}{n!} ∑_{σ ∈ S_n} f(t_{σ(1)}, …, t_{σ(n)})}
	\stoptheorem

	% \startitemize [4]
	% 	\item  Example: \m{}
	% \stopitemize
\stopslide


\stopmode

\stopsection

\page    % Needed for correct color transition

% \startmode [manuscript]

% This is the first part and we are going to talk about generalization of stochastic integrals developed primarily by Professor H.-H. Kuo.

% \stopmode



%%%%%%%%%%%%%%%%%%%%%%%%%%%%%%%%%%%%%%%%%%%
% A Generalization of stochastic calculus %
%%%%%%%%%%%%%%%%%%%%%%%%%%%%%%%%%%%%%%%%%%%

\setupbackgrounds [page] [
	background={color,backgraphics,foreground},
	backgroundcolor=\getvariable{document}{color-background-1}]
\startcolor  [\getvariable{document}{color-foreground-1}]    % text color

\startsection [title={A Generalization of Itô calculus}, reference=sec:generalization-Itô-calculus]

\startmode [presentation]

\startslide [title={Motivation}]

	\startitemize [4]

		\item  Iterated integrals: Consider the iterated integral \m{∫_0^t ∫_0^t \d B_u \d B_v \ugly{= ∫_0^t B_t \d B_v ≟ B_t^2}}.

		\item  Note that \m{𝔼(B_t^2) = \ugly{t ≠ 0}}, so \ugly{no martingale property} \frown.

		\item  Stochastic differential equations with anticipation:
			% \startitemize [5, joinedup]
			% 	\item  \m{\d X_t  =  X_t \d B_t, X_0 = B_T}
			% 	\item  \m{\d Y_t  =  B_T \d B_t, Y_0 = 1}
			% \stopitemize
		\startformula \startalign[m=2, distance=8em, align={right, left, right, left}]
			\NC  \d X_t  \NC =  X_t \d B_t
			\NC  \d Y_t  \NC =  B_T \d B_t
			\NR
			\NC  X_0  \NC =  B_T
			\NC  Y_0  \NC =  1
		\stopalign \stopformula

		\item  Problem: We want to define \m{∫_0^T Z(\argdotmid) \d B_t}, where \m{Z(\argdotmid)} is not (necessarily) adapted.

		\item  Some approaches:
		\startitemize [5, joinedup]
			\item  Enlargement of filtration \m{𝓖_{\argdotsub} = 𝓕_{\argdotsub} ∨ σ(B_T)}, with Itô's decomposition of integrand \cite[short][Itô1978] \m{B_t = \brnd[B_t - ∫_0^t \frac{B_T - B_s}{T - s} \d s] + ∫_0^t \frac{B_T - B_s}{T - s} \d s}.
			\item  White noise theory
			\item  Malliavin calculus
			% \item  …

				% \m{B_t = \brnd[B_t - ∫_0^t \frac{B_T - B_s}{1 - s} \d s] + ∫_0^T \frac{B_T - B_s}{1 - s} \d s}
		\stopitemize

	\stopitemize
\stopslide

\startslide [title={The new integral \cite[short][AyedKuo2008, AyedKuo2010]}: Idea]

	\startitemize[1]

		\item  A process \m{Y^{\argdotsup}} and filtration \m{ℱ_{\argdotsub}} are called \important{instantly independent} if \m{Y^t} and \m{ℱ_t} are independent \m{∀ t}.

			Example: The process \m{(B_T - B_{\argdotsub})} is instantly independent of the filtration generated by \m{B_{\argdotsub}}.

		\item  Ideas
			\startitemize[n, joinedup]

				\item  Decompose the integrand into \good{adapted} and \okay{instantly independent} parts.

				\item  Evaluate the \good{adapted} and the \okay{instantly independent} parts at the \good{left} and \okay{right} endpoints.

			\stopitemize

		\item  Consider two continuous stochastic processes, \good{\m{X_t} adapted} and \okay{\m{Y^t} instantly independent} w.r.t. \m{ℱ_{\argdotsub}}. Then the integral \m{∫_0^T \good{X_t} \okay{Y^t} \d B_t} is \important{defined} as
			\startformula
				∫_0^T \good{X_t} \okay{Y^t} \d B_t  ≜  \lim_{\norm[Δ_n] → 0}  ∑_{j = 0}^{n - 1} \good{X_{t_{j}}} \okay{Y^{t_{j+1}}} ΔB_j ,
			\stopformula
			provided that the limit exists in probability.

		\item  Now, for any stochastic process \m{Z(t) = ∑_{k = 1}^n \good{X_t^{(k)}} \okay{Y^t_{(k)}}} we extend the definition by linearity.

		\item  This is well-defined \cite[short][HwangKuoSaitôZhai2016].
		%\m{∫_0^T Z(t) \d B_t = ∑_{k = 1}^n ∫_0^T Z(t) X_t^{(k)} Y^t_{(k)} \d B_t}.
		% \startformula
		% 	∫_0^T Z(t) \d B_t = ∑_{k = 1}^n ∫_0^T \good{X_t^{(k)}} \okay{Y^t_{(k)}} \d B_t
		% \stopformula

	\stopitemize
\stopslide

\startslide [title={A simple example}]

	\startitemize [4]

		\item  	In the following, \m{\lim} is the limit in \m{L^2}.
			\startformula \startalign
				\NC  ∫_0^t B_T \d B_t
					\NC =  ∫_0^t (\good{B_t} + \okay{(B_T - B_t)}) \d B_t
					    =  \good{∫_0^t B_t \d B_t}  +  \okay{∫_0^t (B_T - B_t) \d B_t}
				\NR \NC
					\NC =  \good{\lim ∑_{j = 0}^{n - 1} B_{t_j} Δ B_j}
						+  \okay{\lim ∑_{j = 0}^{n - 1} (B_T - B_{t_{j + 1}}) Δ B_j}
				\NR \NC
					\NC =  \lim ∑_{j = 0}^{n - 1} \brnd[B_T - Δ B_j] Δ B_j
				\NR \NC
					\NC =  B_T \lim ∑_{j = 0}^{n - 1} Δ B_j - \lim ∑_{j = 0}^{n - 1} (Δ B_j)^2
					    =  B_T B_t - t
			\stopalign \stopformula

		\item  Note that \m{𝔼(B_T B_t - t) = 0}.

		\item  In general, \m{𝔼 ∫_0^t Z(s) \d B_s = 0}. \good{\smile}
	\stopitemize
\stopslide

\startslide [title={The near-martingale property}]

	\startitemize [4]

		\item  Question: What are the analogues of the martingale property and the Markov property?

		% \item  Answer for martingales: near-martingales \cite[short][KuoSaeTangSzozda2012].

		\item  Example: \m{𝔼(B_T B_t - t ∣ ℱ_s) = B_s^2 - s ≠ B_T B_s - s}. \frown

			But \m{𝔼(B_T B_s - s ∣ ℱ_s) = B_s^2 - s}. \smile

		\item  Let \m{Z(t)} be a process such that \m{𝔼\abs[Z(t)] < ∞ \ ∀ t}, and \m{0 ≤ s ≤ t ≤ T}. Then \m{Z(t)} is called a \important{near-martingale} if \m{𝔼(Z(t) ∣ ℱ_s) = 𝔼(Z(s) ∣ ℱ_s)}.
			% \startitemize [5, joinedup]
			% 	\item  \important{near-martingale} if \m{𝔼(Z(t) ∣ ℱ_s) = 𝔼(Z(s) ∣ ℱ_s)},
			% 	\item  \important{near-submartingale} if \m{𝔼(Z(t) ∣ ℱ_s) ≥ 𝔼(Z(s) ∣ ℱ_s)},
			% 	\item  \important{near-supermartingale} if \m{𝔼(Z(t) ∣ ℱ_s) ≤ 𝔼(Z(s) ∣ ℱ_s)}.
			% \stopitemize

	\stopitemize

	\starttheorem [title={\cite[short][KuoSaeTangSzozda2012]}]
		Let \m{f} and \m{ϕ} be continuous functions on \m{ℝ}. Under integrability conditions, the processes \m{X_{\argdotsub} = ∫_0^{\argdotsup} f(B_t) ϕ(B_T - B_t) \d B_t} and \m{Y^{\argdotsup} = ∫_{\argdotsub}^T f(B_t) ϕ(B_T - B_t) \d B_t} are near-martingales.
		% \startformula
		% 	X_{\argdotsub} = ∫_0^{\argdotsup} f(B_t) ϕ(B_T - B_t) \d B_t
		% 	\qquad \text{ and } \qquad
		% 	Y^{\argdotsup} = ∫_{\argdotsub}^T f(B_t) ϕ(B_T - B_t) \d B_t
		% \stopformula
		% are near-martingales.
	\stoptheorem

	\starttheorem [title={\cite[short][HwangKuoSaitôZhai2017]}]
		Let \m{Z(\argdotmid)} be a stochastic process bounded in \m{L^1}, and \m{X_{\argdotsub} = 𝔼(Z(\argdotmid) ∣ ℱ_{\argdotsub})}. Then \m{X_{\argdotsub}} is a (\good{sub}/\bad{super})martingale if and only if \m{Z(\argdotmid)} is a near-(\good{sub}/\bad{super})martingale.
	\stoptheorem
\stopslide

\startslide [title={A generalized Itô formula \cite[short][HwangKuoSaitôZhai2016]}]

	\starttabulate [|c|m|m|]
		\NC  Process
		\NC  \text{Definition}
		\NC  \text{Representation}

		\NR
		\FL

		\NC  \good{Itô}
		\NC  \good{X_{\argdotsub} = X_0 + ∫_0^{\argdotsup} m_t \d t + ∫_0^{\argdotsup} σ_t \d B_t}
		\NC  \good{\d X_t = m_t \d t + σ_t \d B_t}

		\NR

		\NC  \okay{instantly independent}
		\NC  \okay{Y^{\argdotsup} = Y^0 + ∫_{\argdotsub}^T η^t \d t + ∫_{\argdotsub}^T ς^t \d B_t}
		\NC  \okay{\d Y^t = - η^t \d t - ς^t \d B_t}

		\NR
		\BL
	\stoptabulate

	Here \m{η^t} and \m{ς^t} are instantly independent such that \m{Y^t} is also instantly independent.

	\starttheorem[title={\cite[short][HwangKuoSaitôZhai2016]}]
		Let \good{\m{\d X_t = m_t \d t + σ_t \d B_t}} be an \m{d}-dimensional \good{Itô} process, and

		\okay{\m{\d Y^t = - η^t \d t - ς^t \d B_t}} be a \m{k}-dimensional \okay{instantly independent} process.
		If \m{f(t, x, y) ∈ C^{1, 2, 2}(ℝ × ℝ × ℝ)}, then
		\startformula \startalign[n=3]
				\NC \d f(t, X_t, Y^t)  =
				\NC \frac{∂ f}{∂ t}(t, X_t, Y^t) \d t
				\NC + \good{\inn[(\D_x f) (t, X_t, Y^t), \d X_t]
					+ \frac12 \inn[\d X_t, (D_x^2 f)(t, X_t, Y^t) \, \d X_t]}
			\NR \NC
				\NC \qquad \qquad \qquad
				\NC + \okay{\inn[(\D_y f) (t, X_t, Y^t), \d Y^t]
					- \frac12 \inn[\d Y^t, (D_y^2 f)(t, X_t, Y^t) \, \d Y^t]} ,
		\stopalign \stopformula
		where we use the rules \important{\m{\d B_t ⊗ \d B_t = I_d \d t}}.
	\stoptheorem
\stopslide

\startslide [title={Iterated integrals}]

	\starttheorem[title={\cite[short][Itô1951MWI]}]
		Let \m{f ∈ L^2([0, T]^n)} and \m{\bad{\hat{f}}} be its symmetrization. Then
		\startformula
			∫_{[0, T]^n} f(t_1, …, t_n) \d B_{t_1} … \d B_{t_n}  =  \bad{n!} ∫_0^T ⋯ ∫_0^{\bad{t_{n-2}}} \brnd[∫_0^{\bad{t_{n-1}}} \bad{\hat{f}}(t_1, …, t_n) \d B_{t_n}] \d B_{t_{n-1}} … \d B_{t_1} .
		\stopformula
		% \m{\hat{f}(t_1, …, t_n)  =  \frac{1}{n!} ∑_{σ ∈ S_n} f(t_{σ(1)}, …, t_{σ(n)})}
	\stoptheorem

	\starttheorem[title={\cite[short][AyedKuo2010]}]
		Let \m{f ∈ L^2([0, T]^n)}. Then
		\startformula
			∫_{[0, T]^n} f(t_1, …, t_n) \d B_{t_1} … \d B_{t_n}  =  ∫_0^T ⋯ ∫_0^T f(t_1, …, t_n) \d B_{t_n} … \d B_{t_1} .
		\stopformula
	\stoptheorem

	Example\cite[short][HwangKuoSaitôZhai2016]: For the new integral, \m{∫_0^T \brnd[∫_0^T B_u \d u] \d B_v = ∫_0^T \brnd[∫_0^T B_u \d B_v] \d u}.
\stopslide

\startslide [title={A generalization of Itô isometry}]

	\starttheorem [title={\cite[short][KuoSaeTangSzozda2012]}]
		Let \m{ϕ} be an analytic function on \m{ℝ}. Then under integrability conditions and for each \m{t},
		\startformula
			𝔼\bsqr[{\brnd[∫_0^t ϕ(B_T - B_s) \d B_s]}^2]  =  ∫_0^t 𝔼\bsqr[{\brnd[ϕ(B_T - B_s)]}^2] \d s
		\stopformula
	\stoptheorem

	\starttheorem [title={\cite[short][KuoSaeTangSzozda2013]}]
		Let \m{f} and \m{ϕ} be \m{C^1} functions on \m{ℝ}. Then
		\startformula \startalign
			\NC  𝔼\bsqr[{\brnd[∫_0^T f(B_t) ϕ(B_T - B_t) \d B_t]}^2]
			\NC  =  ∫_0^T 𝔼\bsqr[{\brnd[f(B_t) ϕ(B_T - B_t)]}^2] \d t
			\NR  \NC  \NC  \quad + 2 ∫_0^T∫_0^{\bad{t}} 𝔼\bsqr[f(B_s) ϕ'(B_T - B_s) f'(B_s) ϕ(B_T - B_s)] \d s \d t .
		\stopalign \stopformula
	\stoptheorem
\stopslide

\startslide [title={A generalization of Girsanov theorem}]

	\starttheorem[title={\cite[short][KuoPengSzozda2013b]}]
		Let \m{X_{\argdotsub}} and \m{Y^{\argdotsup}} be continuous square-integrable stochastic processes such that \m{X_{\argdotsub}} is adapted and \m{Y^{\argdotsup}} is instantly independent.

		Then the translated stochastic process \m{W_{\argdotsub} = B_{\argdotsub} + ∫_0^{\argdotsup} (X_t + Y^t) \d t} is a near-martingale under the probability measure \m{\widetilde{ℙ}} defined by the Radon-Nikodym derivative \m{\frac{\d \widetilde{ℙ}}{\d ℙ} = 𝓔^{(X + Y)}_T}.
	\stoptheorem
\stopslide

\stopmode

\stopsection

\page    % Needed for correct color transition



%%%%%%%%%%%%%%%%%%%%%%%%%%%
% Large deviations theory %
%%%%%%%%%%%%%%%%%%%%%%%%%%%

\setupbackgrounds [page] [
	background={color,backgraphics,foreground},
	backgroundcolor=\getvariable{document}{color-background-2}]
\startcolor [\getvariable{document}{color-foreground-2}]    % text color

\startsection [title={Large deviations theory}, reference=sec:large-deviations-theory]

\startmode [presentation]

\startslide[title={What is it about?}]

	\startitemize [4]

		\item  A theory to find probabilities of rare events that decay exponentially.

		\item  Started by Swedish actuarials Fredrik Esscher, Harald Cramér, Filip Lundberg.

		\item  Unified by Varadhan in his 1966 paper \cite[short][Varadhan1966].

		\item  Example: A problem faced by the insurance industry.
			\startitemize [5, joinedup]
				\item  Value of claims received on the \m{n}th day: \m{X_n} \$.
				\item  Steady income from premium: \m{x} \$/day.
				\item  Planning period: \m{n} days.
				\item  Average expenditure: \m{\overline{X}_n = \frac1n ∑_{j = 1}^n X_j} \$/day.
				\item  \emph{Question}: How should the company decide on the premium?
				\item  \important\emph{Idea}: {Determine \m{x} such that \m{ℙ\bcrl[\overline{X}_n > x] < ε} (specified)}.
			\stopitemize
	\stopitemize
\stopslide

\startslide [title={Insurance problem: setup}]

	\startitemize [n]
		\item  Let the following hold:
			\startitemize [5, joinedup]
				\item  \m{(Ω, ℱ, ℙ)} is a probability space.
				\item  \m{(X_n)} is a sequence of i.i.d. random variables on \m{(Ω, ℱ, ℙ)} with finite moment generating function \m{M}.
				\item  \important{\m{𝔼 X_1 = m}}, \m{𝕍 X_1 = σ^2}, and \m{X_1 ∼ μ}.
				\item  \m{\overline{X}_n = \frac1n ∑_{j = 1}^n X_j}.
			\stopitemize

		\item  Asymptotic behavior of \m{\overline{X}_n}:
			\startitemize [5, joinedup]
				\item  \important{Weak law of large numbers: \m{\overline{X}_n \xrightarrow{ℙ} m}}.
				\item  Central limit theorem: \m{\sqrt{n} (\overline{X}_n - m) \xrightarrow{𝓓} 𝓝(0, σ^2)}.
			\stopitemize

		\item  What is the rate for LLN?

		% \item  We want to \quotation{control large deviations from the mean}.
	\stopitemize
\stopslide

\startslide [title={Insurance problem: large deviation bounds}]

	\startitemize [n]
		\item  For \m{x > m} and an arbitrary \m{θ > 0}, we get
			\startformula
				ℙ\bcrl[\overline{X}_n ≥ x]
				=  ℙ\bcrl[e^{θn\overline{X}_n} ≥ e^{θnx}]
				≤  e^{-θnx} 𝔼\brnd[e^{θn\overline{X}_n}]
				=  e^{-θnx} M_{X}(θ)^n
				=  e^{-n(θx - \log M_{X}(θ))} .
			\stopformula

		\item  Since \m{θ} was arbitrary, we have
			\startformula
				ℙ\bcrl[\overline{X}_n ≥ x]
				≤  \inf_θ e^{-n(θx - \log M_{X}(θ))}
				=  e^{-n \sup_θ (θx - \log M_{X}(θ))}
				=: e^{-n I(x)} .
			\stopformula

		\item  Generalizing, we get the \important{large deviation upper bound}
			\startformula
				\important{\limsupm \frac1n \log ℙ\bcrl[\overline{X}_n ∈ F]  ≤  -\inf_F I  \qquad  ∀ F \text{ closed}} .
			\stopformula

		\item  We can also obtain a \important{large deviation lower bound} using an exponential change of measure
			\startformula
				\important{\liminfm \frac1n \log ℙ\bcrl[\overline{X}_n ∈ G]  ≥  -\inf_G I  \qquad  ∀ G \text{ open}} .
			\stopformula

		\item  We informally write \m{ℙ\bcrl[\overline{X}_n ∈ \d x] ≍ e^{-n I(x)} \d x} for \m{x ∈ ℝ}.
	\stopitemize
\stopslide

\startslide [title={Definition of large deviation principle}]

	\startitemize [4]

		\item  The setup: \m{(X_n)} is a stochastic process on \m{(Ω, ℱ, ℙ)} taking values in a Polish space \m{(𝓧, d)}.

		\item  A function \m{I: 𝓧 → [0, ∞]} is called a \important{rate function} if it has compact lower level sets.

		% \item  \m{I} is lower semicontinuous and attains its infimum on every nonempty closed set.

		% \item  For any Borel set \m{E}, denote \m{I(E) = \inf I(x)}.
	\stopitemize

	\startdefinition
		\m{(X_n)} is said to satisfy the \important{large deviation principle on \m{𝓧} with rate function \m{I}} if the large deviation \okay{upper} and \okay{lower} bounds hold.
		% \startformula \startalign[n=4]
		% 	\NC  \text{(upper bound)}  \qquad  \NC  \limsupm \frac1n \log ℙ\bcrl[\overline{X}_n ∈ F]  \NC ≤  -\inf_{F} I  \NC  \qquad  ∀ F \text{ closed}  \NR
		% 	\NC  \text{(lower bound)}  \qquad  \NC  \liminfm \frac1n \log ℙ\bcrl[\overline{X}_n ∈ G]  \NC ≥  -\inf_{G} I  \NC  \qquad  ∀ G \text{ open}
		% \stopalign \stopformula
	\stopdefinition

	Example
	\starttheorem [title={\cite[short][Cramér1938]}]
		Let \m{(X_n)} be a sequence of i.i.d. real random variables with finite moment generating function \m{M}. Then \m{(\overline{X}_n)} follows the large deviation principle with rate function \m{I(x) = \sup_θ \brnd[θ x - \log M(θ)]}.
	\stoptheorem
\stopslide

\startslide [title={Applications of the Cramér theorem}]

	Rate functions for some common distributions
	% \placetable[force,none]{}{%    % This centers the table
	\starttabulate [|M|M|M|]
		\FL
		\NC  \text{Distribution}  \VL  M(θ)  \VL  I(x)  \NR
		\FL
		\NC  \text{Bernoulli}(p)  \VL  1 - p + p e^θ  \VL  \brnd[x \log x + (1-x) \log(1-x) - {\brnd[x \log \frac{1-p}{p} + \log{p}]}] 𝟙_{[0, 1]}(x) + ∞ 𝟙_{[0, 1]^∁}(x)  \NR
		\NC \VL \VL \NR
		\NC  \text{Poisson}(λ)  \VL  e^{λ(e^θ - 1)}  \VL  \brnd[λ - x + x \log \frac{x}{λ}] 𝟙_{[0, ∞)}(x) + ∞ 𝟙_{(-∞, 0)}(x)  \NR
		\NC \VL \VL \NR
		\NC  \text{Exp}(λ)   \VL  \inv[{\brnd[1 - \frac{θ}{λ}]}]  \VL  \brnd[λx - 1 + x \log(λ x)] 𝟙_{[0, ∞)}(x) + ∞ 𝟙_{(-∞, 0)}(x)  \NR
		\NC \VL \VL \NR
		\NC  𝓝(m, σ^2)  \VL  e^{m θ + \frac12 σ^2 θ^2}  \VL  \frac{(x - m)^2}{2 σ^2}  \NR %\NR
		% \NC  χ^2(k)  \NC  (1 - 2 θ)^{-\frac{k}{2}}  \NC  \frac12 \brnd[x - k + k \log\frac{k}{x}]  \NR
		\BL
	\stoptabulate%}
	% ToDo: Add more.
	% ToDo: Add the computation in the appendix.
\stopslide

\startslide [title={The Schilder theorem}]

	\startitemize [4]

		\item  Aim: Estimate the probability that a scaled-down sample path of a \important{Brownian motion} will stray far from the mean path.

		\item  Let \m{C_x} denote the set of continuous functions from \m{[0, T]} to \m{ℝ^d} starting at \m{x}, and

			let \m{\text{CM}_x = \bcrl[ω ∈ C_x : ω \text{ is absolutely continuous and } ω_t' ∈ L^2{[0, T]}]}.

		\item  Let \m{B_{\argdotsub}} be a \m{d}-dimensional Brownian motion, so \m{B_{\argdotsub} ∈ C_0 = C_0([0, T]; ℝ^d)}

		\item  Let \m{\frac{1}{\sqrt{n}} B_t ∼ W^{(n)}}. Then \m{W^{(n)} = 𝓝\brnd[0, \frac{t}{n}] ⟹ δ_0} as \m{n → ∞}.
		% This scales down the variance of the Brownian motion to \m{ε t}.

	\stopitemize

	\starttheorem [title={\cite[short][Schilder1966]}]
		On the Banach space \m{\brnd[C_0, {\norm[⋅]}_∞]}, the sequence of probability measures \m{\brnd[W^{(n)}]} satisfies LDP with the rate function \m{I : C_0 → \clsr[ℝ]} given by
		\startformula
			I(ω)  =  \brnd[\frac12 ∫_0^T {\abs[ω_t']}^2 \d t] 𝟙_{\text{CM}_0}(ω) + ∞ 𝟙_{\text{CM}_0^∁}(ω)
		\stopformula
	\stoptheorem
\stopslide

\startslide [title={The Freidlin–Wentzell theorem}]

	\startitemize [4]

		\item  Aim: Estimate the probability that a scaled-down sample path of an \important{Itô diffusion} will stray far from the mean path.

		\item  Let \m{X^{(n)}_{\argdotsub}} be the solution of the \m{d}-dimensional stochastic differential equation

			\important{\m{\d X^{(n)}_t = m(X^{(n)}_t) \d t + \frac{1}{\sqrt{n}} σ(X^{(n)}_t) \d B_t , \ X^{(n)}_0 = x}},
			where \m{m} and \m{σ} are sufficiently nice.

		\item  Let \m{W^{(n)}_x} denote the law of \m{X^{(n)}_{\argdotsub}} starting at \m{x}.

		\item  As \m{n → ∞}, \m{W^{(n)}_x ⟹ δ_ξ}, where \m{ξ} solves the ODE \m{\dot{ξ}(t) = m(ξ(t)), \ ξ(0) = x}.

		% \item  Let \m{\text{CM}_x = \bcrl[ω ∈ C_x : ω \text{ is absolutely continuous and } ω_t' ∈ L^2{[0, T]}]}.
	\stopitemize

	% Generalizes the Schilder theorem for Itô diffusions.

	\starttheorem [title={\cite[short][FreidlinWentzell2012]}]
		For any fixed \m{x}, the sequence of probability measures \m{\brnd[W^{(n)}_x]} satisfies LDP with the rate function \m{I_x : C_0 → \clsr[ℝ]} given by
		\startformula
			I_x(ω)  =  \brnd[\frac12 ∫_0^T {\inn[ω_t' - m(ω_t), {\inv[A](ω_t)}{\brnd[ω_t' - m(ω_t)]}]} \d t] 𝟙_{\text{CM}_x}(ω) + ∞ 𝟙_{\text{CM}_x^∁}(ω) ,
		\stopformula
		where \m{A = σ σ^*}.
	\stoptheorem
\stopslide

\stopmode

\stopsection

\page    % Needed for correct color transition



%%%%%%%%%%%%%%
% Conclusion %
%%%%%%%%%%%%%%

\setupbackgrounds [page] [
	background={color,backgraphics,foreground},
	backgroundcolor=\getvariable{document}{color-background-0}]
\startcolor [\getvariable{document}{color-foreground-0}]    % text color

\startsection [title={The way forward}, reference=sec:directions]

\startmode [presentation]

\startslide [title={Possible research directions}]

	\startcolor [\getvariable{document}{color-foreground-1}]    % text color
	\startitemize [4]

		\item  Identify the class of integrable processes under the new integral.

		\item  Give a broader generalization of the Itô isometry for the new integral.

		\item  Provide a broader generalization of the Girsanov theorem.

		\item  Formulate an extension of the new integral to stochastic differential equations with anticipating coefficients.

		\item  Develop the near-Markov property for the new integral.
	\stopitemize

	\startcolor [\getvariable{document}{color-foreground-2}]    % text color
	\startitemize [4]

		\item  Prove Freidlin–Wentzell type results for SDEs with anticipating initial conditions.

		\item  Study LDP results for SDEs with anticipating coefficients involving the new integral.

		\item  Analyze LDP for linear SPDEs with anticipating initial conditions.
	\stopitemize
\stopslide

\startslide

	\startalign [middle]

		\blank[4*line]

		{\tfd Thank you!}

	\stopalign
\stopslide

\stopmode

\stopsection

\stopbodymatter




%%%%%%%%%%%%%%%
% Back matter %
%%%%%%%%%%%%%%%

\startbackmatter

\startsubject [title={Appendix}, reference=sub:appendix]

\startmode [presentation]

\startslide [title={Laplace principle and equivalence to LDP}]

	\startdefinition [title={Laplace principle}]
		\m{(X_n)} is said to satisfy the \important{Laplace principle on \m{𝓧} with rate function \m{I}} if for all bounded continuous functions \m{h}, we have
		\startformula
			\lim  \frac1n \log 𝔼\exp(-n h(X_n))  =  \inf_𝓧 (h + I)
		\stopformula
	\stopdefinition

	\starttheorem
		\m{(X_n)} satisfies LP on \m{𝓧} with rate function \m{I} if and only if \m{(X_n)} satisfies LDP on \m{𝓧} with the same rate function \m{I}.
	\stoptheorem

	\bold{Some important results}
	\startitemize [5, nowhite]
		\item  Uniqueness of the rate function.
		\item  Continuity principle.
			% Let \m{f:𝓧 → 𝓨} be a continuous map of Polish spaces. Then \m{(f(X_n))} satisfies Laplace principle with rate function \m{J(y) = \inf \bcrl[I(x): x ∈ {\inv[f]}(y)]}.
		\item  Superexponential approximation preserves Laplace principle.
		  % If \m{(Y_n)} is superexponentially close to \m{(X_n)}, then \m{(Y_n)} satisfies Laplace principle with the same rate function \m{I}.
	\stopitemize
\stopslide

% \startslide[title={An application of Schilder theorem}]

% 	\startitemize [4]

% 		\item  \m{W\brnd[C_0 ∖ B_r(0)] = \bcrl[{\norm[B_{\argdotsub}]}_∞ > r]}

% 		\item  \m{\bcrl[{\norm[B_{\argdotsub}]}_∞ > r]}
% 	\stopitemize
% \stopslide

% \startslide[title={Using the continuity principle to extend the Schilder theorem}]
% 	TODO
% \stopslide

% \startslide [title={Sanov theorem}]

% 	\startitemize [4]

% 		\item  \bold{Aim}: Provide a bound on the probability of observing an atypical sequence of samples from a given probability distribution.

% 		\item
% 	\stopitemize
% \stopslide

\startslide

	\startalign [middle]

		\blank[4*line]

		{\tfd Thank you!}

	\stopalign
\stopslide

\stopsubject

\startslide [title={Bibliography}]
	\placelistofpublications
\stopslide
\stopmode
\stopbackmatter

\stoptext
