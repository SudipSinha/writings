\environment env-talks


%%%%%%%%%%%%%%%%%%%%%%%%%%%%%%%%%%%%
% This is where the document starts.
%%%%%%%%%%%%%%%%%%%%%%%%%%%%%%%%%%%%

\starttext

\startfrontmatter

%%%%%%%%%%%%%%%%
% Front matter %
%%%%%%%%%%%%%%%%

\setupbackgrounds[page][
	background={color,backgraphics,foreground},
	backgroundcolor=\getvariable{document}{color-background-0}]
\startcolor[\getvariable{document}{color-foreground-0}]    % text color


% \startmode [handout]
% \startcolumns
% \stopmode


% Introduction
\startcolor [\getvariable{document}{color-foreground-0}]    % text color


\startmode [presentation]

\startslide

\startalign [middle]

	{\tfd
		\color[\getvariable{document}{color-foreground-1}]{Generalization of stochastic calculus}\\
		and its applications in\\
		\color[\getvariable{document}{color-foreground-2}]{large deviations theory}}

	\blank[2*line]

	{\tfb \getvariable{document}{author}}

	\blank[line]

	{\tfa \getvariable{document}{date}}

	\blank[2*line]

	Advisors

	\color[\getvariable{document}{color-foreground-1}]{Prof Hui-Hsiung Kuo}

	\color[\getvariable{document}{color-foreground-2}]{Prof Padmanabhan Sundar}

\stopalign

\stopslide


% Table of contents
\startslide[title={Outline}]
	\placecontent
\stopslide
\stopmode

% \startmode [manuscript]

% This presentation is going to be on two topics:
% \startitemize[n,nowhite,after]
% 	\item  Generalization of stochastic integrals developed primarily by Professor H.-H. Kuo
% 	\item  Applications of generalization in large deviations theory
% \stopitemize

% \stopmode

\stopfrontmatter



\startbodymatter

%%%%%%%%%%%%%%%%%%%%%%%%%%%%%%%%%%%%%%%%%
% Generalization of stochastic calculus %
%%%%%%%%%%%%%%%%%%%%%%%%%%%%%%%%%%%%%%%%%

\setupbackgrounds[page][
	background={color,backgraphics,foreground},
	backgroundcolor=\getvariable{document}{color-background-1}]
\startcolor[\getvariable{document}{color-foreground-1}]    % text color

\startsection[title={Stochastic calculus}, reference=sec:stochastic-calculus]


\startmode [presentation]

\startslide [title={Quick revision}]

	\startitemize[n]
		\item  Properties of Brownian motion \m{B(t)}
			\startitemize[1]
				\item  starts at 0
				\item  continuous paths
				\item  independence of increments
				\item  \m{B(t) - B(s) ∼ N(0, t - s)}
				\item  infinite linear variation
				\item  finite quadratic variation \m{[B(t), B(s)] = t ∧ s}.
				\item  is a martingale
			\stopitemize
		\item  Naive stochastic integration w.r.t. \m{B(t)}: not possible
		% \item  Paley–Wiener integral can handle with
	\stopitemize

\stopslide

\startslide [title={Itô integral: \m{f ∈ L^2}}]
	Definition


	Properties of the associated process \m{X_t = ∫_0^t f(t) \d B(t)}
	\startitemize[n]
		\item  continuity
		\item  martingale
	\stopitemize

\stopslide

\startslide [title={Itô integral: \m{f ∈ 𝓛^2}}]
	Definition

	Properties of the associated process \m{X_t = ∫_0^t f(t) \d B(t)}
	\startitemize[n]
		\item  continuity
		\item  local martingale
	\stopitemize

\stopslide

\startslide [title={Itô formula}]
\stopslide

\startslide [title={Stochastic differential equations}]
\stopslide

\startslide [title={Girsanov theorem}]
\stopslide

% \startmode [manuscript]

% This is the first part and we are going to talk about generalization of stochastic integrals developed primarily by Professor H.-H. Kuo.

% \stopmode


\startslide [title={Itô integral: \m{f ∈ 𝓛^2}}]

	This is a citation \cite[HKSZ2016].

	\startitemize[8,columns,two]
		\item One
		\item Two
		\item Three
		\item Four
	\stopitemize

\stopslide


\startslide [title={Itô isometry: \m{f ∈ L^2}}]
	bla bla bla
\stopslide

\startslide [title={Differential formula (Itô, 1944 TODO:ref)}]
	bla bla bla
\stopslide

\stopmode

\stopsection



%%%%%%%%%%%%%%%%%%%%%%%%%%%
% Large deviations theory %
%%%%%%%%%%%%%%%%%%%%%%%%%%%

\setupbackgrounds[page][
	background={color,backgraphics,foreground},
	backgroundcolor=\getvariable{document}{color-background-2}]
\startcolor[\getvariable{document}{color-foreground-2}]    % text color

\startsection[title={Large deviations theory}, reference=sec:large-deviations-theory]


\startmode [presentation]

\startslide [title={Introduction}]

\stopslide

\startslide [title={Weak convergence of measures}]

\stopslide

\startslide [title={Laplace principle}]

\stopslide

\startslide [title={Cramér theorem}]
	This is the first one:
	\theorem All conjectures are interesting.

	This is the second one:
	\starttheorem
		Let \m{X_{1}, X_{2}, \dots}be a series of i.i.d. real random variables with finite logarithmic moment generating function, e.g. \m{Λ(t) < ∞ \, ∀ t ∈ ℝ}.

		Then the Legendre transform of \m{Λ}, \m{Λ^* = \sup_{t ∈ ℝ} (t x - Λ(t))} satisfies
		\startformula
			\lim_{n → ∞}  \frac{1}{n}  \log ℙ{\brnd[∑_{i=1}^n X_i ≥ n x]}  =  - Λ^*(x)  \quad  ∀ x > 𝔼(X_1)
		\stopformula
	\stoptheorem
\stopslide

\startslide [title={Sanov theorem}]

\stopslide

\startslide [title={Schilder theorem}]

\stopslide

\startslide [title={Freidlin–Wentzell theorem}]

\stopslide

\startslide [title={Freidlin–Wentzell theorem}]

\startcolumns[n=2]
{\tfb Column 1}

\input ward

\column

{\tfb Column 2}

\input weisman
\stopcolumns

% \startitemize[8,columns,two]
% 	\item One
% 	\item Two
% 	\item Three
% \stopitemize

\stopslide

\stopmode

\stopsection



%%%%%%%%%%%%%%
% Conclusion %
%%%%%%%%%%%%%%

\setupbackgrounds[page][
	background={color,backgraphics,foreground},
	backgroundcolor=\getvariable{document}{color-background-0}]
\startcolor[\getvariable{document}{color-foreground-0}]    % text color

\startsection[title={Conclusion}, reference=sec:conclusion]


\startmode [presentation]

\startslide [title={Possible areas of interest}]
	\startitemize[3]
		\item  Extension to SDEs with anticipating coefficients
		\item  Near-Markov property
		\item  Girsanov theorem for generalized integration
		\item  Freidlin-Wintzell type result for SDEs with anticipating initial conditions
	\stopitemize

	\framed [width=local, height=8em, autowidth=force] {    % wiki/Framed
		\input ward
	}
\stopslide

\stopmode

\stopsection

\stopbodymatter




%%%%%%%%%%%%%%%
% Back matter %
%%%%%%%%%%%%%%%

\startbackmatter
\startmode [presentation]
\startslide [title={Bibliography}]
	\placelistofpublications
\stopslide
\stopmode
\stopbackmatter

\stoptext
