% macros=mkvi

% Significant portions have been stolen from the following sources
% https://www.tug.org/pracjourn/2006-2/schmitz/
% https://www.tug.org/pracjourn/2006-4/mahajan/

% Unit conversion: https://www.joomlasrilanka.com/web-design-development-blog/web-design-font-size-measurements-convert-points-pixelsems-percentages-web-designing/



% This is the variables section. Ideally, these are the only modifications that should be made.

% Version and mode
\setvariables [document] [version=final]    % {final, concept, temporary}
\setvariables [document] [mode=presentation]    % {presentation, manuscript, handout}

% Colors
% https://coolors.co/app
\setvariables [document] [color-link-external=royalblue]
\setvariables [document] [color-link-internal=violetred]
\setvariables [document] [color-background-0=white]
\setvariables [document] [color-background-1=mistyrose]
\setvariables [document] [color-background-2=whitesmoke]
\setvariables [document] [color-foreground-0=deepskyblue4]
\setvariables [document] [color-foreground-1=firebrick4]
\setvariables [document] [color-foreground-2=darkslategray]

% Fonts
% Options: palatino, xitsbidi, euler
\setvariables [document] [font=palatino]
\setvariables [document] [fontsize-presentation=38pt]
\setvariables [document] [fontsize-document=12pt]

% Information
\setvariables [document] [title={A generalization of stochastic integrals and its applications to large deviations theory}]
\setvariables [document] [subtitle={General examination presentation}]
\setvariables [document] [author={Sudip Sinha}]
\setvariables [document] [date={2019-04-05}]
\setvariables [document] [keyword={mathematics, probability, stochastic, integral, calculus, large deviations, applications}]



\startenvironment presentation


\mainlanguage[en]

\version[\getvariable{document}{version}]
\enablemode[\getvariable{document}{mode}]

\setupcolors[state=start]    % otherwise you get greyscale
\usecolors[x11]    % X11 colors


% See Wiki/command/setupinteraction
\setupinteraction [
	state=start,  % make hyperlinks clickable(active)
	menu=on,
	click=yes,
	style=normal,
	color=\getvariable{document}{color-link-external},
	contrastcolor=\getvariable{document}{color-link-internal},
	% openaction=ToggleViewer,    % Make the PDF open in 'Fit width' mode
	focus=width,    % Make PDF links open in 'Fit width' mode
	title=\getvariable{document}{title},
	subtitle=\getvariable{document}{subtitle},
	author=\getvariable{document}{author},
	date=\getvariable{document}{date},
	keyword=\getvariable{document}{keyword}]


% General font setup
% https://wiki.contextgarden.net/Protrusion
% \loadtypescriptfile[mathdesign]
\usesymbols[fontawesome]    % https://fontawesome.com/icons

% Interesting: https://tex.stackexchange.com/questions/429621/context-how-to-install-and-use-new-opentype-math-fonts

\definefontfeature
	[default]
	[default]
	[protrusion=quality, expansion=quality]
\setupalign [hanging, hz, lesshyphenation, tolerant, stretch]
\setupbodyfont [\getvariable{document}{font}, \getvariable{document}{fontsize-document}]


% Bibliography (https://www.contextgarden.net/Bibliography_mkiv)
\usebtxdataset[../bib.bib]
\usebtxdataset[../Itô.bib]
\usebtxdataset[../Kuo-integral.bib]
\usebtxdataset[../large-deviations.bib]
\usebtxdefinitions[apa]    % apa or aps
% Use \cite[short][#key] to get XXX19 type citations


% Highlighting
\definehighlight [comment] [color=darkgray]
\definehighlight [good] [color=darkolivegreen4]
\definehighlight [okay] [color=indianred]
\definehighlight [bad] [color=orangered]
\definehighlight [ugly] [color=lightred]
\definehighlight [important] [color=darkblue]

% Emoji
\def \smile{\symbol[fontawesome][smile]}
\def \meh{\symbol[fontawesome][meh]}
\def \frown{\symbol[fontawesome][frown]}

% Tables
% https://source.contextgarden.net/tabl-tbl.mkiv TaBlE compatible specifications
\setuptabulate[rulecolor=\getvariable{document}{color-foreground-0}]


% Mathematics

% I have not completely figured out a way to create a dot between \cdot and \bullet
% The following might provide a way, but is too primitive.
% The problem is that the scaling is about the bottom of the line.
\def \argdotsub{\scale[factor=2]{∙}}    % wiki/Command/scale
\def \argdotmid{\raisebox{0.125em}\vbox{\scale[factor=2]{∙}}}
\def \argdotsup{\raisebox{0.125em}\vbox{\scale[factor=2]{∙}}}
% \def \argdot{\raisebox{-0.2em}\vbox{\scale[factor=6]{⋅}}}
% \def \argdotu{\raisebox{-0em}\vbox{\scale[factor=6]{⋅}}}
% \def \argdot{\raisebox{-0.2em}\vbox{\scale[factor=6]{⋅}}}
% \scale[scale=2]{\cdot}} + Z(\scale[sx=3, sy=2]{⋅})

% Caution: spaces do not work between the elements)
\def \delim[#arguments][#left][#right]{\left#left #arguments \right#right}

% Intervals
\def \intoo[#arguments]{\delim[#arguments][(][)]}
\def \intoc[#arguments]{\delim[#arguments][(][]]}
\def \intco[#arguments]{\delim[#arguments][[][]]}
\def \intcc[#arguments]{\delim[#arguments][[][]]}

% Brackets
\def \brnd[#arguments]{\delim[#arguments][(][)]}
\def \bcrl[#arguments]{\delim[#arguments][\{][\}]}
\def \bsqr[#arguments]{\delim[#arguments][[][]]}

% Commonly used delimiters
\def \inn[#arguments]{\delim[#arguments][⟨][⟩]}
\def \norm[#arguments]{\delim[#arguments][‖][‖]}
\def \abs[#arguments]{\delim[#arguments][|][|]}
\def \floor[#arguments]{\delim[#arguments][⌊][⌋]}
\def \ceil[#arguments]{\delim[#arguments][⌈][⌉]}

% Algebra
\def \linspan[#arguments]{\mfunction{span} \delim[#arguments][\{][\}]}
\def \inv[#arguments]{#arguments^{-1}}

% Differentials
\def \d{\,\text{d}}
\def \D{\,\text{D}}

% Topology
\def \clsr[#set]{\overline{#set}}
\def \intr[#set]{{#set}^°}
% \def \intr[#set]{\mathring{#set}}
\def \bdy[#set]{∂ #set}

% Limit
\definemathcommand [ess] [limop] {\mfunction{ess}}
\definemathcommand [limsupm] [limop] {\overline{\lim}}
\definemathcommand [liminfm] [limop] {\underline{\lim}}

% Algebra
\definemathcommand [ker] [nolop] {\mfunction{ker}}
\definemathcommand [im] [nolop] {\mfunction{im}}

% Arithmetic
\definemathcommand [gcd] [nolop] {\mfunction{gcd}}
\definemathcommand [lcm] [nolop] {\mfunction{lcm}}
\definemathcommand [dim] [limop] {\mfunction{dim}}

% Probabilistic
\definemathcommand [infinitelyoften] [nolop] {\,\mfunction{i.o.}}
\definemathcommand [eventually] [nolop] {\,\mfunction{ev}}

% Matrices
\definemathmatrix
	[rndmatrix]
	[left={\,\left(\,},right={\,\right)\,}]


% Styling for mathematical environments

%  Randomized frame
\startuseMPgraphic{background:random}
	path p;
	for i = 1 upto nofmultipars :
		p = (multipars[i] topenlarged 16pt bottomenlarged 16pt) randomized 4pt;
		fill p withcolor \getvariable{document}{color-background-0};
		draw p withcolor \MPvar{linecolor}
		withpen pencircle scaled \MPvar{linewidth};
	endfor;
\stopuseMPgraphic

% Backgrounds
\definetextbackground [mathenv] [
	mp=background:random,
	location=paragraph,
	rulethickness=1pt,
	framecolor=\getvariable{document}{color-foreground-0},
	width=local,
	leftoffset=1em,
	rightoffset=1em,
	before={\testpage[3]\blank[big]},    % wiki/Command/testpage
	after={\blank[2*big]}]


% Mathematical environments

\defineenumeration [mathematicalstatement] [
	title=yes,    % now we can use title={}
	number=no,    % remove if definitions shoud be numbered
	headstyle=bold,
	headcolor=\getvariable{document}{color-foreground-0},
	width=fit,
	alternative=hanging,  % top
	hang=fit,
	textstyle=italic,
	before=\startmathenv,
	after=\stopmathenv]

\defineenumeration [theorem] [mathematicalstatement] [
	text=Theorem,
	list=all,
	listtext={Theorem }]

\defineenumeration [definition] [mathematicalstatement] [
	text=Definition,
	list=all,
	listtext={Definition }]

\defineenumeration [remark] [mathematicalstatement] [
	text=Remark,
	list=all,
	listtext={Remark }]

\defineenumeration [proof] [mathematicalstatement] [
	text=Proof,
	title=no,
	number=no,
	headstyle=italic,
	style=normal,
	before=,
	after=,
	closesymbol={□}]




%%%%%%%%%%%%%%%%%%%%%%%%%%%%%%%%%%%%%%%%%%%
% This is the markup for the presentation %
%%%%%%%%%%%%%%%%%%%%%%%%%%%%%%%%%%%%%%%%%%%

\startmode [presentation]
\setuppapersize [HD] [HD]
% \setupinteractionscreen[option=max]    % full screen mode (wiki/Command/setupinteractionscreen)
% \showframe

\setupbodyfont [\getvariable{document}{font}, \getvariable{document}{fontsize-presentation}]

\setuplayout [
	width=fit,
	rightmargin=2em,
	leftmargin=2em,
	leftmargindistance=0em,
	rightmargindistance=0em,
	height=fit,
	header=0em,
	footer=0em,
	backspace=4em,
	topspace=0em,
	bottomspace=0em,
	% bottom=0.5em,
	location=singlesided]

% wiki/Command/setuphead
\definehead [slide] [subject] [    % default formatting = subject formatting
	% number=no,
	page=yes,
	style=\tfb,
	alternative=middle,
	before=,    % Negates the effect of setuphead [section].
	aligntitle=yes]

\setuplabeltext [section=Section~]
\setuphead [section] [
	header=empty,    % wiki/Command/setupheader
	page=yes,
	before={\emptylines[4]},    % wiki/Command/emptylines  % \blank[n*big] did not work
	alternative=middle,
	style=\tfd\bold\smallcaps]

\stopmode


% %%%%%%%%%%%%%%%%%%%%%%%%%%%%%%%%%%%%%%%%%
% % This is the markup for the manuscript %
% %%%%%%%%%%%%%%%%%%%%%%%%%%%%%%%%%%%%%%%%%

% \startmode[manuscript]

% \setupframedtexts[location=middle,
%                   before={\blank[line]},
%                   after={\blank[line]}]

% \definehead [slide] [subject] [
% 	style=\tfb,
% 	alternative=middle,
% 	page=yes]

% % \let\startslide\startslidedtext
% % \let\stopslide\stopslidedtext

% % \define[1]\slidetitle{
% % 	\startalignment[middle]
% % 		#1
% % 	\stopalignment
% % 	\blank[small]}

% \setupbackgrounds[page][background={backgraphics,foreground}]

% \stopmode


% %%%%%%%%%%%%%%%%%%%%%%%%%%%%%%%%%%%%%%
% % This is the markup for the handout %
% %%%%%%%%%%%%%%%%%%%%%%%%%%%%%%%%%%%%%%

% \startmode [handout]
% \setuppapersize [A4] [A4]

% \setuplayout [width=fit,
%               rightmargin=0cm,
%               leftmargin=0cm,
%               backspace=2em]

% \setupframedtexts[location=left,
%                   height=.75\textwidth,
%                   width=\textwidth,
%                   before={\blank[line]},
%                   after={\blank[line]}]

% \definehead [slide] [subject] [
% 	style=\tfb,
% 	alternative=middle,
% 	page=yes]

% % \let\startslide\startslidedtext
% % \let\stopslide\stopslidedtext

% % \define[1]\slidetitle{
% % 	\startalignment[middle]
% % 		#1
% % 	\stopalignment
% % 	\blank[small]}

% \setupcolumns[2,distance=2mm]

% \setupbackgrounds [page] [background={backgraphics,foreground}]

% \stopmode

\stopenvironment
