% macros=mkvi    % https://wiki.contextgarden.net/MkVI

\startenvironment *

%%%%%%%%%%%%%
% Variables %
%%%%%%%%%%%%%

% Version and mode
\setvariables [document] [version=temporary]    % {final, concept, temporary}
\setvariables [document] [mode=presentation]    % {presentation, manuscript, handout}
% Colors
\setvariables [document] [color-link-external=royalblue]
\setvariables [document] [color-link-internal=violetred]
% \setvariables [document] [color-background=azure]
\setvariables [document] [color-background=white]
\setvariables [document] [color-foreground=darkslategray]
\setvariables [document] [color-headings=firebrick]
% Fonts
\setvariables [document] [font=palatino]
\setvariables [document] [fontsize=12pt]
% Document specific information
\setvariables [document] [title={Solution Manual}]
\setvariables [document] [subtitle={Using \ConTeXt}]
\setvariables [document] [author={Sudip Sinha}]
\setvariables [document] [date={\date}]
\setvariables [document] [keyword={mathematics, solutions}]


%%%%%%%%%%%%
% Preamble %
%%%%%%%%%%%%

% mode=mkiv
\mainlanguage [en]
\version [\getvariable{document}{version}]

% NOTE: export cannot deal with 'part's as of 2019-04-05
% \setupbackend [export=yes]    % export to other formats

\setupcolors [state=start]       % otherwise you get greyscale
\usecolors [x11]    % X11 colors, use \showcolor[x11] to get a list

% See wiki/Command/setupinteraction
\setupinteraction [
	state=start,  % make hyperlinks active
	menu=on,
	click=yes,
	style=normal,
	color=\getvariable{document}{color-link-external},   % Seems like this is the only color working
	contrastcolor=\getvariable{document}{color-link-internal},   % This is not working
	openaction=FitWidth,    % Make the PDF open in 'fit width' mode
	focus=standard,    % stop switching to 'fit page' mode when clicking an interdocument hyperlink
	title=\getvariable{document}{title},
	subtitle=\getvariable{document}{subtitle},
	author=\getvariable{document}{author},
	date=\getvariable{document}{date},
	keyword=\getvariable{document}{keyword}]
\setupinteractionscreen [option=bookmark]    % wiki/Command/setupinteractionscreen

\placebookmarks
	[part,chapter,section,subsection]    % creates bookmarks for the specified headers
	[chapter]    % but opens only chapter ones by default

% Export
% \setupexport [
% 	hyphen=yes,
% 	% This metadata is used by ConTeXt’s ePub script
% 	% title, subtitle and author are taken from \setupinteraction, if not set
% 	% title={Solutions Manual},
% 	% subtitle={Using ConTeXt},
% 	% author={Sudip Sinha}]t

% \settaggedmetadata [
% 	% here you can set as many metadata entries as you like, but you need to process them yourself
% 	title={Solution Manual},
% 	subtitle={Using ConTeXt},
% 	author={Sudip Sinha},
% 	version={2019-04-05}] % \date doesn’t expand

% For US paper, use [letter][letter]; the sensible default is [A4][A4] (A4 typesetting, printed on A4 paper)
\setuppapersize [A4] [A4]
% wiki/Layout
\setuplayout [
	topspace=4em,
	backspace=6em,
	% leftmargin=2em,
	header=2em,
	footer=2em,
	height=middle,
	width=middle]
% \showframe    % show the layout

% useful urls
\useURL[author-personal-email][mailto:SudipSinha.Bappa@Gmail.com][][SudipSinha.Bappa@Gmail.com]
\useURL[author-official-email][mailto:ssinha4@lsu.edu][][ssinha4@lsu.edu]
\useURL[author-home][https://SudipSinha.GitHub.io][][homepage]

% headers and footers
% \setupfooter[style=\it]
% \setupfootertexts[\date\hfill \ConTeXt\ template]
% \setuppagenumbering[location={header,right}, style=bold]

% Formatting sections
\setuphead [chapter] [
	color=\getvariable{document}{color-headings},
	alternative=margin,
	style=\tfc\smallcaps]

\setuphead [section] [
	color=\getvariable{document}{color-headings},
	alternative=margin,
	style=\tfb]

\setuphead [subsection] [
	color=\getvariable{document}{color-headings},
	alternative=margin,
	style=\tfa]

% \setuphead[section,chapter,subject][color=headingcolor]
% \setuphead[section,subject][style={\ss\bfa},
%   before={\bigskip\bigskip}, after={}]
% \setuphead[chapter][style={\ss\bfd}]
% \setuphead[title][style={\ss\bfd},
%   before={\begingroup\setupbodyfont[14.4pt]},
%   after={\leftline{\ss\tfa A. U. Thor $\langle$\from[author-email]$\rangle$}
%          \bigskip\bigskip\endgroup}]

% \setupitemize[inbetween={}, style=bold]

% % set inter-paragraph spacing
% \setupwhitespace[medium]
% % comment the next line to not indent paragraphs
% \setupindenting[medium, yes]

% Contents (https://www.contextgarden.net/Table_of_Contents)
\setupcombinedlist [content] [list={part, chapter, section}]

% Bibliography
% Bibliography (https://www.contextgarden.net/Bibliography_mkiv)
\usebtxdataset[../bibliographies/default.bib]
\usebtxdataset[../bibliographies/number-theory.bib]
\usebtxdataset[../bibliographies/combinatorics.bib]
\usebtxdataset[../bibliographies/Itô.bib]
\usebtxdataset[../bibliographies/Kuo-integral.bib]
\usebtxdataset[../bibliographies/large-deviations.bib]
\usebtxdefinitions[apa]    % apa or aps
% Use \cite[short][#key] to get XXX19 type citations

% Diagrams
\usemodule[tikz]
\usetikzlibrary[cd]

% Description environment
\definedescription [dscr] [
	headstyle=bold,
	style=normal,
	align=flushleft,
	alternative=hanging,
	width=broad,
	margin=1em]

% Highlighting
\definehighlight [comment] [color=darkgray]
\definehighlight [good] [color=darkgreen]
\definehighlight [okay] [color=darkorange]
\definehighlight [ugly] [color=orangered]

% % Abbreviations
% \definesynonyms[abbreviation][abbreviations][\infull]
% % \setupsynonyms[style=cap]
% \abbreviation{rv}{random variable}
% \abbreviation{sp}{stochastic process}
% \abbreviation{wlog}{without loss of generality}


% Mathematics

% I have not completely figured out a way to create a dot between \cdot and \bullet
% The following might provide a way, but is too primitive.
% The problem is that the scaling is about the bottom of the line.
\def \argdotsub{\scale[factor=2]{∙}}    % wiki/Command/scale
\def \argdotmid{\raisebox{0.125em}\vbox{\scale[factor=2]{∙}}}
\def \argdotsup{\raisebox{0.125em}\vbox{\scale[factor=2]{∙}}}
% \def \argdot{\raisebox{-0.2em}\vbox{\scale[factor=6]{⋅}}}
% \def \argdotu{\raisebox{-0em}\vbox{\scale[factor=6]{⋅}}}
% \def \argdot{\raisebox{-0.2em}\vbox{\scale[factor=6]{⋅}}}
% \scale[scale=2]{\cdot}} + Z(\scale[sx=3, sy=2]{⋅})

% Caution: spaces do not work between the elements)
\def \delim[#arguments][#left][#right]{\left#left #arguments \right#right}

% Intervals
\def \intoo[#arguments]{\delim[#arguments][(][)]}
\def \intoc[#arguments]{\delim[#arguments][(][]]}
\def \intco[#arguments]{\delim[#arguments][[][]]}
\def \intcc[#arguments]{\delim[#arguments][[][]]}

% Brackets
\def \brnd[#arguments]{\delim[#arguments][(][)]}
\def \bcrl[#arguments]{\delim[#arguments][\{][\}]}
\def \bsqr[#arguments]{\delim[#arguments][[][]]}

% Commonly used delimiters
\def \inn[#arguments]{\delim[#arguments][⟨][⟩]}
\def \norm[#arguments]{\delim[#arguments][‖][‖]}
\def \abs[#arguments]{\delim[#arguments][|][|]}
\def \floor[#arguments]{\delim[#arguments][⌊][⌋]}
\def \ceil[#arguments]{\delim[#arguments][⌈][⌉]}

% Algebra
\def \linspan[#arguments]{\mfunction{span} \delim[#arguments][\{][\}]}
\def \inv[#arguments]{#arguments^{-1}}

% Differentials
\def \d{\,\text{d}}
\def \D{\,\text{D}}

% Topology
\def \clsr[#set]{\overline{#set}}
\def \intr[#set]{{#set}^°}
% \def \intr[#set]{\mathring{#set}}
\def \bdy[#set]{∂ #set}

% Limits
\definemathcommand [ess] [limop] {\mfunction{ess}}
\definemathcommand [limsupm] [limop] {\overline{\lim}}
\definemathcommand [liminfm] [limop] {\underline{\lim}}

% Algebra
\definemathcommand [ker] [nolop] {\mfunction{ker}}
\definemathcommand [im] [nolop] {\mfunction{im}}

% Arithmetic
\definemathcommand [gcd] [nolop] {\mfunction{gcd}}
\definemathcommand [lcm] [nolop] {\mfunction{lcm}}
\definemathcommand [dim] [limop] {\mfunction{dim}}
% \def \congruent[#modulus]{≡_{#modulus}}

% Probability
\definemathcommand [infinitelyoften] [nolop] {\,\mfunction{i.o.}}
\definemathcommand [eventually] [nolop] {\,\mfunction{ev}}

% Matrices
\definemathmatrix
	[rndmatrix]
	[left={\,\left(\,},right={\,\right)\,}]


% Styling for mathematical environments

%  Randomized frame
\startuseMPgraphic{background:random}
	path p;
	for i = 1 upto nofmultipars :
		p = (multipars[i] topenlarged 16pt bottomenlarged 16pt) randomized 4pt;
		fill p withcolor \getvariable{document}{color-background-0};
		draw p withcolor \MPvar{linecolor}
		withpen pencircle scaled \MPvar{linewidth};
	endfor;
\stopuseMPgraphic

% Backgrounds
\definetextbackground [mathenv] [
	mp=background:random,
	location=paragraph,
	rulethickness=1pt,
	framecolor=\getvariable{document}{color-foreground-0},
	width=local,
	leftoffset=1em,
	rightoffset=1em,
	before={\testpage[3]\blank[big]},    % wiki/Command/testpage
	after={\blank[2*big]}]


% Mathematical environments

\definereferenceformat [eqref] [left=(, right=)]    % Referencing formulas

\defineenumeration [mathematicalstatement] [
	title=yes,    % now we can use title={}
	number=no,    % remove if definitions shoud be numbered
	headstyle=bold,
	headcolor=\getvariable{document}{color-foreground-0},
	width=fit,
	alternative=hanging,  % top
	hang=fit,
	style=italic]
	% before=\startmathenv,
	% after=\stopmathenv

\defineenumeration [definition] [mathematicalstatement] [
	text=Definition,
	list=all,
	listtext={Definition }]

\defineenumeration [lemma] [mathematicalstatement] [
	text=Lemma,
	list=all,
	listtext={Lemma }]

\defineenumeration [theorem] [mathematicalstatement] [
	text=Theorem,
	list=all,
	listtext={Theorem }]

\defineenumeration [corollary] [mathematicalstatement] [
	text=Corollary,
	list=all,
	listtext={Corollary }]

\defineenumeration [proposition] [mathematicalstatement] [
	text=Proposition,
	list=all,
	listtext={Proposition }]

\defineenumeration [remark] [mathematicalstatement] [
	text=Remark,
	list=all,
	listtext={Remark }]

\defineenumeration [example] [mathematicalstatement] [
	text=Example,
	list=all,
	listtext={Example }]

\defineenumeration [exercise] [mathematicalstatement] [
	text=Exercise,
	list=all,
	listtext={Exercise }]

\defineenumeration [proof] [mathematicalstatement] [
	text=Proof.,
	% title=no,
	number=no,
	headstyle=italic,    % This is for "Proof"
	titlestyle=italic,    % This is for the title
	style=normal,
	before=,
	after=,
	closesymbol={□}]

\defineenumeration [solution] [mathematicalstatement] [
	text=Solution,
	% title=no,
	number=no,
	headstyle=italic,    % This is for "Solution"
	titlestyle=italic,    % This is for the title
	style=normal,
	before=,
	after=,
	closesymbol={□}]


% Tables
\startsetups table
	\setupTABLE [row] [1] [style=bold]
	\setupTABLE [each] [each] [width=2em, height=2em, align={middle,midddle}, frame=off]
	\setupTABLE [r] [1] [topframe=on, rulethickness=2bp]
	\setupTABLE [r] [2] [topframe=on, rulethickness=0.5bp]
	\setupTABLE [r] [last] [bottomframe=on, rulethickness=2bp]
\stopsetups


% Setup colors
\setupbackgrounds[page][
	background={color, backgraphics, foreground},
	backgroundcolor=\getvariable{document}{color-background}]
\startcolor[\getvariable{document}{color-foreground}]    % text color


% Fonts
% https://wiki.contextgarden.net/Protrusion
% \loadtypescriptfile[mathdesign]
\usesymbols [fontawesome]    % https://fontawesome.com/icons
\definefontfeature
	[default]
	[default]
	[protrusion=quality, expansion=quality]
\usetypescript [\getvariable{document}{font}]
\setupbodyfont
	[reset, \getvariable{document}{font}, \getvariable{document}{fontsize}]
	% alternatives: xitsbidi, euler
\setupalign [hanging, hz, lesshyphenation, tolerant, stretch]

\stopenvironment
