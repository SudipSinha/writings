\startchapter [title={Binomial Theorem}]

	\starttheorem [title={Binomial theorem}]
		\startformula
			(x + y)^n = ∑_{k = 0}^n \binom{n}{k} x^k y^{n - k}.
		\stopformula
	\stoptheorem
	
	\startproof [title={Inductive}]
		Homework.
	\stopproof

	\startproof [title={Combinatorial}]
		Consider the product \m{(x_1 + y_1) (x_2 + y_2) ⋯ (x_n + y_n)}.

		First, note that the expansion consists of \m{2^n} terms, each being a product of \m{n} factors.

		Secondly, each product contains either \m{x_j} xor \m{y_j} for each \m{j ∈ [n]}.

		\comment{For example, \m{(x_1 + y_1) (x_2 + y_2) = x_1 x_2 + x_1 y_2 + y_1 x_2 + y_1 y_2}.}

		Now, we can we choose \m{k} of the \m{x_j}s and \m{n - k} of the \m{y_j}s in \m{\binom{n}{k}} ways, so there are precisely those many terms with m{k} \m{x_j}s and \m{n - k} \m{y_j}s in the expansion.

		Finally, letting \m{x_j = x} and \m{y_j = y} for each \m{j ∈ [n]}, we get the result.
	\stopproof

	\startremark
		This can be generalized to a finite number of experiments.
	\stopremark


\stopchapter
