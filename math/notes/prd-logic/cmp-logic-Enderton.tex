\startcomponent *

\product  prd-logic


\startchapter [title={Set theory}]
	
	For any sets \m{A}, \m{B} and \m{C}, prove the following.
	
	\emph{Note}: Let \m{A} and \m{B} are sets. In order to prove \m{A = B}, it is enough to show \m{A ⊆ B} and \m{A ⊇ B}. In each of the following problems, we show each inclusion separately. Moreover, to show \m{A ⊆ B}, it suffices to show that for \m{x} arbitrary, \m{x ∈ A ⟹ x ∈ B}.

	\startexercise [title={Notes, 1.8}]
		\m{A ∩ B ⊆ A}.
	\stopexercise
	\startsolution
		Let \m{x ∈ A ∩ B} be arbitrary. This means \m{x ∈ A} and \m{x ∈ B}. Therefore \m{x ∈ A}. Since every element in \m{A ∩ B} is also an element of \m{A}, we have \m{A ∩ B ⊆ A}.
	\stopsolution


	\startexercise [title={Notes, 1.10}]
		we have \m{A ∩ ∅ = ∅}.
	\stopexercise
	\startsolution
		\bold{(\m{⊆})}  \qquad
		Let \m{x ∈ A ∩ ∅} be arbitrary. This means \m{x ∈ A} and \m{x ∈ ∅}. But there does not exist \m{x ∈ ∅}. Therefore, the statement is vacuously true.

		\bold{(\m{⊇})}  \qquad
		Now, let \m{x ∈ ∅} be arbitrary. Again, since there does not exist \m{x ∈ ∅}, the statement vacuously true.
	\stopsolution


	\startexercise [title={Notes, 1.13}]
		If \m{A ⊆ B}, then \m{A ∪ B = B}.
	\stopexercise
	\startsolution
		\bold{(\m{⊆})}  \qquad
		Let \m{x ∈ A ∪ B} be arbitrary. This means \m{x ∈ A} or \m{x ∈ B}. If \m{x ∈ A}, then by the condition \m{A ⊆ B}, we obtain \m{x ∈ B}. Therefore, in either case, \m{x ∈ B}.

		\bold{(\m{⊇})}  \qquad
		Let \m{x ∈ B} be arbitrary. Therefore, \m{x ∈ A} or \m{x ∈ B}. Hence \m{x ∈ A ∪ B}.
	\stopsolution


	\startexercise
		If \m{A ⊆ B}, then \m{A ∩ B = A}.
	\stopexercise
	\startsolution
		\startitemize
			
			\sym{\bold{(⊆)}}  \qquad
				Let \m{x ∈ A ∩ B} be arbitrary. This mean \m{x ∈ A} and \m{x ∈ B}. So \m{x ∈ A}.

			\sym{\bold{(⊇)}}  \qquad
				Let \m{x ∈ A} be arbitrary. Then by the hypothesis \m{x ∈ B} since \m{A ⊆ B}. Therefore, \m{x ∈ A} and \m{x ∈ B}, and thus \m{x ∈ A ∩ B}.
		\stopitemize
	\stopsolution


	\startexercise
		If \m{A ∩ B = ∅}, then \m{A ∖ B = A}.
	\stopexercise
	\startsolution
		\startitemize
			
			\sym{\bold{(⊆)}}  \qquad
				Let \m{x ∈ A ∖ B} be arbitrary. Then \m{x ∈ A} and \m{x ∉ B}, so \m{x ∈ A}.

			\sym{\bold{(⊇)}}  \qquad
				Let \m{x ∈ A} be arbitrary. Now, either \m{x ∈ B} or \m{x ∉ B}. If \m{x ∈ B}, then \m{x ∈ A ∩ B} since \m{x ∈ A} by hypothesis. But this is an impossibility since \m{A ∩ B = ∅}. Therefore, it must be that \m{x ∉ B}. So \m{x ∈ A ∖ B}.
		\stopitemize
	\stopsolution


	\startexercise
		\m{A ∩ (B ∪ C) = (A ∩ B) ∪ (A ∩ C)}.
	\stopexercise
	\startsolution
		\startitemize
			
			\sym{\bold{(⊆)}}  \qquad
				Let \m{x ∈ A ∩ (B ∪ C)} be arbitrary. Then \m{x ∈ A} and \m{x ∈ B ∪ C}. Note that \m{x ∈ B ∪ C} means \m{x ∈ B} or \m{x ∈ C}.
				Now, either \m{x ∈ B} or \m{x ∉ B}, so have two cases.
				\startitemize [1]
					\item  (\m{x ∈ B})  \qquad
						In this case, \m{x ∈ A} and \m{x ∈ B}, so \m{x ∈ A ∩ B}. Therefore \m{x ∈ A ∩ B} or \m{x ∈ A ∩ C}. Hence \m{x ∈ (A ∩ B) ∪ (A ∩ C)}.
					\item  (\m{x ∈ C})  \qquad
						We get the exact same result by interchanging the roles of \m{B} and \m{C} in the previous case.
				\stopitemize

			\sym{\bold{(⊇)}}  \qquad
				Let \m{x ∈ (A ∩ B) ∪ (A ∩ C)} be arbitrary. This means \m{x ∈ A ∩ B} or \m{x ∈ A ∩ C}. So we have two cases:
				\startitemize [1]
					\item  (\m{x ∈ A ∩ B})  \qquad
						In this case, \m{x ∈ A} and \m{x ∈ B}. Now, so \m{x ∈ B} implies \m{x ∈ B} or \m{x ∈ C}, that is, \m{x ∈ B ∪ C}. Therefore \m{x ∈ A ∩ (B ∪ C)}.
					\item  (\m{x ∈ A ∩ C})  \qquad
						Again, we get the exact same result by interchanging the roles of \m{B} and \m{C} in the previous case.
				\stopitemize
		\stopitemize
	\stopsolution
\stopchapter



\startchapter [title={Zeroth-order logic}]

	\startexercise [title={Construction}]
		\startitemize [i]

			\item  Write down a construction sequence for \m{((¬((¬A_1) ∨ A_4)) ∧ ((A_1 → A_3) ↔ A_7))}.

				\startsolution
					\m{⟨A_1,\ A_3,\ A_4,\ A_7,\ (¬A_1),\ ((¬A_1) ∨ A_4),\ (¬((¬A_1) ∨ A_4)),\ (A_1 → A_3),\ ((A_1 → A_3) ↔ A_7),\ ((¬((¬A_1) ∨ A_4)) ∧ ((A_1 → A_3) ↔ A_7))⟩}.
				\stopsolution
				
			\item  Write down a construction tree for \m{(((¬(¬ A_2)) ∧ A_5) → ((A_5 ∨ (¬A_2)) → A_5))}.

				\startsolution
					\starttikzpicture [
						baseline,    % Sets the baseline to the baseline of the text.
						level distance=2em,
						every node/.style={fill=\getvariable{document}{color-headings}!10, circle, inner sep=1pt},
						level 1/.style={sibling distance=8em},
						level 2/.style={sibling distance=4em},
						level 3/.style={sibling distance=2em}]
						
						\node {→} % root
						child { node {∧}
							child { node {¬}
								child { node {¬}
									child { node {\m{A_2}} }
								}
							}
							child { node {\m{A_5}} }
						}
						child { node {→}
							child { node {∨}
								child { node {\m{A_5}} }
								child { node {¬}
									child { node {\m{A_2}} }
								}
							}
							child { node {\m{A_5}} }
						};
					\stoptikzpicture				
				\stopsolution
		\stopitemize
	\stopexercise

	
	\startexercise [title={Enderton, 1.2.1}]
		Show that neither of the following two formulas tautologically implies the other:
		\startformula  \startalign[align={right, left}]
			\NC  α =  \NC  (A ↔ (B ↔ C))  \NR
			\NC  β =  \NC  ((A ∧ (B ∧ C)) ∨ ((¬ A) ∧ ((¬ B) ∧ (¬ C))))  \NR
		\stopalign  \stopformula
	\stopexercise
	\startsolution
		We have to show that \m{α ⊭ β} and \m{β ⊭ α}.

		\bold{(\m{α ⊭ β})}  \qquad
		For this, it suffices to produce a truth assignment \m{v} such that \m{\bar{v}(α) = T} and \m{\bar{v}(β) = F}.

		Consider \m{v} such that \m{v(A) = v(B) = F} and \m{v(C)  = T}. Under \m{\bar{v}}, we get exactly what is required as is shown in the computations below. (Here the truth assignments by \m{\bar{v}} is denoted under each symbol.)
		\startformula  \startalign[n=6, align={left, right, middle, right, middle, left}]
			\NC  α =  \NC  (A  \NC  ↔  \NC  (B  \NC  ↔  \NC  C))  \NR
			\NC  T    \NC   F  \NC  T  \NC   F  \NC  F  \NC  T    \NR
		\stopalign  \stopformula
		\startformula  \startalign[n=15, align={left, right, middle, middle, middle, middle, middle, right, left, middle, right, left, middle, right, left}]
			\NC  β =  \NC  ((A  \NC  ∧  \NC  (B  \NC  ∧  \NC  C))  \NC  ∨  \NC  ((¬  \NC  A)  \NC  ∧  \NC  ((¬  \NC  B)  \NC  ∧  \NC  (¬  \NC  C))))  \NR
			\NC  F    \NC    F  \NC  F  \NC      \NC     \NC       \NC  F  \NC       \NC      \NC  F  \NC       \NC      \NC  F  \NC   F  \NC  T      \NR
		\stopalign  \stopformula

		\bold{(\m{β ⊭ α})}  \qquad
		Again, it suffices to produce \m{v} such that \m{\bar{v}(β) = T} and \m{\bar{v}(α) = F}.

		Consider \m{v} such that \m{v(A)  = v(B) = v(C) = F}. Under \m{\bar{v}}, we get exactly what is required as is shown in the computations below.
		\startformula  \startalign[n=15, align={left, right, middle, middle, middle, middle, middle, right, left, middle, right, left, middle, right, left}]
			\NC  β =  \NC  ((A  \NC  ∧  \NC  (B  \NC  ∧  \NC  C))  \NC  ∨  \NC  ((¬  \NC  A)  \NC  ∧  \NC  ((¬  \NC  B)  \NC  ∧  \NC  (¬  \NC  C))))  \NR
			\NC  T =  \NC       \NC     \NC      \NC     \NC       \NC  T  \NC    T  \NC  F   \NC  T  \NC    T  \NC  F   \NC  T  \NC   T  \NC  F      \NR
		\stopalign  \stopformula
		\startformula  \startalign[n=6, align={left, right, middle, right, middle, left}]
			\NC  α =  \NC  (A  \NC  ↔  \NC  (B  \NC  ↔  \NC  C))  \NR
			\NC  F =  \NC   F  \NC  F  \NC   F  \NC  T  \NC  F    \NR
		\stopalign  \stopformula
	\stopsolution


	\startexercise [title={Enderton, 1.2.4}]
		Show that \m{Σ ∪ \bcrl[α] ⊨ β} iff \m{Σ ⊨ (α → β)}.
	\stopexercise
	\startsolution
		We show each direction separately.

		\bold{(\m{⟹})}  \qquad
		% We show this by contrapositive. Suppose \m{Σ ⊭ (α → β)}. Since
		We suppose \m{Σ ∪ \bcrl[α] ⊨ β}. Let \m{v} be an arbitrary truth assignment that satisfies \m{Σ}. We have to show that \m{v} satisfies \m{(α → β)}. We have two cases.
		\startitemize [i, joinedup]
			\item  \m{\bar{v}(α) = T}: In this case, from the supposition, we get \m{\bar{v}(β) = T}. So \m{\bar{v}(α → β) = T}.
			\item  \m{\bar{v}(α) = F}: In this case, \m{\bar{v}(α → β) = T} since the antecedent is \m{F}.
		\stopitemize
		Since \m{v} was arbitrary, we have \m{Σ ⊨ (α → β)}.

		\bold{(\m{⟸})}  \qquad
		We suppose \m{Σ ⊨ (α → β)}. Let \m{v} be an arbitrary truth assignment that satisfies \m{Σ ∪ \bcrl[α]}. We have to show that \m{v} satisfies \m{β}. Since \m{v} satisfies \m{Σ ∪ \bcrl[α]}, it satisfies \m{Σ}. Therefore, by our supposition, \m{v} satisfies \m{(α → β)}. Now, since \m{v} satisfies \m{α}, it can only be that \m{v} satisfies \m{β}, since the only other way the material implication can be satisfied is when \m{v} does not satisfies \m{α}. This proves our claim.
	\stopsolution


	\startexercise [title={Enderton, 1.2.5}]
		Prove or refute each of the following assertions:
		\startitemize [a, joinedup]

			\item  If either \m{Σ ⊨ α} or \m{Σ ⊨ β}, then \m{Σ ⊨ (α ∨ β)}.

				\startsolution
					(\bold{T})  \qquad
					There are two cases: \m{Σ ⊨ α} and \m{Σ ⊨ β}. Without loss of generality, we can assume that \m{Σ ⊨ α}, as the argument for other case is exactly the same. This means any arbitrary truth assignment \m{v} satisfying \m{Σ} also satisfies \m{α}. This implies \m{\bar{v}(α ∨ β) = T} by the definition of extension of \m{\bar{v}} for \m{∨}.
				\stopsolution

			\item  If \m{Σ ⊨ (α ∨ β)}, then either \m{Σ ⊨ α} or \m{Σ ⊨ β}.

				\startsolution
					(\bold{F})  \qquad
					We give a counterexample. Let \m{α} be a sentence symbol and \m{Σ = ∅}. Then it is always true that \m{⊨ (α ∨ (¬α))}. But it does not follow that \m{⊨ α} or \m{⊨ (¬α)}.

					For an explicit example, consider two truth assignments \m{v_1} and \m{v_2}, such that \m{v_1(α) = T} and \m{v_2(α) = F}. In this case, \m{⊨ α} is not true since \m{v_2} does not satisfy \m{α}, and \m{⊨ (¬α)} is not true since \m{v_1} does not satisfy \m{(¬α)}.
				\stopsolution
		\stopitemize
	\stopexercise


	\startexercise [title={Enderton, 1.2.6}]
		\comment{See Roland's solution for this problem.}

		\startitemize [a, joinedup]

			\item[uniqueness]  Show that if \m{v_1} and \m{v_2} are truth assignments which agree on all the sentence symbols in the wff \m{α}, then \m{\bar{v}_1(α)} = \m{\bar{v}_2(α)}.

				\startsolution
					
					Let \m{G} be the set of sentence symbols used in \m{α}, and let \m{B = \bcrl[ϕ \text{ wff}: \bar{v}_1(ϕ) = \bar{v}_2(ϕ)]}. All we need to show is that \m{α ∈ B}.

					Firstly, \m{G ⊆ B} since \m{v_1} and \m{v_2} agree on the sentence symbols used in \m{α}.

					Secondly, let \m{ϕ, ψ ∈ B} (arbitrary), so \m{v_1} and \m{v_2} agree on \m{ϕ} and \m{ψ}. Let \m{⊡ ∈ \bcrl[∧, ∨, →, ↔]}. Since conditions 1–5 on page 20–21 are the same for \m{\bar{v}_1} and \m{\bar{v}_2}, we have \m{\bar{v}_1(¬ ϕ) = \bar{v}_2(¬ ϕ)} and \m{\bar{v}_1(ϕ ⊡ ψ) = \bar{v}_2(ϕ ⊡ ψ)}. Hence \m{(¬ϕ), (ϕ ⊡ ψ) ∈ B}, that is, \m{B} is closed with respect to the formula building operations.

					Therefore, by the induction principle, \m{B} is the set of \emph{all} wffs generated by the formula building operations. So \m{α ∈ B}, and we are done.
				\stopsolution
			
			\item  Let \m{S} be a set of sentence symbols that includes those in \m{Σ} and \m{τ} (and possibly more). Show that \m{Σ ⊨ τ} iff every truth assignment for \m{S} which satisfies every member of \m{Σ} also satisfies \m{τ}.

				\startsolution
					In this part, we use \m{v} to denote truth assignments and \quotation{\m{v} on a set} means \m{v} is defined on that set. Let \m{G} be the set of sentence symbols used in \m{Σ} and \m{τ}. Clearly, \m{G ⊆ S}.

					We show each direction separately.

					\bold{(\m{⟹})}  \qquad
					From the definition of tautological implication,
					\startformula  \startalign[n=3]
						\NC     \NC  Σ ⊨ τ  \NR
						\NC  ⟺  \NC  (∀ v \text{ on } G) ((v \text{ satisfies } Σ) → (v \text{ satisfies } τ))  \NR
						\NC  ⟹  \NC  (∀ v \text{ on } S) ((v \text{ satisfies } Σ) → (v \text{ satisfies } τ))  \NC  \text{[Part (\in[uniqueness])]}  \NR
					\stopalign  \stopformula
					
					\bold{(\m{⟸})}  \qquad
					Since \m{Σ} and \m{τ} does not depend on any element of \m{S ∖ G}, restricting the definition of \m{v} from \m{S} to \m{G} will not change anything on \m{Σ} and \m{τ}. Therefore,
					\startformula  \startalign
						\NC     \NC  (∀ v \text{ on } S) ((v \text{ satisfies } Σ) → (v \text{ satisfies } τ))  \NR
						\NC  ⟹  \NC  (∀ v \text{ on } G) ((v \text{ satisfies } Σ) → (v \text{ satisfies } τ))  \NR
						\NC  ⟺  \NC  Σ ⊨ τ  \NR
					\stopalign  \stopformula
				  
				\stopsolution
		\stopitemize
	\stopexercise


	\startexercise [title={Enderton, 1.2.8(a) (Substitution)}]
		Let \m{α_1, α_2, …} be a sequence of wffs. For each wff \m{ϕ} and \m{n ∈ ℕ}, let \m{ϕ^∗} be the result of replacing the sentence symbol \m{A_n} in \m{ϕ} by the wff \m{α_n}. Suppose that \m{v} is a truth assignment for the set of all sentence symbols and that \m{u} is a truth assignment defined by \m{u(A_n) = \bar{v}(α_n)}. Show that \m{\bar{u}(ϕ) = \bar{v}(ϕ^∗)}. 

		(Hint: Use the induction principle.)
	\stopexercise
	\startsolution
		We show this via induction on the complexity of any arbitrary wff \m{ϕ}.
		\startitemize [1]
			
			\item  \bold{(Base case)}  \qquad
				Assume \m{ϕ = A_n} for some \m{n ∈ ℕ}, so \m{ϕ^* = α_n}. Now \m{\bar{u}(ϕ) = \bar{u}(A_n) = u(A_n) = \bar{v}(α_n) = \bar{v}(ϕ^∗)}, so the result holds when \m{ϕ} is a sentence symbol.

			\item  \bold{(Induction step)}  \qquad
				We assume that the result holds for all wffs less complex than \m{ϕ} (induction hypothesis). Now, we show that the result holds under all the formula building operations.
				\startitemize
					\sym{(¬)}  \qquad
						Assume \m{ϕ = (¬ψ)} for some wff \m{ψ}, so \m{ϕ^* = (¬ψ^*)}. Then
						\startformula  \startalign[n=4, align={left, right, left, left}]
							\NC     \NC  \bar{u}( ϕ  )  \NC  =  T  \NR
							\NC  ⟺  \NC  \bar{u}(¬ψ  )  \NC  =  T  \NC  \qquad  [\text{Def of \m{ϕ}}]  \NR
							\NC  ⟺  \NC  \bar{u}( ψ  )  \NC  =  F  \NC  \qquad  [\text{Def of \m{\bar{u}} under \m{¬}}]  \NR
							\NC  ⟺  \NC  \bar{v}( ψ^*)  \NC  =  F  \NC  \qquad  [\text{Induction hypothesis}]  \NR
							\NC  ⟺  \NC  \bar{v}(¬ψ^*)  \NC  =  T  \NC  \qquad  [\text{Def of \m{\bar{v}} under \m{¬}}]  \NR
							\NC  ⟺  \NC  \bar{v}( ϕ^*)  \NC  =  T  \NC  \qquad  [\text{Def of \m{ϕ^*}}]  \NR
						\stopalign  \stopformula
					\sym{(∧)}  \qquad
						Assume \m{ϕ = (ψ ∧ θ)} for some wffs \m{ψ, θ}, so \m{ϕ^* = (ψ^* ∧ θ^*)}. Then
						\startformula  \startalign[n=4, align={right, right, left, left}]
							\NC     \NC  \bar{u}(ϕ)  \NC  =  T  \NR
							\NC  ⟺  \NC  \bar{u}(ψ ∧ θ)  \NC  =  T  \NC  \qquad  [\text{Def of \m{ϕ}}]  \NR
							\NC  ⟺  \NC   \bar{u}(ψ) = T  \NC  \text{ and }  \bar{u}(θ) = T  \NC  \qquad  [\text{Def of \m{\bar{u}} under \m{∧}}]  \NR
							\NC  ⟺  \NC   \bar{v}(ψ^*) = T  \NC  \text{ and }  \bar{v}(θ^*) = T  \NC  \qquad  [\text{Induction hypothesis}]  \NR
							\NC  ⟺  \NC  \bar{v}(ψ^* ∧ θ^*)  \NC  =  T  \NC  \qquad  [\text{Def of \m{\bar{v}} under \m{∧}}]  \NR
							\NC  ⟺  \NC   \bar{v}(ϕ^*)  \NC  =  T  \NC  \qquad  [\text{Def of \m{ϕ^*}}]  \NR
						\stopalign  \stopformula
					\sym{(∨)}  \qquad
						Assume \m{ϕ = (ψ ∨ θ)} for some wffs \m{ψ, θ}, so \m{ϕ^* = (ψ^* ∨ θ^*)}. Then
						\startformula  \startalign[n=4, align={right, right, left, left}]
							\NC     \NC  \bar{u}(ϕ)  \NC  =  T  \NR
							\NC  ⟺  \NC  \bar{u}(ψ ∨ θ)  \NC  =  T  \NC  \qquad  [\text{Def of \m{ϕ}}]  \NR
							\NC  ⟺  \NC   \bar{u}(ψ) = T  \NC  \text{ or }  \bar{u}(θ) = T  \NC  \qquad  [\text{Def of \m{\bar{u}} under \m{∨}}]  \NR
							\NC  ⟺  \NC   \bar{v}(ψ^*) = T  \NC  \text{ or }  \bar{v}(θ^*) = T  \NC  \qquad  [\text{Induction hypothesis}]  \NR
							\NC  ⟺  \NC  \bar{v}(ψ^* ∨ θ^*)  \NC  =  T  \NC  \qquad  [\text{Def of \m{\bar{v}} under \m{∨}}]  \NR
							\NC  ⟺  \NC   \bar{v}(ϕ^*)  \NC  =  T  \NC  \qquad  [\text{Def of \m{ϕ^*}}]  \NR
						\stopalign  \stopformula
					\sym{(→)}  \qquad
						Assume \m{ϕ = (ψ → θ)} for some wffs \m{ψ, θ}, so \m{ϕ^* = (ψ^* → θ^*)}. Then
						\startformula  \startalign[n=4, align={right, right, left, left}]
							\NC     \NC  \bar{u}(ϕ)  \NC  =  T  \NR
							\NC  ⟺  \NC  \bar{u}(ψ → θ)  \NC  =  T  \NC  \qquad  [\text{Def of \m{ϕ}}]  \NR
							\NC  ⟺  \NC   \bar{u}(ψ) = F  \NC  \text{ or }  \bar{u}(θ) = T  \NC  \qquad  [\text{Def of \m{\bar{u}} under \m{→}}]  \NR
							\NC  ⟺  \NC   \bar{v}(ψ^*) = F  \NC  \text{ or }  \bar{v}(θ^*) = T  \NC  \qquad  [\text{Induction hypothesis}]  \NR
							\NC  ⟺  \NC  \bar{v}(ψ^* → θ^*)  \NC  =  T  \NC  \qquad  [\text{Def of \m{\bar{v}} under \m{→}}]  \NR
							\NC  ⟺  \NC   \bar{v}(ϕ^*)  \NC  =  T  \NC  \qquad  [\text{Def of \m{ϕ^*}}]  \NR
						\stopalign  \stopformula
					\sym{(↔)}  \qquad
						Assume \m{ϕ = (ψ ↔ θ)} for some wffs \m{ψ, θ}, so \m{ϕ^* = (ψ^* ↔ θ^*)}. Then
						\startformula  \startalign[n=4, align={right, right, left, left}]
							\NC     \NC  \bar{u}(ϕ)  \NC  =  T  \NR
							\NC  ⟺  \NC  \bar{u}(ψ ↔ θ)  \NC  =  T  \NC  \qquad  [\text{Def of \m{ϕ}}]  \NR
							\NC  ⟺  \NC  \bar{u}(ψ)  \NC  =  \bar{u}(θ)  \NC  \qquad  [\text{Def of \m{\bar{u}} under \m{↔}}]  \NR
							\NC  ⟺  \NC  \bar{v}(ψ^*)  \NC  =  \bar{v}(θ^*)  \NC  \qquad  [\text{Induction hypothesis}]  \NR
							\NC  ⟺  \NC  \bar{v}(ψ^* ↔ θ^*)  \NC  =  T  \NC  \qquad  [\text{Def of \m{\bar{v}} under \m{↔}}]  \NR
							\NC  ⟺  \NC  \bar{v}(ϕ^*)  \NC  =  T  \NC  \qquad  [\text{Def of \m{ϕ^*}}]  \NR
						\stopalign  \stopformula
				\stopitemize
		\stopitemize
		Therefore, the induction step holds under all the formula building operations. By the method of induction, \m{\bar{u}(ϕ) = \bar{v}(ϕ)} for every wff \m{ϕ}.
	\stopsolution


	\startexercise [title={Enderton, 1.2.14}]
		\startitemize [i]

			\item  Let \m{S} be the set of all sentence symbols, and assume that \m{v : S → \bcrl[F, T]} is a truth assignment. Show there is at most one extension \m{v} meeting conditions 0–5 on pp. 20–21.

				(Hint: Show that if \m{v_1} and \m{v_2} are such extensions, then \m{\bar{v}_1(α) = \bar{v}_2(α)} for every wff \m{α}. Use the induction principle.)

				\startsolution
					We show this via induction on the complexity of any arbitrary wff \m{α}.
					\startitemize [1]
						
						\item  \bold{(Base case)}  \qquad
							Assume \m{α} is a sentence symbol. Then \m{\bar{v}_1(α) = v(α) = \bar{v}_2(α)} since \m{\bar{v}_1} and \m{\bar{v}_2} are both extensions of \m{v}.

						\item  \bold{(Induction step)}  \qquad
							We assume that the result holds for all wffs less complex than \m{α} (induction hypothesis). Now, we show that the result holds under all the formula building operations.
							\startitemize
								\sym{(¬)}  \qquad
									Assume \m{α = (¬β)} for some wff \m{β}. Then
									\startformula  \startalign[n=4, align={left, right, left, left}]
										\NC     \NC  \bar{v}_1( α)  \NC  =  T  \NR
										\NC  ⟺  \NC  \bar{v}_1(¬β)  \NC  =  T  \NC  \qquad  [\text{Def of \m{α}}]  \NR
										\NC  ⟺  \NC  \bar{v}_1( β)  \NC  =  F  \NC  \qquad  [\text{Def of \m{\bar{v}} under \m{¬}}]  \NR
										\NC  ⟺  \NC  \bar{v}_2( β)  \NC  =  F  \NC  \qquad  [\text{Induction hypothesis}]  \NR
										\NC  ⟺  \NC  \bar{v}_2(¬β)  \NC  =  T  \NC  \qquad  [\text{Def of \m{\bar{v}} under \m{¬}}]  \NR
										\NC  ⟺  \NC  \bar{v}_2( α)  \NC  =  T  \NC  \qquad  [\text{Def of \m{α}}]  \NR
									\stopalign  \stopformula
								\sym{(∧)}  \qquad
									Assume \m{α = (β ∧ γ)} for some wffs \m{β, γ}. Then
									\startformula  \startalign[n=4, align={right, right, left, left}]
										\NC     \NC  \bar{v}_1(α)  \NC  =  T  \NR
										\NC  ⟺  \NC  \bar{v}_1(β ∧ γ)  \NC  =  T  \NC  \qquad  [\text{Def of \m{α}}]  \NR
										\NC  ⟺  \NC   \bar{v}_1(β) = T  \NC  \text{ and }  \bar{v}_1(γ) = T  \NC  \qquad  [\text{Def of \m{\bar{v}} under \m{∧}}]  \NR
										\NC  ⟺  \NC   \bar{v}_2(β) = T  \NC  \text{ and }  \bar{v}_2(γ) = T  \NC  \qquad  [\text{Induction hypothesis}]  \NR
										\NC  ⟺  \NC  \bar{v}_2(β ∧ γ)  \NC  =  T  \NC  \qquad  [\text{Def of \m{\bar{v}} under \m{∧}}]  \NR
										\NC  ⟺  \NC   \bar{v}_2(α)  \NC  =  T  \NC  \qquad  [\text{Def of \m{α}}]  \NR
									\stopalign  \stopformula
								\sym{(∨)}  \qquad
									Assume \m{α = (β ∨ γ)} for some wffs \m{β, γ}. Then
									\startformula  \startalign[n=4, align={right, right, left, left}]
										\NC     \NC  \bar{v}_1(α)  \NC  =  T  \NR
										\NC  ⟺  \NC  \bar{v}_1(β ∨ γ)  \NC  =  T  \NC  \qquad  [\text{Def of \m{α}}]  \NR
										\NC  ⟺  \NC   \bar{v}_1(β) = T  \NC  \text{ or }  \bar{v}_1(γ) = T  \NC  \qquad  [\text{Def of \m{\bar{v}} under \m{∨}}]  \NR
										\NC  ⟺  \NC   \bar{v}_2(β) = T  \NC  \text{ or }  \bar{v}_2(γ) = T  \NC  \qquad  [\text{Induction hypothesis}]  \NR
										\NC  ⟺  \NC  \bar{v}_2(β ∨ γ)  \NC  =  T  \NC  \qquad  [\text{Def of \m{\bar{v}} under \m{∨}}]  \NR
										\NC  ⟺  \NC   \bar{v}_2(α)  \NC  =  T  \NC  \qquad  [\text{Def of \m{α}}]  \NR
									\stopalign  \stopformula
								\sym{(→)}  \qquad
									Assume \m{α = (β → γ)} for some wffs \m{β, γ}. Then
									\startformula  \startalign[n=4, align={right, right, left, left}]
										\NC     \NC  \bar{v}_1(α)  \NC  =  T  \NR
										\NC  ⟺  \NC  \bar{v}_1(β → γ)  \NC  =  T  \NC  \qquad  [\text{Def of \m{α}}]  \NR
										\NC  ⟺  \NC   \bar{v}_1(β) = F  \NC  \text{ or }  \bar{v}_1(γ) = T  \NC  \qquad  [\text{Def of \m{\bar{v}} under \m{→}}]  \NR
										\NC  ⟺  \NC   \bar{v}_2(β) = F  \NC  \text{ or }  \bar{v}_2(γ) = T  \NC  \qquad  [\text{Induction hypothesis}]  \NR
										\NC  ⟺  \NC  \bar{v}_2(β → γ)  \NC  =  T  \NC  \qquad  [\text{Def of \m{\bar{v}} under \m{→}}]  \NR
										\NC  ⟺  \NC   \bar{v}_2(α)  \NC  =  T  \NC  \qquad  [\text{Def of \m{α}}]  \NR
									\stopalign  \stopformula
								\sym{(↔)}  \qquad
									Assume \m{α = (β ↔ γ)} for some wffs \m{β, γ}. Then
									\startformula  \startalign[n=4, align={right, right, left, left}]
										\NC     \NC  \bar{v}_1(α)  \NC  =  T  \NR
										\NC  ⟺  \NC  \bar{v}_1(β ↔ γ)  \NC  =  T  \NC  \qquad  [\text{Def of \m{α}}]  \NR
										\NC  ⟺  \NC  \bar{v}_1(β)  \NC  =  \bar{v}_1(γ)  \NC  \qquad  [\text{Def of \m{\bar{v}} under \m{↔}}]  \NR
										\NC  ⟺  \NC  \bar{v}_2(β)  \NC  =  \bar{v}_2(γ)  \NC  \qquad  [\text{Induction hypothesis}]  \NR
										\NC  ⟺  \NC  \bar{v}_2(β ↔ γ)  \NC  =  T  \NC  \qquad  [\text{Def of \m{\bar{v}} under \m{↔}}]  \NR
										\NC  ⟺  \NC  \bar{v}_2(α)  \NC  =  T  \NC  \qquad  [\text{Def of \m{α}}]  \NR
									\stopalign  \stopformula
							\stopitemize
					\stopitemize
					Therefore, the induction step holds under all the formula building operations. By the method of induction, \m{\bar{v}_1(α) = \bar{v}_2(α)} for every wff \m{α}, which proves the uniqueness of the extension.
				\stopsolution

			\item[satisfiability]  Show that for a set of wffs \m{Σ} and a wff \m{α}: \m{Σ ∪ \bcrl[¬¬α]} is satisfiable ⟺ \m{Σ ∪ \bcrl[α]} is satisfiable.

				\startsolution
					First, note that for any wff \m{α} and truth assignment \m{v},
					\startformula
						\bar{v}(α) = T  \quad ⟺ \quad  \bar{v}(¬α) = F  \quad ⟺ \quad  \bar{v}(¬¬α) = T .
					\stopformula
					Therefore, we have the following (\m{v} always represents a truth assignment):
					\startitemize [joinedup]
						\sym{ }  \qquad  \m{Σ ∪ \bcrl[α]} is satisfiable.
						\sym{⟺}  \qquad  \m{∃v} such that \m{v} satisfies \m{Σ} and \m{\bar{v}(α) = T}.
						\sym{⟺}  \qquad  \m{∃v} such that \m{v} satisfies \m{Σ} and \m{\bar{v}(¬α) = F}.
						\sym{⟺}  \qquad  \m{∃v} such that \m{v} satisfies \m{Σ} and \m{\bar{v}(¬¬α) = T}.
						\sym{⟺}  \qquad  \m{Σ ∪ \bcrl[¬¬α]} is satisfiable.
					\stopitemize
				\stopsolution
		\stopitemize
	\stopexercise


	\startexercise [title={Enderton, 1.7.1}]
		Assume that every finite subset of \m{Σ} is satisfiable. Show that the same is true of at least one of the sets \m{Σ ∪ \bcrl[α]} and \m{Σ ∪ \bcrl[¬α]}.

		(This is part of the proof of the compactness theorem.)

		\emph{Suggestion}: If not, then \m{Σ_1 ∪ \bcrl[α]} and \m{Σ_2 ∪ \bcrl[¬α]} are unsatisfiable for some finite \m{Σ_1 ⊆ Σ} and \m{Σ_2 ⊆ Σ}. Look at \m{Σ_1 ∪ Σ_2}.
	\stopexercise
	\startsolution
		Since \m{Σ_1 ∪ Σ_2 ⊆ Σ} is finite, it must be satisfiable with some truth assignment, say \m{v}. But then \m{v} satisfies \m{Σ_1} since \m{Σ_1 ⊆ Σ_1 ∪ Σ_2}. Therefore the only way to make \m{Σ_1 ∪ \bcrl[α]} unsatisfiable would be to have \m{\bar{v}(α) = F}. By arguing similarly about \m{Σ_2}, we get \m{\bar{v}(¬α) = F}, which gives a contradiction.
	\stopsolution


	\startexercise [title={Enderton, 1.7.2}]
		Let \m{Δ} be a set of wffs such that (i) every finite subset of \m{Δ} is satisfiable, and (ii) for every wff \m{α}, either \m{α ∈ Δ} or \m{(¬ α) ∈ Δ}. Define the truth assignment \m{v}:
		\startformula
			v(A)  =
			\startmathcases
				\NC  T  \MC  \text{if } A ∈ Δ  \NR
				\NC  F  \MC  \text{if } A ∉ Δ  \NR
			\stopmathcases
		\stopformula
		for each sentence symbol \m{A}. Show that for every wff \m{ϕ}, \m{v(ϕ) = T} iff \m{ϕ ∈ Δ}.

		(This is part of the proof of the compactness theorem.)

		\emph{Suggestion}: Use induction on \m{ϕ}.
	\stopexercise
	\startsolution
		This has to be copied from the notes. TODO!
	\stopsolution



	\startexercise [title={Enderton, 1.7.3}]

		Recall the Compactness Theorem: \emph{A set of wffs is satisfiable iff it is finitely satisfiable.}
	
		Recall Corollary 17A: \emph{If \m{Σ ⊨ τ}, then \m{Σ_0 ⊨ τ} for some finite \m{Σ_0 ⊆ Σ}.}
	
		Prove that they are equivalent, i.e., prove that the Compactness Theorem holds iff Corollary 17A holds.

		(Hint: Use the fact that \m{Γ ⊨ σ} iff \m{Γ ∪ \bcrl[¬σ]} is unsatisfiable and 3.3.\in[satisfiability] above.)
	\stopexercise
	\startsolution
		The proof of Corollary 17A in the book shows that the Compactness Theorem implies Corollary 17A. Therefore, we are left to show that Corollary 17A implies the Compactness Theorem.
		
		For this, we assume Corollary 17A and prove Compactness Theorem. Note that if a set of wffs is satisfiable with a truth assignment, then it is finitely satified with the same truth assignment. Therefore, we only have to show that finite satisfiability implies satisfiability.

		Suppose not. That is, assume that \m{Σ} is a set of wffs such that \m{Σ} is finitely satisfiable but \m{Σ} is unsatisfiable. Fix a wff \m{τ}. Since \m{Σ} is unsatisfiable, it is vacously true that \m{Σ ⊨ τ} and \m{Σ ⊨ ¬τ} (as in page 23 of Enderton). Since \m{Σ ⊨ τ}, using Corollary 17A, there is a finite subset \m{Σ_1 ⊆ Σ} such that \m{Σ_1 ⊨ τ}. Similarly, there exists \m{Σ_2 ⊆ Σ} finite such that \m{Σ_2 ⊨ ¬τ}. Now, since \m{Σ_1 ∪ Σ_2 ⊆ Σ} is finite, it is satisfiable by a truth assignment, say \m{v}. Clearly, since \m{Σ_1, Σ_2 ⊆ Σ}, the assignment \m{v} satisfies both \m{Σ_1} and \m{Σ_2}. Then \m{v} satisfies both \m{τ} and \m{¬τ}, which is an impossibility. This contradiction shows that if \m{Σ} is finitely satisfiable then \m{Σ} is satisfiable. This concludes the proof.
	\stopsolution

\stopchapter




\startchapter [title={First-order logic}]
	
	\emph{Note}: \m{∄} abbreviates \m{¬∃}, and \m{∉} abbreviates \m{¬∈}. We shall also use the convention that grouping for conditionals is from the right. That is, \m{(p → q → r) = ((p → (q → r)))}.

	\startexercise [title={Enderton, 2.1.1}]
		Assume that we have a language with the following parameters: \m{∀}, intended to mean \quotation{for all things}; \m{N}, intended to mean \quotation{is a number}; \m{I}, intended to mean \quotation{is interesting}; \m{<}, intended to mean \quotation{is less than}; and \m{0}, a constant symbol intended to denote zero. Translate into this language the English sentences listed below. If the English sentence is ambiguous, you will need more than one translation.
		\startitemize[a]
		
			\item  Zero is less than any number.
				\startsolution
					\m{∀x\,(Nx → <0x)}
				\stopsolution

			\item  If any number is interesting, then zero is interesting.
				\startsolution
					The word \emph{any number} can be interpretated as \emph{every number} or \emph{some number}. Using the exportation tautology, the corresponding translations are
					\startitemize [joinedup]
						\sym{(every)}  \qquad \qquad
							\m{∀x\,(Nx → Ix → I0)}
						\sym{(some)}  \qquad \qquad
							\m{∃x (Nx → Ix) → I0}
					\stopitemize
				\stopsolution

			\item  No number is less than zero.
				\startsolution
					\startformula  \startalign
						\NC     \NC  ∄x (Nx ∧ <x0)  \NR
						\NC  ⟺  \NC  ¬∃x (Nx ∧ <x0)  \NR
						\NC  ⟺  \NC  ¬¬∀x\,(¬(Nx ∧ <x0))  \NR
						\NC  ⟺  \NC  ¬¬∀x\,(¬¬(Nx → ¬<x0))  \NR
						\NC  ⟺  \NC  ∀x\,(Nx → ¬<x0))  \NR
					\stopalign  \stopformula
				\stopsolution

			\item  Any uninteresting number with the property that all smaller numbers are interesting certainly is interesting.
				\startsolution
					\m{∀x\,(Nx → ¬Ix → ∀y\,(Ny → <yx → Iy) → Ix)}
				\stopsolution

			\item  There is no number such that all numbers are less than it.
				\startsolution
					\startformula  \startalign
						\NC     \NC  ∄x (Nx ∧ ∀y\,(Ny → <xy))  \NR
						\NC  ⟺   \NC  ¬∃x (¬(Nx → ¬∀y\,(Ny → <xy)))  \NR
						\NC  ⟺  \NC  ¬¬∀x\,¬(¬(Nx → ¬∀y\,(Ny → <xy)))  \NR
						\NC  ⟺  \NC  ∀x (Nx → ¬∀y\,(Ny → <xy))  \NR
					\stopalign  \stopformula
				\stopsolution

			\item  There is no number such that no number is less than it.
				\startsolution
					\startformula  \startalign
						\NC     \NC  ∄x (Nx ∧ ∄y (Ny → (y < x)))  \NR
						\NC  ⟺  \NC  ¬∃x (Nx ∧ ¬∃y (Ny → <xy))  \NR
						\NC  ⟺  \NC  ¬¬∀x\,¬(¬(Nx → ¬¬∃y (Ny → <xy)))  \NR
						\NC  ⟺  \NC  ∀x (Nx → ∃y (Ny → <xy))  \NR
						\NC  ⟺  \NC  ∀x (Nx → ¬∀y\,¬(Ny → <xy))  \NR
					\stopalign  \stopformula
				\stopsolution
		\stopitemize
	\stopexercise

	
	\startexercise [title={Enderton, 2.1.3}]
		Translate the English sentence into the first-order language specified by \m{∀}, for all sets; \m{∈}, is a member of; \m{a}, \m{a}; \m{b}, \m{b}.

		\quotation{Neither \m{a} nor \m{b} is a member of every set.}
	\stopexercise
	\startsolution
		\startformula  \startalign
			\NC     \NC  \text{Neither \m{a} nor \m{b} is a member of every set.}  \NR
			\NC  ⟺  \NC  \text{\m{a} is not a member of every set and \m{b} is not a member of every set.}  \NR
			\NC  ⟺  \NC  \text{There is a set that \m{a} is not a member of and there is a set that \m{b} is not a member of.}  \NR
			\NC  ⟺  \NC  (∃x (a ∉ x)) ∧ (∃y (b ∉ y))  \NR
			\NC  ⟺  \NC  (∃x (¬∈ax)) ∧ (∃y (¬∈bx))  \NR
			\NC  ⟺  \NC  (¬∀x\,(¬¬∈ax)) ∧ (¬∀y\,(¬¬∈by))  \NR
			\NC  ⟺  \NC  (¬∀x\,∈ax) ∧ (¬∀y\,∈by)  \NR
		\stopalign  \stopformula
	\stopsolution


	\startexercise [title={Enderton, page 87}]
		Prove that \m{\logicimpl[𝔄][s]{α ∨ β}} iff \m{\logicimpl[𝔄][s]{α}} or \m{\logicimpl[𝔄][s]{β}}.
	\stopexercise
	\startsolution
		\startformula  \startalign[n=3]
			\NC     \NC  \logicimpl[𝔄][s]{(α ∨ β)}  \NR
			\NC  ⟺  \NC  \logicimpl[𝔄][s]{(¬α →β)}  \NC  \qquad  [\text{Expansion of \m{∨}}]  \NR
			\NC  ⟺  \NC  \nlogicimpl[𝔄][s]{¬α}  \text{ or }  \logicimpl[𝔄][s]{β} \NC  \qquad  [\text{Definition of \m{s} for \m{→}}]  \NR
			\NC  ⟺  \NC  \logicimpl[𝔄][s]{α} \text{ or }  \logicimpl[𝔄][s]{β} \NC  \qquad  [\text{Definition of \m{s} for \m{¬}}]  \NR
		\stopalign  \stopformula
	\stopsolution

	
	\startexercise [title={Enderton, 2.2.1}]
		Show that
		\startitemize [a]
			
			\item  \m{Γ ∪ \bcrl[α] ⊨ ϕ} iff \m{Γ ⊨ (α → ϕ)}.
				\startsolution
					\startitemize

						\sym{\bold{(⟹)}}  \qquad
							\m{Γ ∪ \bcrl[α] ⊨ ϕ} means that any structure \m{𝔄} and satisfaction function \m{s: V → \abs[𝔄]} such that \m{𝔄} satisfies every member of \m{Γ} and \m{α} with \m{s}, the structure \m{𝔄} also satisfies \m{ϕ} with \m{s}. Now, let \m{𝔄} and \m{s} be such a combination. Now, either \m{\nlogicimpl[𝔄][s]{α}} or \m{\logicimpl[𝔄][s]{α}}. If \m{\nlogicimpl[𝔄][s]{α}}, then \m{\logicimpl[𝔄][s]{α → ϕ}} by definition. If \m{\logicimpl[𝔄][s]{α}}, then \m{Γ ∪ \bcrl[α] ⊨ ϕ} and \m{\logicimpl[𝔄][s]{α}} implies \m{\logicimpl[𝔄][s]{(α → ϕ)}}. Therefore, \m{Γ ⊨ (α → ϕ)}.

						\sym{\bold{(⟸)}}  \qquad
							Since we have \m{Γ ⊨ (α → ϕ)}, any structure \m{𝔄} and satisfaction function \m{s: V → \abs[𝔄]} such that \m{𝔄} satisfies every member of \m{Γ} and \m{α} with \m{s}, since we have \m{\logicimpl[𝔄][s]{α}} and \m{Γ ⊨ (α → ϕ)}, it must be that \m{\logicimpl[𝔄][s]{ϕ}}.
					\stopitemize
				\stopsolution

			\item  \m{ϕ ⟚ ψ} iff \m{⊨ (ϕ ↔ ψ)}.
				\startsolution
					% For any structure \m{𝔄} and satisfaction function \m{s: V → \abs[𝔄]}, either \m{\nlogicimpl[𝔄][s]{ϕ}} or \m{\logicimpl[𝔄][s]{ϕ}}.

					% If \m{\nlogicimpl[𝔄][s]{ϕ}}, we have \m{\nlogicimpl[𝔄][s]{ψ}}, so \m{\logicimpl[𝔄][s]{ϕ ↔ ψ}}.

					% If \m{\logicimpl[𝔄][s]{ϕ}}, we have \m{\logicimpl[𝔄][s]{ψ}}, so \m{\logicimpl[𝔄][s]{ϕ ↔ ψ}}.
					\startformula  \startalign[n=3]
						\NC     \NC  ϕ ⟚ ψ  \NR
						\NC  ⟺  \NC  ⊨ (ϕ → ψ) \text{ and }  ⊨ (ψ → ϕ)  \NC  [Using (a)]  \NR
						\NC  ⟺  \NC  \text{for every } (𝔄, s),  \logicimpl[𝔄][s]{(ϕ → ψ)}  \text{ and }  \logicimpl[𝔄][s]{(ψ → ϕ)}  \NR
						\NC  ⟺  \NC  \text{for every } (𝔄, s),  \logicimpl[𝔄][s]{(ϕ → ψ) ∧ (ψ → ϕ)}  \NC  [\text{Definition of } ∧]  \NR
						\NC  ⟺  \NC  \text{for every } (𝔄, s),  \logicimpl[𝔄][s]{(ϕ ↔ ψ)}  \NC  [\text{Definition of } ↔]  \NR
						\NC  ⟺  \NC  ⊨ (ϕ ↔ ψ)  \NR
					\stopalign  \stopformula
				\stopsolution
		\stopitemize
	\stopexercise

	
	\startexercise [title={Enderton, 2.2.3}]
		Show that \m{\bcrl[∀x\,α, ∀x\,(α → β)] ⊨ ∀x\,β}.
	\stopexercise
	\startsolution
		Let \m{𝔄} be an arbitrary structure and \m{s: V → \abs[𝔄]} be an arbitrary satisfaction function such that \m{𝔄} satisfies \m{∀x\,α} and \m{∀x\,(α → β)} with \m{s}.

		Now, by the definition of the satisfaction function,
		\placeformula  \startformula  \startalign
			\NC  \logicimpl[𝔄][s]{∀x\,α}        ⟺  \NC  \text{for every } a ∈ \abs[𝔄], \logicimpl[𝔄][s(x∣a)]{α} , \qquad \text{ and }  \NR[eq:condition1]
			\NC  \logicimpl[𝔄][s]{∀x\,(α → β)}  ⟺  \NC  \text{for every } a ∈ \abs[𝔄], \logicimpl[𝔄][s(x∣a)]{α → β}  \NR
			\NC  ⟺  \NC  \text{for every } a ∈ \abs[𝔄], \nlogicimpl[𝔄][s(x∣a)]{α} \text{ or } \logicimpl[𝔄][s(x∣a)]{β} .  \NR[eq:condition2]
		\stopalign  \stopformula

		Fix an arbitrary \m{a ∈ \abs[𝔄]}. If \m{\nlogicimpl[𝔄][s(x∣a)]{α}} in (\in[eq:condition2]), then combining it with (\in[eq:condition1]) gives a contradiction. Therefore, we must have \m{ \logicimpl[𝔄][s(x∣a)]{β}}. This means for every \m{a ∈ \abs[𝔄]}, we have \m{ \logicimpl[𝔄][s(x∣a)]{β}}, which says \m{\logicimpl[𝔄][s]{∀x\,β}}.
	\stopsolution


	\startexercise [title={Enderton, 2.2.4}]
		Show that if \m{x} does not occur free in \m{α}, then \m{α ⊨ ∀x\,α}.
	\stopexercise
	\startsolution
		Let \m{𝔄} be an arbitrary structure and \m{s: V → \abs[𝔄]} be an arbitrary satisfaction function such that \m{𝔄} satisfies \m{α} with \m{s}. Now, fix an arbitrary \m{a ∈ \abs[𝔄]}. Then the satisfaction functions \m{s} and \m{s(x∣a)} agree on all variables except possibly at \m{x}. But \m{x} does not occur free in \m{α}, so \m{s} and \m{s(x∣a)} agree on all variables that occur free in \m{α}. Now by Theorem 22A, \m{\logicimpl[𝔄][s(x∣a)]{α}}. Since \m{a} was arbitrary, this holds for every \m{a ∈ \abs[𝔄]}. Therefore by the definition of satisfaction function, \m{\logicimpl[𝔄][s]{∀x\,α}}. Therefore \m{α ⊨ ∀x\,α}.
	\stopsolution


	\startexercise [title={Enderton, 2.2.6}]
		Show that a formula \m{θ} is valid iff \m{∀x\,θ} is valid.
	\stopexercise
	\startsolution
		\startformula  \startalign
			\NC     \NC  θ \text{ is valid }  \NR
			\NC  ⟺  \NC  \text{for every } 𝔄 \text{ and } s: V → \abs[𝔄], \text{ we have } \logicimpl[𝔄][s]{θ}  \NR
			\NC  ⟺  \NC  \text{for every } 𝔄 \text{ and } s: V → \abs[𝔄] \text{ and } a ∈ \abs[𝔄], \text{ we have } \logicimpl[𝔄][s(x∣a)]{θ}  \NR
			\NC  ⟺  \NC  \text{for every } 𝔄 \text{ and } s: V → \abs[𝔄], \text{ we have } \logicimpl[𝔄][s]{∀x\,θ}  \NR
			\NC  ⟺  \NC  ∀x\,θ \text{ is valid }  \NR
		\stopalign  \stopformula
		The second \m{⟺} above works because
		\startitemize [joinedup]

			\sym{\bold{(⟹)}}  \qquad
				\m{s(x∣a): V → \abs[𝔄]} is also a satisfaction function, and
		
			\sym{\bold{(⟸)}}  \qquad
				\m{s = s(x∣s(x))}.
		\stopitemize
		
	\stopsolution


	\startexercise [title={Enderton, 2.2.8}]
		Assume that \m{Σ} is a set of sentences such that for any sentence \m{τ}, either \m{Σ ⊨ τ} or \m{Σ ⊨ ¬τ}. Assume that \m{𝔄} is a model of \m{Σ}. Show that for any sentence \m{τ}, we have \m{⊨_𝔄 τ} iff \m{Σ ⊨ τ}.
	\stopexercise
	\startsolution
		\startitemize

			\sym{\bold{(⟸)}}  \qquad
				Since \m{Σ ⊨ τ}, and \m{𝔄} is a model of \m{Σ}, it follows from definition of logical implication that \m{⊨_𝔄 τ}.

			\sym{\bold{(⟹)}}  \qquad
				We prove this side by proving the contrapositive. If \m{Σ ⊭ τ}, then \m{Σ ⊨ ¬τ} by the given condition. Since \m{𝔄} is a model of \m{Σ}, \m{⊨_𝔄 ¬τ}, and therefore \m{⊭_𝔄 τ}.

		\stopitemize
	\stopsolution


	\startexercise [title={Enderton, 2.2.13}]
		Prove part (a) of the homomorphism theorem.
	\stopexercise
	\startsolution
		We prove this by induction of the complexity of a term \m{t}.

		\startitemize [i, joinedup]

			\item  (base case: \m{t} is a variable)  In this case,
				\startformula  \startalign[n=3]
					\NC  (h ∘ \overline{s})(t)  =  \NC  h(\overline{s}(t))  \NR
					\NC  =  \NC  h(s(t))  \NC  [\text{Definition of \m{\overline{s}}}]  \NR
					\NC  =  \NC  (h ∘ s)(t)  \NR
					\NC  =  \NC  (\overline{h ∘ s})(t) .  \NC  \qquad  [\text{Definition of \m{\overline{h ∘ s}}}]  \NR
				\stopalign  \stopformula

			\item  (base case: \m{t} is a constant)  In this case,
				\startformula  \startalign[n=3]
					\NC  (h ∘ \overline{s})(t)  =  \NC  h(\overline{s}(t))  \NR
					\NC  =  \NC  h(t^𝔄)  \NC  [\text{Definition of \m{\overline{s}}}]  \NR
					\NC  =  \NC  t^𝔅  \NC  [\text{Definition of homomorphism}]  \NR
					\NC  =  \NC  (\overline{h ∘ s})(t) .  \NC  \qquad  [\text{Definition of \m{\overline{h ∘ s}}}]  \NR
				\stopalign  \stopformula

			\item  (induction step: \m{t = f t_1 ⋯ t_n})  In this case, we use the induction hypothesis that the result holds for all terms less complex than \m{t}. Now,
				\startformula  \startalign[n=3]
					\NC  (h ∘ \overline{s})(t)  =  \NC  h(\overline{s}(f t_1 ⋯ t_n))  \NR
					\NC  =  \NC  h(f^𝔄(\overline{s}(t_1), …, \overline{s}(t_n)))  \NC  [\text{Definition of \m{\overline{s}}}]  \NR
					\NC  =  \NC  h(f^𝔅(h(\overline{s}(t_1)), …, h(\overline{s}(t_n))))  \NC  [\text{Definition of homomorphism}]  \NR
					\NC  =  \NC  h(f^𝔅(\overline{h ∘ s})(t_1), …, (\overline{h ∘ s})(t_n)))  \NC  \qquad  [\text{Induction hypothesis}]  \NR
					\NC  =  \NC  (\overline{h ∘ s})(f t_1 ⋯ t_n)  \NC  [\text{Definition of \m{\overline{h ∘ s}}}]  \NR
					\NC  =  \NC  (\overline{h ∘ s})(t) . \NC  [\text{Definition of \m{\overline{h ∘ s}}}]  \NR					
				\stopalign  \stopformula
		\stopitemize

		Therefore, \m{\overline{h ∘ s} ≡ h ∘ \overline{s}} on \m{T}.
	\stopsolution


	\startexercise [title={Enderton, 2.2.18}]
		A universal (\m{∀_1}) formula is one of the form \m{∀ x_1 ⋯ ∀ x_n \, θ}, where \m{θ} is quantifier-free. An existential (\m{∃_1}) formula is of the dual form \m{∃ x_1 ⋯ ∃ x_n θ}. Let \m{𝔄} be a substructure of \m{𝔅}, and let \m{s : V → \abs[𝔄]}.
		\startitemize [a]

			\item  Show that if \m{\logicimpl[𝔄][s]{ψ}} and \m{ψ} is existential, then \m{\logicimpl[𝔅][s]{ψ}}. And if \m{\logicimpl[𝔅][s]{ϕ}} and \m{ϕ} is universal, then \m{\logicimpl[𝔄][s]{ϕ}}.
				\startsolution
					Firstly, note that since \m{𝔄} be a substructure of \m{𝔅}, the identity map \m{i: \abs[𝔄] → \abs[𝔅]} is a isomorphism. Therefore, parts (a), (b), and (c) of the Homomorphism Theorem are true.
					\startitemize [i]
						
						\item  \bold{If \m{\logicimpl[𝔅][s]{ϕ}} and \m{ϕ} is universal, then \m{\logicimpl[𝔄][s]{ϕ}}.}

							Since \m{ϕ} is universal, \m{ϕ = ∀ x_1 ⋯ ∀ x_n \, θ} for some quantifier-free wff \m{θ}. Therefore, \m{\logicimpl[𝔅][s]{∀ x_1 ⋯ ∀ x_n \, θ}} means that for every \m{b_1, …, b_n ∈ \abs[𝔅]}, we have \m{\logicimpl[𝔅][s {\brnd[{\tuple[x_1, …, x_n]} ∣ {\tuple[b_1, …, b_n]}]}] {θ}}. Therefore, in particular, for every \m{a_1, …, a_n ∈ \abs[𝔄] ⊆ \abs[𝔅]}, we have \m{\logicimpl[𝔅][s {\brnd[{\tuple[x_1, …, x_n]} ∣ {\tuple[a_1, …, a_n]}]}] {θ}}. By the Homomorphism Theorem, for every \m{a_1, …, a_n ∈ \abs[𝔄]}, we have \m{\logicimpl[𝔄][s {\brnd[{\tuple[x_1, …, x_n]} ∣ {\tuple[a_1, …, a_n]}]}] {θ}}, that is, \m{\logicimpl[𝔄][s]{∀ x_1 ⋯ ∀ x_n \, θ}}.

						\item  \bold{If \m{\logicimpl[𝔄][s]{ψ}} and \m{ψ} is existential, then \m{\logicimpl[𝔅][s]{ψ}}.}

							Since \m{ψ} is existential, \m{ψ = ∃ x_1 ⋯ ∃ x_n θ = ¬∀x_1¬ ¬∀x_2¬ ⋯ ¬∀x_2¬ θ}, which is logically equivalent to saying \m{ψ = ¬∀x_1 ∀x_2 ⋯ ∀x_n ¬θ}. Therefore, \m{\logicimpl[𝔄][s]{ψ}} is the same as \m{\logicimpl[𝔄][s]{¬ ∀x_1 ∀x_2 ⋯ ∀x_n ¬θ}}, which is in turn the same as \m{\nlogicimpl[𝔄][s]{∀x_1 ∀x_2 ⋯ ∀x_n ¬θ}}. Now, using the above result, we get that this implies \m{\nlogicimpl[𝔅][s]{∀x_1 ∀x_2 ⋯ ∀x_n ¬θ}}. Reversing the above steps, we get \m{\logicimpl[𝔅][s]{ψ}}.

					\stopitemize
					
				\stopsolution

			\item  Conclude that the sentence \m{∃x\,Px} is not logically equivalent to any universal sentence, nor \m{∀x\,Px} to any existential sentence.
				\startsolution
					% Let \m{𝔄 = (\bcrl[0, 1], \bcrl[0], \bcrl[1])}
				\stopsolution
		\stopitemize
	\stopexercise


	\startexercise [title={Enderton, 2.2.20(a)}]
		Assume the language has equality and a two-place predicate symbol \m{P}. Consider the two structures \m{(ℕ; <)} and \m{(ℝ; <)} for the language. Find a sentence true in one structure and false in the other.
	\stopexercise
	\startsolution
		\startitemize [i, joinedup]
			\item  True in \m{(ℕ; <)} but false in \m{(ℝ; <)}: There is an smallest object, i.e.,
				\startformula
					∃x\,∀y\,((y ≠ x) → (x < y)) .
				\stopformula
			\item  True in \m{(ℝ; <)} but false in \m{(ℕ; <)}: There is an object between any two objects, i.e.,
				\startformula
					∀x\,∀y\,∃z\,((z < y ∧ z > x) ∨ (z < x ∧ z > y)) .
				\stopformula
		\stopitemize
	\stopsolution

	
	\startexercise [title={Enderton, 2.4.2}]
		To which axiom groups, if any, do each of the following formulas belong?
		\startitemize [a]

			\item  \m{[(∀x\,Px → ∀y\,Py) → Pz] → [∀x\,Px →(∀y\,Py → Pz)]}
				\startsolution
					Group 1.

					This is of the form \m{((p → q) → r) → (p → (q → r))}, which is a tautology. If not, there would be a truth assignment \m{v} of \m{p, q, r} such that \m{\bar{v}(((p → q) → r) → (p → (q → r))) = F}. Now, this is only possible iff \m{\bar{v}((p → q) → r) = T} and \m{\bar{v}(p → (q → r)) = F}. Looking at the consequent, \m{\bar{v}(p → (q → r)) = F} iff \m{v(p) = T} and \m{\bar{v}(q → r) = F}. Furthermore, \m{\bar{v}(q → r) = F} iff \m{v(q) = T} and \m{v(r) = F}. Now, using these values, \m{\bar{v}((p → q) → r) = F}, which gives a contradiction.
				\stopsolution

			\item  \m{∀y\, [∀x\,(Px → Px) → (Pc → Pc)]}
				\startsolution
					Generalization of group 2.
				\stopsolution
			
			\item  \m{∀x\, ∃y\, Pxy → ∃y\, Pyy}
				\startsolution
					None (\m{x} is not substitutable by \m{y} in \m{∃y\, Pxy}).

					In fact, this is false. A counterexample is given by \m{𝔄 = \tuple[ℕ, <]} as every number is smaller than some number but no number is smaller than itself.					
				\stopsolution
		\stopitemize
	\stopexercise


	\startexercise [title={Enderton, 2.4.6(a)}]
		 Show that if \m{⊢ α → β}, then \m{⊢ ∀x \, α → ∀x \, β}.
	\stopexercise
	\startsolution
		If \m{⊢ α → β}, using the Generalization Theorem, we have \m{⊢ ∀x \, (α → β)}. Now using axiom group 3, we have \m{⊢ ∀x \, (α → β) → (∀x \, α → ∀x \, β)}. Using these and modus ponens, we get \m{⊢ ∀x \, α → ∀x \, β}.
	\stopsolution


	\startexercise [title={Enderton, 2.4.3}]
		\startitemize [a]

			\item[ex2.4.3a]   Let \m{𝔄} be a structure and let \m{s: V → \abs[𝔄]}. Define a truth assignment \m{v} on the set of prime formulas by
				\startformula
					v(α) = T  \qquad  \text{iff}  \qquad  \logicimpl[𝔄][s]{α} . 
				\stopformula
				Show that for any formula (prime or not),
				\startformula
					\bar{v}(α) = T  \qquad  \text{iff}  \qquad  \logicimpl[𝔄][s]{α} . 
				\stopformula
				\startsolution
					We show this by induction on the complexity of \m{α}, with the induction hypothesis that the result holds for all wffs less complex than \m{α}.
					\startitemize [i]
						
						\item  \bold{Base case: \m{α} is prime.}  \qquad
							In this case, \m{\bar{v}(α) = v(α) = T} iff \m{\logicimpl[𝔄][s]{α}} by the given condition.

						\item  \bold{Induction step: \m{α} is \m{¬β} for some \m{β}.}  \qquad
							\startformula  \startalign[n=3]
								\NC  \bar{v}(α) = T  ⟺  \NC  \bar{v}(¬β) = T  \NR
								\NC  ⟺  \NC  \bar{v}(β) = F  \NC  [\text{Definition of } \bar{v}]  \NR
								\NC  ⟺  \NC  \nlogicimpl[𝔄][s]{β}  \qquad  \NC  [\text{Induction hypothesis}]  \NR
								\NC  ⟺  \NC  \logicimpl[𝔄][s]{¬β}  \NC  [\text{Definition of satisfaction}]  \NR
								\NC  ⟺  \NC  \logicimpl[𝔄][s]{α}  \NR
							\stopalign  \stopformula

						\item  \bold{Induction step: \m{α} is \m{β → γ} for some \m{β} and \m{γ}.}  \qquad
							\startformula  \startalign[n=3]
								\NC  \bar{v}(α) = T  ⟺  \NC  \bar{v}(β → γ) = T  \NR
								\NC  ⟺  \NC  \bar{v}(β) = F  \text{ or }  \bar{v}(γ) = T  \NC  [\text{Definition of } \bar{v}]  \NR
								\NC  ⟺  \NC  \nlogicimpl[𝔄][s]{β}  \text{ or }  \logicimpl[𝔄][s]{γ}  \qquad  \NC  [\text{Induction hypothesis}]  \NR
								\NC  ⟺  \NC  \logicimpl[𝔄][s]{β → γ}  \NC  [\text{Definition of satisfaction}]  \NR
								\NC  ⟺  \NC  \logicimpl[𝔄][s]{α}  \NR
							\stopalign  \stopformula
					\stopitemize
				\stopsolution

			\item  Conclude that if \m{Γ} tautologically implies \m{ϕ}, then \m{Γ} logically implies \m{ϕ}.
				\startsolution
					\emph{Note}: Satisfying a set means to satisfy each element of that set.
					
					Suppose \m{Γ} tautologically implies \m{ϕ}.

					Let \m{𝔄} by an arbitrary structure and \m{s: V → \abs[𝔄]} by an arbitrary satisfaction function. Corresponding to \m{\tuple[𝔄, s]}, we have a truth assignment, say \m{v}. Now,

					% Using Part (\in[ex2.4.3a]), \m{𝔄} satisfies \m{Γ} with \m{s} iff \m{v} satisfies \m{Γ}. This implies \m{v} satisfies \m{ϕ} since \m{Γ} tautologically implies \m{ϕ}. But \m{v} satisfies \m{ϕ} iff \m{𝔄} satisfies \m{ϕ} with \m{s}, which is what we wanted.
					\startformula  \startalign[n=3]
						\NC  𝔄 \text{ satisfies } Γ \text{ with } s  ⟺  \NC  v \text{ satisfies } Γ  \NC  [\text{Part (\in[ex2.4.3a])}]  \NR
						\NC  ⟹  \NC  v \text{ satisfies } ϕ  \NC  [Γ \text{ tautologically implies } ϕ]  \NR
						\NC  ⟺  \NC  𝔄 \text{ satisfies } ϕ \text{ with } s  \qquad  \NC  [\text{Part (\in[ex2.4.3a])}]  \NR
					\stopalign  \stopformula
					Since this holds for any arbitrary structure and satisfaction function, \m{Γ} logically implies \m{ϕ}.
				\stopsolution
		\stopitemize
	\stopexercise


	\startexercise [title={Enderton, 2.4.9(b)}]
		Show that if \m{y} does not occur at all in \m{ϕ}, then \m{x} is substitutable for \m{y} in \m{ϕ^x_y} and \m{(ϕ^x_y)^y_x = ϕ}.
	\stopexercise
	\startsolution
		We use induction on the complexity of \m{ϕ}, with the induction hypothesis that the result holds for all wffs less complex than \m{ϕ}.
		\startitemize [i]

			\item  \bold{Atomics: \m{ϕ} is atomic.}  \qquad
				Since \m{ϕ} is atomic, so is \m{ϕ^x_y}, and therefore \m{x} is substitutable for \m{y} in \m{ϕ^x_y}.

				To show that \m{(ϕ^x_y)^y_x = ϕ}, we consider each case separately.
				\startitemize [4, joinedup]
					\item  \emph{Constants: \m{ϕ} is a constant symbol.}  \qquad				
						Since \m{ϕ^x_y = ϕ}, so \m{(ϕ^x_y)^y_x = ϕ}, as desired.

					\item  \emph{Variables: \m{ϕ} is a variable (that is not \m{y}).}  \qquad				
						If \m{ϕ = x}, then \m{ϕ^x_y = y}, so \m{(ϕ^x_y)^y_x = x = ϕ}, and if \m{ϕ ≠ x}, then \m{ϕ^x_y = ϕ}, so \m{(ϕ^x_y)^y_x = ϕ} since \m{y} does not occur at all in \m{ϕ}.

					\item  \emph{Terms: \m{ϕ} is \m{f t_1 ⋯ t_n} for some function symbol \m{f} and terms \m{t_1, …, t_n}.}  \qquad
						Since \m{ϕ^x_y = f (t_1)^x_y ⋯ (t_n)^x_y}, so by the induction hypothesis, \m{(ϕ^x_y)^y_x = f ((t_1)^x_y)^y_x ⋯ ((t_n)^x_y)^y_x = f t_1 ⋯ t_n = ϕ}.
					
					\item  \emph{Predicates: \m{P t_1 ⋯ t_n} for some predicate symbol \m{P} and terms \m{t_1, …, t_n}.}
						Since \m{ϕ^x_y = P (t_1)^x_y ⋯ (t_n)^x_y}, so by the induction hypothesis, \m{(ϕ^x_y)^y_x = P ((t_1)^x_y)^y_x ⋯ ((t_n)^x_y)^y_x = P t_1 ⋯ t_n = ϕ}. Note that this includes the case for equality.
				\stopitemize
				
			\item  \bold{Negation: \m{ϕ} is \m{(¬ψ)} for some \m{ψ}.}  \qquad
				We know that \m{x} is substitutable for \m{y} in \m{ϕ^x_y} iff \m{x} is substitutable for \m{y} in \m{ψ^x_y}, which it is by the induction principle.

				Moreover, \m{ϕ^x_y = (¬ ψ^x_y)}, so using the induction hypothesis, \m{(ϕ^x_y)^y_x = (¬ ψ^x_y)^y_x = (¬ (ψ^x_y)^y_x) = (¬ ψ) = ϕ}.

			\item  \bold{Implication: \m{ϕ} is \m{(ψ → θ)} for some \m{ψ} and \m{θ}.}  \qquad
				We know that \m{x} is substitutable for \m{y} in \m{ϕ^x_y} iff \m{x} is substitutable for \m{y} in both \m{ψ^x_y} and \m{θ^x_y}, which they are by the induction principle.

				Moreover, \m{ϕ^x_y = (ψ → θ)^x_y = (ψ^x_y → θ^x_y)}, so using the induction hypothesis, \m{(ϕ^x_y)^y_x = (ψ^x_y → θ^x_y)^y_x = ((ψ^x_y)^y_x → (θ^x_y)^y_x) = (ψ → θ) = ϕ}.

			\item  \bold{Quantifiers: \m{ϕ} is \m{∀z \, ψ} for some \m{ψ}.}  \qquad
				We have two cases.
				\startitemize [4, joinedup]
					
					\item  \emph{\m{x = z}.}  \qquad
					In this case, \m{ϕ^x_y = ∀z \, ψ = ϕ}. Since \m{y} does not occur free in \m{ϕ}, \m{x} is substitutable for \m{y} in \m{ϕ^x_y}. Moreover, \m{(ϕ^x_y)^y_x = (∀z \, ψ)^y_x = ∀z \, ψ = ϕ} since \m{y} does not occur at all in \m{ϕ}.

					\item  \emph{\m{x ≠ z}.}  \qquad
					In this case, \m{ϕ^x_y = ∀z \, ψ^x_y}. Using the induction hypothesis, \m{x} is substitutable for \m{y} in \m{ψ^x_y}. Moreover, since \m{z} does not occur in \m{x}, so \m{x} is substitutable for \m{y} in \m{ϕ^x_y}. Moreover, \m{ϕ^x_y = ∀z \, ψ^x_y}, so by the induction hypothesis, \m{(ϕ^x_y)^y_x = (∀z \, ψ^x_y)^y_x = ∀z \, (ψ^x_y)^y_x = ∀z \, ψ = ϕ}.
				\stopitemize
		\stopitemize
	\stopsolution


	\startexercise [title={Enderton, 2.4.12}]
		Show that any consistent set \m{Γ} of formulas can be extended to a consistent set \m{Δ} having the property that for any formula \m{α}, either \m{α ∈ Δ} or \m{(¬α) ∈ Δ}.

		(Assume that the language is countable. Do not use the compactness theorem of sentential logic.)
	\stopexercise
	\startsolution
		Since the language is countable, let \m{\bcrl[α_1, α_2, …]} an enumeration of all wffs of the language. Define by recursion (on the natural numbers)
		\startformula  \startalign[n=3]
			\NC  Δ_0  =  \NC  Γ ,  \NR
			\NC  Δ_{n+1}  =  \NC
				\startcases
					\NC  Δ_n ∪ \bcrl[α_{n+1}]  \NC  if this is consistent,  \NR
					\NC  Δ_n ∪ \bcrl[¬α_{n+1}]  \NC  otherwise .  \NR
				\stopcases  \NR
		\stopalign  \stopformula
		Let \m{Δ = \lim_{n → ∞} Δ_n = ⋃_{n ∈ ℕ} Δ_n}. Clearly, for any formula \m{α}, either \m{α ∈ Δ} or \m{(¬α) ∈ Δ} since \m{α = α_n} for some \m{n ∈ ℕ}. Notice that the procedure ensures that all finite subsets of \m{Δ} are consistent, since every subset of \m{Δ_n} is consistent for every \m{n ∈ ℕ}.

		Therefore we just have to prove that \m{Δ} is consistent. Suppose not. Say \m{Δ ⊢ ⊥}, where \m{⊥} is some unsatisfiable, refutable formula like \m{¬∀x x=x} or \m{β ∧ ¬β}. Since \m{Δ ⊢ ⊥}, there exists a deduction for \m{⊥}. Let \m{Σ} the wffs in the deduction that are in \m{Δ}. Now, \m{Σ} is inconsistent since it produces \m{⊥}, and is a finite subset of \m{Δ} since deductions are necessarily finite. This gives us an inconsistent finite subset of \m{Δ}, which is a contradiction. Hence \m{Δ} must be consistent.
	\stopsolution


	\startexercise [title={Enderton, 2.5.2}]
		Prove the equivalence of parts (a) and (b) of the completeness theorem.

		\emph{Suggestion}: \m{Γ ⊨ ϕ} iff \m{Γ ∪ ¬ϕ} is unsatisfiable. And \m{Δ} is satisfiable iff \m{Δ ⊨ ⊥}, where \m{⊥} is some unsatisfiable, refutable formula like \m{¬∀x x=x}.

		\emph{Remark}: Similarly, the soundness theorem is equivalent to the statement that every satisfiable set of formulas is consistent.
	\stopexercise
	\startsolution
		First we prove the suggestion. That is, we prove that \m{Γ ⊨ ϕ} iff \m{Γ ∪ ¬ϕ} is unsatisfiable. This is equivalent to proving \m{Γ ⊭ ϕ} iff \m{Γ ∪ ¬ϕ} is satisfiable.

		\startlemma
			\m{Γ ⊨ ϕ} iff \m{Γ ∪ \bcrl[¬ϕ]} is unsatisfiable.
		\stoplemma
		\startproof
			\emph{Notation}: In what follows, we use \quotation{\m{∃ \tuple[𝔄, s] \ P}} to denote \quotation{there exists a structure \m{𝔄} and a satisfaction function \m{s:V → \abs[𝔄]} such that \m{P}}. We shall also write \m{\logicimpl[𝔄][s]{Γ}} to mean that \m{\logicimpl[𝔄][s]{γ}} for every \m{γ ∈ Γ}.
			
			\startformula  \startalign[n=3]
				\NC  Γ ⊭ ϕ  ⟺  \NC  ∃ \tuple[𝔄, s] (\logicimpl[𝔄][s]{Γ}  \text{ and }  \nlogicimpl[𝔄][s]{ϕ})  \NC  [\text{Definition of satisfiability}]  \NR
				\NC  ⟺  \NC  ∃ \tuple[𝔄, s] (\logicimpl[𝔄][s]{Γ}  \text{ and }  \logicimpl[𝔄][s]{¬ϕ})  \NC  [\text{Definition of satisfaction function}]  \NR
				\NC  ⟺  \NC  ∃ \tuple[𝔄, s] (\logicimpl[𝔄][s]{Γ ∪ \bcrl[¬ϕ]}  \NR
				\NC  ⟺  \NC  Γ ∪ \bcrl[¬ϕ] \text{ is satisfiable} .  \NC  [\text{Definition of satisfiability}]  \NR
			\stopalign  \stopformula
			Moreover, \m{Δ} is satisfiable iff \m{Δ ∪ \bcrl[¬⊥]} is satisfiable iff \m{Δ ⊭ ⊥}, since \m{¬⊥} is always satisfiable.
		\stopproof

		Now we prove the equivalence.
		\startitemize [i, joinedup]			
			\item  \bold{((a) ⟹ (b))}  \qquad
				Let \m{Γ} be consistent. Then \m{Γ ⊬ ⊥}, where \m{⊥} is as defined in the problem. Using (a), we have \m{Γ ⊭ ⊥}. By the claim above, \m{Γ} is satisfiable.

			\item  \bold{((b) ⟹ (a))}  \qquad
				Let \m{Γ ⊨ ϕ}. So \m{Γ ∪ \bcrl[¬ϕ]} is unsatisfiable. Using the contrapositive of (b), we have \m{Γ ∪ \bcrl[¬ϕ]} is inconsistent. Using RAA and Rule T, we get \m{Γ ⊢ ϕ}.
		\stopitemize
	\stopsolution
\stopchapter

\stopcomponent

% \startformula  \startalign

% \stopalign  \stopformula
