\startcomponent *

\product  prd-functional-analy_0sis

\startchapter [title={Elementary ideas}]

	\startsection [title={Definitions}]
		\dscr{Compact}  \m{T ∈ 𝓚 ⟺ λ → 0}

		\dscr{Hilbert-Schmidt}  \m{T ∈ 𝓑^2 ⟺ λ ∈ ℓ^2}

		\dscr{Trace-class}  \m{T ∈ 𝓑^1 ⟺ λ ∈ ℓ^1}
	\stopsection

	\startsection [title={Inclusions: \m{𝓓 ⊂ 𝓑^1 ⊂ 𝓑^2 ⊂ 𝓚 ⊂ 𝓑^∞}}]
		\dscr{\m{𝓚 ⊆ 𝓑^∞}}
			(\cite[authoryear][BMC2009], Proposition 4.6) If \m{T} is unbounded, we can find a sequence of unit vectors \m{(e_n)} such that \m{\norm[T e_n] ↗ ∞}. So \m{T e_n} cannot have a convergent subsequence, for if \m{T e_n → x}, then \m{\norm[T e_n] → \norm[x]}.

		\dscr{\m{𝓚 ≠ 𝓑^∞}}
			The identity operator \m{I ∈ 𝓑^∞} is not compact because for the bounded sequence of unit vectors \m{(e_n)}, \m{I e_n = e_n} does not converge as \m{\norm[e_n - e_m] = \sqrt2 \ ∀ n ≠ m}.

		\dscr{\m{𝓑^2 ⊆ 𝓚}}  TODO

		\dscr{\m{𝓑^2 ≠ 𝓚}}
			\m{T: ℓ^2 → ℓ^2, T e_n = \frac{1}{\sqrt{n}} e_n; T ∈ 𝓚 ∖ 𝓑^2}.

		\dscr{\m{𝓑^1 ⊆ 𝓑^2}}  TODO

		\dscr{\m{𝓑^1 ≠ 𝓑^2}}
			\m{T: ℓ^2 → ℓ^2, T e_n = \frac{1}{n} e_n; T ∈ 𝓑^2 ∖ 𝓑^1}.
	\stopsection

	\startsection [title={For \m{T ∈ 𝓑^∞}, \m{\norm[T]_∞ = \sup\bcrl[{\abs[{\inn[{Tx, y}]}] : \norm[x] = 1, \norm[y] = 1}]}}]
		\dscr{(≤)}  Since \m{\norm[Tx]  =  \frac{\norm[Tx]^2}{\norm[Tx]}  =  \frac{\inn[Tx, Tx]}{\norm[Tx]}  =  \inn[Tx, \frac{Tx}{\norm[Tx]}]}, we have \m{\norm[T]_∞  =  \sup\bcrl[{\norm[Tx] : \norm[x] = 1}]  ≤  \sup\bcrl[{\abs[{\inn[{Tx, y}]}] : \norm[x] = 1, \norm[y] = 1}]}.

		% \dscr{(≥)}  On the other hand, \m{\inn[Tx, y]  ≤  \norm[Tx] \norm[y]  ≤  \norm[T]_∞ \norm[x] \norm[y]}, so \m{\sup\bcrl[{\abs[{\inn[{Tx, y}]}] : \norm[x] = 1, \norm[y] = 1}]  ≤  \norm[T]_∞}.    % For some reason, this does not work.
		\dscr{(≥)}  On the other hand, \m{\inn[Tx, y]  ≤  \norm[Tx] \norm[y]  ≤  \norm[T]_∞ \norm[x] \norm[y]}, so \m{\sup\{\abs[{\inn[Tx, y]}] : \norm[x] = 1, \norm[y] = 1 \}  ≤  \norm[T]_∞}.

	\stopsection

	\startsection [title={\m{\norm[P]_∞ ≤ 1}}]
		Since \m{\norm[Px]^2 = \inn[Px, Px] = \inn[P^* P x, x] = \inn[P P x, x] = \inn[P x, x] ≤ \norm[Px] \norm[x]}, we have \m{\norm[P]_∞ ≤ 1}.
	\stopsection

	\startsection [title={Projection operator is compact iff its image is finite dimensional}]
		\dscr{(⟹)}  Let \m{P: H → H} be a projection operator, so that \m{P^2 = P}, or \m{P(P - I) = 0}.

		\dscr{(⟸)}  Since the image is finite dimensional, fix an orthonormal basis \m{e_1, …, e_n} of \m{\im T}.
	\stopsection

\stopchapter



\startchapter [title={Optimization}]
	
	\startsection [title={Duality in optimization is the same as duality in functional analysis}]

		For an various intuitions of duality in optimization, see \goto{MSx223235}[url(https://math.stackexchange.com/questions/223235/please-explain-the-intuition-behind-the-dual-problem-in-optimization)].

		Let \m{X} and \m{Y} be Banach spaces, and \m{X^*} and \m{Y^*} be their (algebraic?) duals.
		Consider the two problems, with \m{ϕ_0, y_0} fixed. Here \m{(⋅, ⋅)} denotes the canonical duality pairing.

		\startcolumns [n=2]
		
		\startformula \startalign
			\NC  \max  \qquad  \NC  (ϕ_0, x)  \NR
			\NC  \comment{\text{(Primal)}}  \qquad  \text{s.t.}  \qquad  \NC  T x ≤ y_0  \NR
			\NC  \NC  \ \ \  x ≥ 0  \NR
		\stopalign \stopformula
		\startformula \startalign
			\NC  \min  \qquad  \NC  (ψ, y_0)  \NR
			\NC  \comment{\text{(Dual)}}  \qquad  \text{s.t.}  \qquad  \NC  T^* ψ ≥ ϕ_0  \NR
			\NC  \NC  \quad  ψ ≥ 0  \NR
		\stopalign \stopformula
		\stopcolumns
		
		See the following diagram for more details.
		
		\starttikzcd [sep=tiny]
			\NC  x \arrow[rrrrrrrr, darkblue, mapsto, "T"]  \NC  \NC  \NC  \NC  \NC  \NC  \NC  \NC  T x  \NR
			x ∈  \NC  X \arrow[rrrrrrrr, darkblue, "T"]  \arrow[ddd, middlegreen, dash]  \NC  \NC  \NC  \NC  \NC  \NC  \NC  \NC  y_0 \arrow[ddd, middlegreen, dash]  \NC  ∋ T x, y_0  \NR
			\NR  \NR
			ϕ_0, T^* ψ ∈  \NC  X^*  \NC  \NC  \NC  \NC  \NC  \NC  \NC  \NC  y_0^*  \arrow[llllllll, red, "T^*"]  \NC  ∋ ψ  \NR
			\NC  T^* ψ  \NC  \NC  \NC  \NC  \NC  \NC  \NC  \NC  ψ  \arrow[llllllll, red, mapsto, "T^*"]  \NR
		\stoptikzcd
	\stopsection
\stopchapter
