\startcomponent *

\product  prd-research


\startchapter [title={Friedlin-Wentzell theorem}, reference=chp:Friedlin-Wentzell-theorem]


\stopchapter


\startchapter [title={Friedlin-Wentzell theorem for anticipating initial condition with extension of filtration}, reference=chp:FW-anticipating-IC-extension-filtration]
	
	Our aim is to formulate a large deviations principle for an SDE with anticipating initial conditions. We start of with a very simple case
	\startformula
		X_t^ε = W_T + \sqrt{ε} ∫_0^t σ(X_t^ε) \d W_t ,
	\stopformula
	where \m{t ∈ [0, T]} for some \m{T < ∞}, and conditions on \m{σ} shall be imposed as necessary.

	We shall look at the method of enlargement of filtration by \cite[short][Itô1978]. We denote the enlarged filtration by \m{\tilde{ℱ}_t = ℱ_t ∨ σ(W_T)}. For \m{t ∈ [0, T]}, define the process \m{A_t = ∫_0^t \frac{W_T - W_u}{T - u} \d u}. Then \m{W_t = \tilde{W}_t + A_t}, where \m{\tilde{W}_⋅} is a Wiener process w.r.t. \m{\tilde{ℱ}_⋅}.

	Using this, we write our original SDE as
	\startformula
		X_t^ε = W_T + \sqrt{ε} ∫_0^t σ(X_t^ε) \d \tilde{W}_t + \sqrt{ε} ∫_0^t σ(X_t^ε) \frac{W_T - W_s}{T - s} \d s .
	\stopformula
	Now, let \m{Y_t^ε = \sqrt{ε} (W_T - W_t)}. Then \m{X_t^ε} is given by
	\startformula
		X_t^ε = W_T + \sqrt{ε} ∫_0^t σ(X_t^ε) \d \tilde{W}_t + ∫_0^t σ(X_t^ε) \frac{Y_s^ε}{T - s} \d s .
	\stopformula

	Moreover,
	\startformula  \startalign
		\NC  Y_t^ε  \NC  =  \sqrt{ε} W_T - \sqrt{ε} W_t  \NR
		\NC  \NC  =  \sqrt{ε} W_T - \sqrt{ε} \brnd[\tilde{W}_t + ∫_0^t \frac{W_T - W_s}{T - s} \d s]  \NR
		\NC  \NC  =  \sqrt{ε} W_T - \sqrt{ε} \tilde{W}_t - ∫_0^t \frac{Y_s^ε}{T - s} \d s  \NR
	\stopalign  \stopformula

	Therefore, we have the joint process
	\startformula
		\startmatrix
			\NC A \NC B \NC C \NR
			\NC a \NC b \NC c \NR
		\stopmatrix
	\stopformula

	\startsection [title={\m{\tilde{W}_⋅} is a Wiener process}]

		% We shall use  throughtout the text.
		
		We show that \m{(\tilde{W}_t)} is a \m{(\tilde{ℱ}_t)}-martingale with quadratic variation \m{t}. Then by Lévy's Characterization of Wiener process, we obtain that \m{\tilde{W}_⋅} is a Wiener process.

		\startsubsection [title={\m{\tilde{W}_⋅} is a \m{\tilde{ℱ}_⋅}-martingale}]
			Before we jump into the main result, we prove some lemmas.

			\startlemma [reference={lemma:equality-of-σalgebras}]
				The σ-algebras \m{ℱ_s ∨ σ(W_T)} and \m{ℱ_s ∨ σ(W_T - W_s)} are the same.
			\stoplemma

			\startproof
				TODO
			\stopproof

			\startlemma [reference={lemma:conditional-expectation-of-W_t-W_s}]
				For \m{0 ≤ s ≤ t ≤ T}, we have
				\startformula
					𝔼(W_t - W_s ∣ W_T - W_s) = \frac{t - s}{T - s} (W_T - W_s) .
				\stopformula
			\stoplemma

			\startproof
				We partition the interval \m{[0, T]} into \m{n = k n_0} equal parts, where \m{n_0 = \brnd[\min{\bcrl[s, t - s, T - t]}]^{-1}} and \m{k ∈ ℕ}. Let \m{n_s = s \frac{n}{T}} and \m{n_t = t \frac{n}{T}}. That is, the partition is
				\startformula
					P = \bcrl[0, \frac{T}{n}, …, \frac{n_s T}{n} = s, …, \frac{n_t T}{n} = t, …, \frac{(n - 1)T}{n}, T] .
				\stopformula

				Let \m{Δ_i W = W_{\frac{(i + 1) T}{n}} - W_{\frac{i T}{n}}}.

				Firstly, note that the \m{Δ_i W}s are independent and identically distributed from the definition of Wiener process. Now, using the linearity of conditional expectation, we have
				\startformula  \startalign
					\NC  𝔼(W_t - W_s ∣ W_T - W_s)  =  \NC  𝔼 \brnd[∑_{i = n_s}^{n_t - 1} Δ_i W ∣ ∑_{i = n_s}^{n_t - 1} Δ_i W]  \NR
					\NC  =  \NC  ∑_{i = n_s}^{n_t - 1} 𝔼 \brnd[Δ_i W ∣ ∑_{i = n_s}^{n - 1} Δ_i W]  \NR
					\NC  =  \NC  ∑_{i = n_s}^{n_t - 1} \frac{1}{n - n_s} ∑_{i = n_s}^{n - 1} 𝔼 \brnd[Δ_i W ∣ ∑_{i = n_s}^{n - 1} Δ_i W]  \NR
					\NC  =  \NC  ∑_{i = n_s}^{n_t - 1} \frac{1}{n - n_s}  𝔼 \brnd[∑_{i = n_s}^{n - 1} Δ_i W ∣ ∑_{i = n_s}^{n - 1} Δ_i W]  \NR
					\NC  =  \NC  ∑_{i = n_s}^{n_t - 1} \frac{1}{n - n_s}  ∑_{i = n_s}^{n - 1} Δ_i W  \NR
					\NC  =  \NC  ∑_{i = n_s}^{n_t - 1} \frac{1}{n - n_s}  (W_T - W_s)  \NR
					\NC  =  \NC  \frac{n_t - n_s}{n - n_s}  (W_T - W_s)  \quad  =  \quad  \frac{t - s}{T - s}  (W_T - W_s) . \NR
				\stopalign  \stopformula
			\stopproof


			\startproposition
				\m{\tilde{W}_⋅} is a \m{\tilde{ℱ}_⋅}-martingale.
			\stopproposition

			\startproof
				Let \m{0 ≤ s ≤ t ≤ T}. Then
				\startformula
					\tilde{W}_t - \tilde{W}_s
					=  (W_t - W_s) - ∫_s^t \frac{W_T - W_u}{T - u} \d u
					=  (W_t - W_s) - ∫_s^t \brnd[\frac{W_T - W_s}{T - u} - \frac{W_u - W_s}{T - u}] \d u .
				\stopformula

				Moreover, since \m{W_t - W_s} is independent of \m{ℱ_s} for every \m{t ≥ s}, using lemmas \in[lemma:equality-of-σalgebras] and \in[lemma:conditional-expectation-of-W_t-W_s], we get
				\startformula
					𝔼(W_t - W_s ∣ \tilde{ℱ}_s)
					=  𝔼 \brnd[W_t - W_s ∣ ℱ_s ∨ σ(W_T - W_s)]
					=  𝔼 \brnd[W_t - W_s ∣ W_T - W_s] = \frac{t - s}{T - s} (W_T - W_s) .
				\stopformula

				Therefore, using the fact that \m{W_T} and \m{W_s} are \m{\tilde{ℱ}_s}-measurable with conditional Fubini's theorem, we get
				\startformula  \startalign
					\NC  𝔼(\tilde{W}_t - \tilde{W}_s ∣ \tilde{ℱ}_s)
					     =  \NC  𝔼 \brnd[W_t - W_s ∣ \tilde{ℱ}_s] - ∫_s^t \brnd[\frac{W_T - W_s}{T - u} - \frac{𝔼 \brnd[W_u - W_s ∣ \tilde{ℱ}_s]}{T - u}] \d u  \NR
					\NC  =  \NC  \frac{t - s}{T - s} (W_T - W_s) - ∫_s^t \brnd[\frac{W_T - W_s}{T - u} - \frac{u - s}{T - s} \frac{W_T - W_s}{T - u}] \d u  \NR
					\NC  =  \NC  \frac{t - s}{T - s} (W_T - W_s) - ∫_s^t \frac{W_T - W_s}{T - s} \d u  \NR
					\NC  =  \NC  \frac{t - s}{T - s} (W_T - W_s) - \frac{t - s}{T - s} (W_T - W_s)  \quad  =  \quad  0 . \NR
				\stopalign  \stopformula

				Now, since \m{\tilde{W}_s} is \m{\tilde{ℱ}_s}-measurable, \m{𝔼(\tilde{W}_t ∣ \tilde{ℱ}_s) = 𝔼(\tilde{W}_t - \tilde{W}_s ∣ \tilde{ℱ}_s) + \tilde{W}_s = \tilde{W}_s}.
			\stopproof

		\stopsubsection
	\stopsection

\stopchapter

\stopcomponent

% \startformula  \startalign
	
% \stopalign  \stopformula
