\startcomponent *

\product  prd-analysis


\startchapter [title={Giuseppe Da Prato}]
	
	\startproposition [title={Proposition 1.2 in the book}, reference=prp:1d-Gaussian-measure]
		Let \m{a ∈ ℝ}, \m{λ > 0}, and \m{μ = N_{a, Q}}. Then
		\startitemize [i, joinedup]
			\item[1d-mean]  \m{∫_ℝ x N_{a, λ}(\d x) = a},
			\item[1d-cov]  \m{∫_ℝ (x - a)^2 N_{a, λ}(\d x) = λ}, and
			\item[1d-Fourier]  \m{\widehat{N_{a, λ}}(h) := ∫_ℝ e^{𝚤 h x} N_{a, λ}(\d x) = e^{𝚤 a h - \frac12 λ h^2}}, \m{h ∈ ℝ}.
		\stopitemize
	\stopproposition
	\startproof
		\startitemize [i]
			\item  \m{∫_ℝ x N_{a, λ}(\d x) = a},
			\item  \m{∫_ℝ (x - a)^2 N_{a, λ}(\d x) = λ}, and
			\item  \m{\widehat{N_{a, λ}}(h) := ∫_ℝ e^{𝚤 h x} N_{a, λ}(\d x) = e^{𝚤 a h - \frac12 λ h^2}}, \m{h ∈ ℝ}.
		\stopitemize
	\stopproof


	\startproposition [title={Proposition 1.3 in the book}]
		Let \m{H ≃ ℝ^d} \m{a ∈ H}, \m{Q ∈ L_+(H)}, and \m{μ = N_{a, Q}}. Then
		\startitemize [i, joinedup]
			\item[fd-mean]  \m{∫_H x N_{a, Q}(\d x) = a},
			\item[fd-cov]  \m{∫_H \inn[y, x - a] \inn[z, x - a] N_{a, Q}(\d x) = \inn[Qy, z]}, and
			\item[fd-Fourier]  \m{\widehat{N_{a, Q}}(h) := ∫_H e^{𝚤 \inn[h, x]} N_{a, Q}(\d x) = e^{𝚤 \inn[a, h] - \frac12 \inn[Qh, h]}}, \m{h ∈ H}.
		\stopitemize
 	\stopproposition
	\startproof
		All indices vary from \m{1} to \m{d}.
		\startitemize [i]

			\item  \m{∫_H x N_{a, Q}(\d x) = a}.
				\startformula  \startalign[n=3]
					\NC  ∫_H x N_{a, Q}(\d x)  =  \NC  ∫_H x ⨉_j N_{a_j, λ_j} \brnd[⨉_i \d x_i]  \NR
					\NC  =  \NC  ∫_H ∑_k(x_k e_k) ∏_j N_{a_j, λ_j}(\d x_j)  \NC  [\text{Fubini}]  \NR
					\NC  =  \NC  ∑_k \brnd[∫_ℝ ⋯ ∫_ℝ x_k ∏_j N_{a_j, λ_j}(\d x_j)] e_k  \qquad  \NC  [\text{Fubini}]  \NR
					\NC  =  \NC  ∑_k ∫_ℝ x_k N_{a_k, λ_k}(\d x_k) e_k  \NC  \bsqr[∫_ℝ N_{a_j, λ_j}(\d x_j) = 1 \ ∀j]  \NR
					\NC  =  \NC  ∑_k a_k e_k  \quad  =  a .  \NC  [\text{Proposition \in[prp:1d-Gaussian-measure](\in[1d-mean])}]  \NR
				\stopalign  \stopformula

			\item  \m{∫_H \inn[y, x - a] \inn[z, x - a] N_{a, Q}(\d x) = \inn[Qy, z]}.
				\startformula  \startalign
					\NC     \NC  ∫_H \inn[y, x - a] \inn[z, x - a] N_{a, Q}(\d x)  \NR
					\NC  =  \NC  ∫_H ∑_k y_k (x_k - a_k) ∑_l z_l (x_l - a_l) ∏_j N_{a_j, λ_j}(\d x_j)  \NR
					\NC  =  \NC  ∑_k ∑_l y_k z_l ∫_H (x_k - a_k) (x_l - a_l) ∏_j N_{a_j, λ_j}(\d x_j)  \NR
					\NC  =  \NC  ∑_k y_k z_k ∫_H (x_k - a_k)^2 ∏_j N_{a_j, λ_j}(\d x_j)  +  ∑_{i ≠ j} y_k z_l ∫_H (x_k - a_k) (x_l - a_l) ∏_j N_{a_j, λ_j}(\d x_j)  \NR
					\NC  =  \NC  ∑_k y_k z_k ∫_ℝ (x_k - a_k)^2 N_{a_k, λ_k}(\d x_k)  +  ∑_{i ≠ j} y_k z_l ∫_ℝ (x_k - a_k) N_{a_k, λ_k} (\d x_k) ∫_ℝ (x_l - a_l) N_{a_l, λ_l}(\d x_l)  \NR
					\NC  =  \NC  ∑_k y_k z_k λ_k  \quad = \inn[Qy, z] .  \qquad  [\text{Proposition \in[prp:1d-Gaussian-measure](\in[1d-cov])}]  \NR
				\stopalign  \stopformula

			\item  \m{\widehat{N_{a, Q}}(h) := ∫_H e^{𝚤 \inn[h, x]} N_{a, Q}(\d x) = e^{𝚤 \inn[a, h] - \frac12 \inn[Qh, h]}}, \m{h ∈ H}.
				\startformula  \startalign[n=3]
					\NC  \widehat{N_{a, Q}}(h)  =  \NC  ∫_H e^{𝚤 \inn[h, x]} N_{a, Q}(\d x)  \NR
					\NC  =  \NC  ∫_H e^{𝚤 ∑_k h_k x_k} ∏_j N_{a_j, λ_j}(\d x_j)  \NR
					\NC  =  \NC  ∫_H ∏_k e^{𝚤 h_k x_k} ∏_j N_{a_j, λ_j}(\d x_j)  \NR
					\NC  =  \NC  ∏_k ∫_H e^{𝚤 h_k x_k} N_{a_k, λ_k}(\d x_k)  \NC  [\text{\comment{Why? This is false in general!}}]  \NR
					\NC  =  \NC  ∏_k e^{𝚤 a_k h_k - \frac12 λ_k h_k^2}  \qquad  \NC  [\text{Proposition \in[prp:1d-Gaussian-measure](\in[1d-Fourier])}]  \NR
					\NC  =  \NC  e^{𝚤 ∑_k a_k h_k - \frac12 ∑_k λ_k h_k h_k}  \NR
					\NC  =  \NC  e^{𝚤 \inn[a, h] - \frac12 \inn[Qh, h]} .  \NR
				\stopalign  \stopformula
		\stopitemize
	\stopproof

	
	\startlemma [title={The operator \m{1 - εQ} in Page 14}]
		The operator \m{1 - εQ} is invertible with a finite positive determinant. Moreover, the inverse is bounded.
	\stoplemma
	\startproof
		Firstly, note that for the operator
		
		Since \m{ε < \frac1λ_1}, we have \m{1 > ε λ_1}. Combining this with \m{λ_1 ≥ λ_2 ≥ ⋯}, we get \m{1 > ε λ_1 ≥ ε λ_2 ≥ ⋯}, which futher implies \m{∞ > \inv[(1 - ε λ_1)] ≥ \inv[(1 - ε λ_2)] ≥ ⋯}. Therefore the operator \m{1 - εQ} is invertible and \m{\inv[(1 - εQ)]} is bounded.

		Now, since \m{(1 - ε Q) e_k = (1 - ε λ_k) e_k} for every \m{k ∈ ℕ}, we have \m{\inv[(1 - ε Q)] e_k = \inv[(1 - ε λ_k)] e_k} for every \m{k ∈ ℕ}. This gives us for every \m{x ∈ H},
		\startformula
			\inv[(1 - εQ)]x = ∑_{k = 1}^∞ \frac{1}{1 - ε λ_k} \inn[x, e_k] e_k .
		\stopformula

		\comment{Why is the infinite product finite and positive?}
	\stopproof

	\startlemma [title={Lemma for Proposition 1.13}]
		\startformula
			∫_ℝ e^{\frac{ε}{2} \abs[x]^2} N_{a, λ}(\d x)  =  \frac{e^{\frac{ε a^2}{2 (1 - ε λ)}}}{\sqrt{1 - ε λ}} .
		\stopformula
	\stoplemma
	\startproof
		First, note that using completion of squares, we get
		\startformula  \startalign
			\NC  \frac{ε}{2} x^2 - \frac{1}{2 λ} (x - a)^2  =  \NC  - \frac{1}{2 λ} \brnd[(1 - ε λ) x^2 - 2 a x + a^2]  \NR
			\NC  =  \NC  - \frac{1 - ε λ}{2 λ} \bsqr[x^2 - 2 \frac{a}{1 - ε λ} x + \frac{a^2}{(1 - ε λ)^2} - \frac{a^2}{(1 - ε λ)^2} + \frac{a^2}{1 - ε λ}]  \NR
			\NC  =  \NC  - \frac{1 - ε λ}{2 λ} \bsqr[{\brnd[x - \frac{a}{1 - ε λ}]^2} - \frac{ε λ a^2}{(1 - ε λ)^2}]  \NR
			\NC  =  \NC  - \frac12 \frac{\brnd[x - \frac{a}{1 - ε λ}]^2}{\frac{λ}{1 - ε λ}} + \frac{ε a^2}{2 (1 - ε λ)} .  \NR
		\stopalign  \stopformula
		Therefore,
		\startformula \startalign
			\NC  ∫_ℝ e^{\frac{ε}{2} x} N_{a, λ}(\d x)
			     =  \NC  ∫_ℝ e^{\frac{ε}{2} x^2} \frac{1}{\sqrt{2 π λ}} e^{-\frac{1}{2 λ} \brnd[x - a]^2} \d x  \NR
			\NC  =  \NC  \frac{e^{\frac{ε a^2}{2 (1 - ε λ)}}}{\sqrt{1 - ε λ}}
			             ∫_ℝ \frac{1}{\sqrt{2 π \frac{λ}{1 - ε λ}}}
			                 e^{- \frac12 \frac{\brnd[x - \frac{a}{1 - ε λ}]^2}{\frac{λ}{1 - ε λ}}} \d x  \NR
			\NC  =  \NC  \frac{e^{\frac{ε a^2}{2 (1 - ε λ)}}}{\sqrt{1 - ε λ}} .  \NR
		\stopalign \stopformula
	\stopproof


	\startproposition [title={Hint for Exercise 1.14}]
		\startformula
			J_m  =  2^m F^{(m)}(0), \ m ∈ ℕ ;  \qquad \text{where} \quad  F(ε) = ∫_H e^{\frac{ε}{2} \abs[x]^2} μ(\d x) , \ ε > 0.
		\stopformula
	\stopproposition
	\startproof
		\startformula \startalign
			\NC  F(ε) =  \NC  ∫_H e^{\frac{ε}{2} \abs[x]^2} μ(\d x)  \NR
			\NC  =  \NC  ∫_H ∑_{m = 0}^∞ \frac{1}{m!} \brnd[\frac{ε}{2} {\abs[x]^2}]^m μ(\d x)  \NR
			\NC  =  \NC  ∑_{m = 0}^∞ \frac{ε^m}{2^m m!}  ∫_H \abs[x]^{2m} μ(\d x)  \qquad  [\text{Monotone convergence theorem}]  \NR
			\NC  =  \NC  ∑_{m = 0}^∞ \frac{J_m}{2^m m!} ε^m  \NR
		\stopalign \stopformula
	\stopproof

\stopchapter




\startchapter [title={Abstract Wiener spaces}, reference=chp:abstract-Wiener-spaces]

\stopchapter


\startchapter [title={White noise distribution theory}, reference=chp:white-noise-distribution-theory]

	\startsection [title={Characterization theorem}]

		Importance and history.

		In the following, \m{F} is defined on \m{S_ℂ}, and \m{F(z ξ + η)} is entire \m{∀ z ∈ ℂ}.

		\startformula \startalign [
			n=9,
			align={middle, middle, middle, middle, middle, middle, middle, middle, middle}]
			\NC  (S)_β  \NC ⊂ \NC  (S)  \NC ⊂ \NC  (L^2)  \NC ⊂ \NC  (S)^*  \NC ⊂ \NC  (S)_β^*  \NR
			\NC  \NC \NC  S  \NC ⊂ \NC  L^2  \NC ⊂ \NC  S'  \NC \NC  \NR
		\stopalign \stopformula

	\stopsection

\stopchapter

\stopcomponent
